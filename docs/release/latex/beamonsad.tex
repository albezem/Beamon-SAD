%% Generated by Sphinx.
\def\sphinxdocclass{report}
\documentclass[letterpaper,10pt,english]{sphinxmanual}
\ifdefined\pdfpxdimen
   \let\sphinxpxdimen\pdfpxdimen\else\newdimen\sphinxpxdimen
\fi \sphinxpxdimen=.75bp\relax

\PassOptionsToPackage{warn}{textcomp}
\usepackage[utf8]{inputenc}
\ifdefined\DeclareUnicodeCharacter
% support both utf8 and utf8x syntaxes
  \ifdefined\DeclareUnicodeCharacterAsOptional
    \def\sphinxDUC#1{\DeclareUnicodeCharacter{"#1}}
  \else
    \let\sphinxDUC\DeclareUnicodeCharacter
  \fi
  \sphinxDUC{00A0}{\nobreakspace}
  \sphinxDUC{2500}{\sphinxunichar{2500}}
  \sphinxDUC{2502}{\sphinxunichar{2502}}
  \sphinxDUC{2514}{\sphinxunichar{2514}}
  \sphinxDUC{251C}{\sphinxunichar{251C}}
  \sphinxDUC{2572}{\textbackslash}
\fi
\usepackage{cmap}
\usepackage[T1]{fontenc}
\usepackage{amsmath,amssymb,amstext}
\usepackage{babel}



\usepackage{times}
\expandafter\ifx\csname T@LGR\endcsname\relax
\else
% LGR was declared as font encoding
  \substitutefont{LGR}{\rmdefault}{cmr}
  \substitutefont{LGR}{\sfdefault}{cmss}
  \substitutefont{LGR}{\ttdefault}{cmtt}
\fi
\expandafter\ifx\csname T@X2\endcsname\relax
  \expandafter\ifx\csname T@T2A\endcsname\relax
  \else
  % T2A was declared as font encoding
    \substitutefont{T2A}{\rmdefault}{cmr}
    \substitutefont{T2A}{\sfdefault}{cmss}
    \substitutefont{T2A}{\ttdefault}{cmtt}
  \fi
\else
% X2 was declared as font encoding
  \substitutefont{X2}{\rmdefault}{cmr}
  \substitutefont{X2}{\sfdefault}{cmss}
  \substitutefont{X2}{\ttdefault}{cmtt}
\fi


\usepackage[Bjarne]{fncychap}
\usepackage{sphinx}

\fvset{fontsize=\small}
\usepackage{geometry}


% Include hyperref last.
\usepackage{hyperref}
% Fix anchor placement for figures with captions.
\usepackage{hypcap}% it must be loaded after hyperref.
% Set up styles of URL: it should be placed after hyperref.
\urlstyle{same}


\usepackage{sphinxmessages}
\setcounter{tocdepth}{4}
\setcounter{secnumdepth}{4}


\title{Beamon SAD}
\date{Mar 19, 2021}
\release{alpha}
\author{Maher Albezem}
\newcommand{\sphinxlogo}{\vbox{}}
\renewcommand{\releasename}{Release}
\makeindex
\begin{document}

\pagestyle{empty}
\sphinxmaketitle
\pagestyle{plain}
\sphinxtableofcontents
\pagestyle{normal}
\phantomsection\label{\detokenize{index::doc}}


Beamon is an open\sphinxhyphen{}source\sphinxhyphen{}based development in the institute of structural analysis and
lightweight design at RWTH University in Aachen. Beamon Static Analysis Design (SAD) is mainly
a graphical finite element program for modeling static structural systems and simulating those using
the direct stiffness method.

\noindent\sphinxincludegraphics{{intro}.png}


\chapter{Beamon In A Glance}
\label{\detokenize{index:beamon-in-a-glance}}
\noindent\sphinxincludegraphics{{intro2}.jpg}

Users or specifically designers are capable of defining node coordinates and structure boundary conditions (Structural support).

Elements are connection information between nodes. In this context, a static model should build a
\sphinxhref{https://en.wikipedia.org/wiki/Graph\_(discrete\_mathematics)}{Graph}.

Each element has modifiable local system orientation and properties, profiles.

Designers can apply external forces on structures such as, nodal forces or constant element forces.

Geometries can be exported and imported as modifiable text files
(\sphinxhref{https://de.wikipedia.org/wiki/American\_Standard\_Code\_for\_Information\_Interchange}{ASCII}\sphinxhyphen{}Symbols).

Simulation results summed up in section forces and displacements for each element, and support forces and displacements
for each node can be viewed in tables and exported as Excel or CSV files.

Many models can be viewed simultaneously in separate windows. A project overview shows a report for each model.


\chapter{Documentations Contents}
\label{\detokenize{index:documentations-contents}}

\section{Getting Started}
\label{\detokenize{getting_started:getting-started}}\label{\detokenize{getting_started::doc}}

\subsection{Installation}
\label{\detokenize{getting_started:installation}}
Beamon SAD can be used as Program that can be installed or run directly as a python package.


\subsubsection{Windows Installer}
\label{\detokenize{getting_started:windows-installer}}
Executing the Beamons installer in \sphinxstylestrong{release} directory shows a Wizard to help user threw the installation process.
Make sure to specify an installation location with access rights.


\subsubsection{Windows Portable Version}
\label{\detokenize{getting_started:windows-portable-version}}
Run Beamon executable files from \sphinxstylestrong{releases} directory.


\subsubsection{Python Package (for developers and testers)}
\label{\detokenize{getting_started:python-package-for-developers-and-testers}}
Use the package manager \sphinxhref{https://pip.pypa.io/en/stable/}{pip} to install requirements from requirements.txt file.


\subsection{Starting Beamon}
\label{\detokenize{getting_started:starting-beamon}}

\subsubsection{How To Use \sphinxstyleemphasis{beamon} Package}
\label{\detokenize{getting_started:how-to-use-beamon-package}}
To run Beamon package use the following command in your python environment:

\begin{sphinxVerbatim}[commandchars=\\\{\}]
\PYG{n}{python} \PYG{o}{\PYGZhy{}}\PYG{n}{m} \PYG{n}{beamon}
\end{sphinxVerbatim}

This will execute the user interface of Beamon


\subsubsection{Beamon’s command Line Arguments}
\label{\detokenize{getting_started:beamon-s-command-line-arguments}}
Available command line arguments are:
\begin{enumerate}
\sphinxsetlistlabels{\arabic}{enumi}{enumii}{}{.}%
\item {} 
\sphinxhyphen{}i \textless{}path to geometry file\textgreater{}

\item {} 
\sphinxhyphen{}m model name for the imported geometry file

\item {} 
\sphinxhyphen{}t test mode (lunches all submodules on start)

\item {} 
\sphinxhyphen{}r ram mode (runs database on RAM)

\item {} 
\sphinxhyphen{}d \textless{}path to database file\textgreater{}

\end{enumerate}


\subsection{Using Beamon}
\label{\detokenize{getting_started:using-beamon}}

\subsubsection{Browsing Models}
\label{\detokenize{getting_started:browsing-models}}
A list of project names and their last modification date can be seen in the project overview window.
Model names can be changed when the user double clicks on the model name in the table.

On the right side, there is a report for each selected model showing the number of nodes, elements,  joints,
and how many profiles are used by the model.

The user can import geometry to the model or export existing geometry. Upon importing a geometry file to a
selected model user can choose between appending and overwriting geometry to the model.
Appending geometry means current geometry will be extended.
Overwriting means the current geometry will be erased and replaced with the imported ones.

\noindent\sphinxincludegraphics{{vis_project_browser}.png}


\subsection{Using the Visualizer}
\label{\detokenize{getting_started:using-the-visualizer}}
\noindent\sphinxincludegraphics{{how_to_visualization}.png}

The Visualizer consists of a virtual 3D room, called visualization, to display the structure
and a tree of selectable elements to modify the structure.
To modify nodes for example click on the Nodes element on the left\sphinxhyphen{}hand side section and a table will be displayed.
To see visualization controls click on Visualization \sphinxhyphen{}\textgreater{} Help on the toolbar.


\section{Making A Model}
\label{\detokenize{making_a_model:making-a-model}}\label{\detokenize{making_a_model::doc}}
There are two ways to define Geometry information. Either with an input file (geometry file) or using data
tables in the user interface.


\subsection{Nodes And Boundry Conditions}
\label{\detokenize{making_a_model:nodes-and-boundry-conditions}}\label{\detokenize{making_a_model:nodes-input}}
Nodes are labeled with index numbers and have coordinates in Room.
Each node has 6 degrees of freedom (DOF) defined in order \(u_x, u_y, u_z, \varphi_x, \varphi_y, \varphi_z\)
You can set boundary conditions by locking movement in the desired direction.


\subsubsection{With Geometry File}
\label{\detokenize{making_a_model:with-geometry-file}}
Nodes entry begins with keyword ‘{\color{red}\bfseries{}*}node’.
To lock a movement in a certain direction use 1 and to keep the movement unlocked use 0.
The following syntax diagram shows the syntax for nodes input.

\begin{figure}[htbp]
\centering
\capstart

\noindent\sphinxincludegraphics{{nodeEntry}.png}
\caption{Syntax diagram for nodes entry in geometry file}\label{\detokenize{making_a_model:id21}}\end{figure}

Where dof is defined as:

\begin{figure}[htbp]
\centering
\capstart

\noindent\sphinxincludegraphics{{dof}.png}
\caption{Syntax diagram for dof in nodes entry in geometry file}\label{\detokenize{making_a_model:id22}}\end{figure}

\sphinxstylestrong{Example}:

\begin{sphinxVerbatim}[commandchars=\\\{\}]
\PYG{o}{*}\PYG{n}{node}
\PYG{o}{\PYGZhy{}}\PYG{l+m+mf}{1.0}\PYG{p}{,}\PYG{l+m+mf}{0.0}\PYG{p}{,}\PYG{l+m+mf}{0.0}\PYG{p}{,}\PYG{l+m+mi}{1}\PYG{p}{,}\PYG{l+m+mi}{1}\PYG{p}{,}\PYG{l+m+mi}{1}\PYG{p}{,}\PYG{l+m+mi}{1}\PYG{p}{,}\PYG{l+m+mi}{1}\PYG{p}{,}\PYG{l+m+mi}{1}
\PYG{l+m+mf}{0.0}\PYG{p}{,}\PYG{l+m+mf}{0.0}\PYG{p}{,}\PYG{o}{\PYGZhy{}}\PYG{l+m+mf}{1.0}\PYG{p}{,}\PYG{l+m+mi}{1}\PYG{p}{,}\PYG{l+m+mi}{1}\PYG{p}{,}\PYG{l+m+mi}{1}\PYG{p}{,}\PYG{l+m+mi}{0}\PYG{p}{,}\PYG{l+m+mi}{0}\PYG{p}{,}\PYG{l+m+mi}{0}
\PYG{l+m+mf}{1.0}\PYG{p}{,}\PYG{l+m+mf}{0.0}\PYG{p}{,}\PYG{l+m+mf}{2.0}
\PYG{l+m+mf}{3.0}\PYG{p}{,}\PYG{l+m+mf}{0.0}\PYG{p}{,}\PYG{o}{\PYGZhy{}}\PYG{l+m+mf}{2.0}\PYG{p}{,}\PYG{l+m+mi}{1}\PYG{p}{,}\PYG{l+m+mi}{1}\PYG{p}{,}\PYG{l+m+mi}{1}\PYG{p}{,}\PYG{l+m+mi}{0}\PYG{p}{,}\PYG{l+m+mi}{1}\PYG{p}{,}\PYG{l+m+mi}{1}
\PYG{l+m+mf}{4.0}\PYG{p}{,}\PYG{l+m+mf}{0.0}\PYG{p}{,}\PYG{l+m+mf}{1.0}\PYG{p}{,}\PYG{l+m+mi}{1}\PYG{p}{,}\PYG{l+m+mi}{1}\PYG{p}{,}\PYG{l+m+mi}{1}\PYG{p}{,}\PYG{l+m+mi}{1}\PYG{p}{,}\PYG{l+m+mi}{0}\PYG{p}{,}\PYG{l+m+mi}{1}
\PYG{l+m+mf}{2.0}\PYG{p}{,}\PYG{l+m+mf}{0.0}\PYG{p}{,}\PYG{l+m+mf}{2.0}\PYG{p}{,}\PYG{l+m+mi}{1}\PYG{p}{,}\PYG{l+m+mi}{1}\PYG{p}{,}\PYG{l+m+mi}{1}\PYG{p}{,}\PYG{l+m+mi}{1}\PYG{p}{,}\PYG{l+m+mi}{1}\PYG{p}{,}\PYG{l+m+mi}{0}
\PYG{l+m+mf}{3.0}\PYG{p}{,}\PYG{l+m+mf}{0.0}\PYG{p}{,}\PYG{l+m+mf}{2.0}\PYG{p}{,}\PYG{l+m+mi}{1}\PYG{p}{,}\PYG{l+m+mi}{1}\PYG{p}{,}\PYG{l+m+mi}{1}\PYG{p}{,}\PYG{l+m+mi}{1}\PYG{p}{,}\PYG{l+m+mi}{1}\PYG{p}{,}\PYG{l+m+mi}{1}
\end{sphinxVerbatim}

first node location is (\sphinxhyphen{}1,0,0) and has it’s all dof directions locked. Second node location is (0,0,\sphinxhyphen{}1) and has all
translation dof directions locked.

Notice that the third node has only its coordinates defined. All undefined dof numbers will be substituted
with 0 (unlocked).


\subsubsection{With User Interface}
\label{\detokenize{making_a_model:with-user-interface}}
As in nodes, elements and other input tables
right click on a table row or on an emtpy table to add/remove rows.

Use the checkbox to lock (checked) and
unlock (unchecked) the movement in a certain direction.

\noindent\sphinxincludegraphics{{NodeEntryGUI}.png}


\subsubsection{Nodes and Boundry Conditions Visualization}
\label{\detokenize{making_a_model:nodes-and-boundry-conditions-visualization}}
Nodes are black square points. Orange cones are the boundary conditions (bc) for locking translation in each global direction
in the cartesian coordinate system. Two nested blue cones are bc for locking rotation in each direction.

\noindent\sphinxincludegraphics{{vis_node}.png}


\subsection{Elements}
\label{\detokenize{making_a_model:elements}}
Elements are, in this current development stage, 3D Beams. Each element has an index number and is defined using starting
and ending node index number. Vector \(v\) is used to define local z\sphinxhyphen{}axis. (for further info about element
orientation see {\hyperref[\detokenize{theory::doc}]{\sphinxcrossref{\DUrole{doc}{Theory}}}})


\subsubsection{With Geometry File}
\label{\detokenize{making_a_model:id3}}
Elements entry begins with keyword ‘{\color{red}\bfseries{}*}ßelement’. A starting and ending node index must be given.
Element orientation is optional input and is defined using vector \(\vec{v}\).
If not given default element orientation will be used.
At last optional but necessary profile number should be given.

The following syntax diagram shows the syntax for elements input.

\begin{figure}[htbp]
\centering
\capstart

\noindent\sphinxincludegraphics{{elementEntry}.png}
\caption{Syntax diagram for element entry in geometry file}\label{\detokenize{making_a_model:id23}}\end{figure}

\sphinxstylestrong{Example}:
from nodes example above we can define following elements:

\begin{sphinxVerbatim}[commandchars=\\\{\}]
\PYG{o}{*}\PYG{n}{element}
\PYG{l+m+mi}{1}\PYG{p}{,}\PYG{l+m+mi}{2}\PYG{p}{,}\PYG{l+m+mi}{1}\PYG{p}{,}\PYG{o}{\PYGZhy{}}\PYG{l+m+mi}{1}\PYG{p}{,}\PYG{l+m+mi}{0}\PYG{p}{,}\PYG{l+m+mi}{0}
\PYG{l+m+mi}{2}\PYG{p}{,}\PYG{l+m+mi}{3}\PYG{p}{,}\PYG{p}{,}\PYG{l+m+mi}{0}\PYG{p}{,}\PYG{l+m+mi}{0}\PYG{p}{,}\PYG{l+m+mi}{1}
\PYG{l+m+mi}{3}\PYG{p}{,}\PYG{l+m+mi}{4}\PYG{p}{,}\PYG{l+m+mi}{1}\PYG{p}{,}\PYG{o}{\PYGZhy{}}\PYG{l+m+mi}{1}\PYG{p}{,}\PYG{l+m+mi}{0}\PYG{p}{,}\PYG{l+m+mi}{0}
\end{sphinxVerbatim}

first element is defined with nodes 1 and 2 and has profile 1 and orientation vector (1,\sphinxhyphen{}1,0).
Second element has no profile defined and will be excluded from simulation later.


\subsubsection{With User Interface}
\label{\detokenize{making_a_model:id6}}
Analogous to {\hyperref[\detokenize{making_a_model:nodes-input}]{\sphinxcrossref{\DUrole{std,std-ref}{Nodes And Boundry Conditions}}}}


\subsubsection{Elements Visualization}
\label{\detokenize{making_a_model:elements-visualization}}
Visual example of an element defined with nodes 5 and 6. With local axes orientation different from global axes.

\noindent\sphinxincludegraphics{{vis_element}.png}


\subsection{Nodes Joints}
\label{\detokenize{making_a_model:nodes-joints}}
Joints specify which element degree of freedom should be excluded from force transition between two elements.
In Beamon, the user can change the element’s degrees of freedom.


\subsubsection{With Geometry File}
\label{\detokenize{making_a_model:id7}}
To define joints use the keyword ‘{\color{red}\bfseries{}*}joint’.
Each entry starts with a valid element index number followed by two sets of 6 optional integer numbers.
The following syntax diagram hopefully clarifies syntax for joints.

\begin{figure}[htbp]
\centering
\capstart

\noindent\sphinxincludegraphics{{jointEntry}.png}
\caption{Syntax diagram for joint entry in geometry file}\label{\detokenize{making_a_model:id24}}\end{figure}

\sphinxstylestrong{Example}:
from elements example above we can define the following joints:

\begin{sphinxVerbatim}[commandchars=\\\{\}]
\PYG{o}{*}\PYG{n}{joint}
\PYG{l+m+mi}{1}\PYG{p}{,}\PYG{l+m+mi}{123456}\PYG{p}{,}\PYG{l+m+mi}{123456}
\PYG{l+m+mi}{2}\PYG{p}{,}\PYG{l+m+mi}{123}\PYG{p}{,}\PYG{l+m+mi}{456}
\PYG{l+m+mi}{3}\PYG{p}{,}\PYG{p}{,}\PYG{l+m+mi}{1}
\end{sphinxVerbatim}

The first join entry detaches element 1 of all degrees of freedom from the structure.
This will produce a singular system of equations.
The second entry makes more sense. It detaches \(u_x, u_y, u_z\) degrees of freedom
at the beginning of element 2 and detaches \(\varphi_x, \varphi_y, \varphi_z\) degrees of freedom at the end of
element 2 from the structure. The third entry detaches \(u_x\) at the end of element 3.


\subsubsection{With User Interface}
\label{\detokenize{making_a_model:id10}}
Element degrees of freedom can be seen only if the user has already defined elements.
All degrees of freedom can be changed using the checkboxes as in the figure below.

\noindent\sphinxincludegraphics{{jointEntryGUI}.png}


\subsubsection{Joints Visualization}
\label{\detokenize{making_a_model:joints-visualization}}
Joint nodes can be detected when yellow spheres are displayed near the nodes in each element’s direction.

\noindent\sphinxincludegraphics{{vis_joint}.png}


\subsection{Nodes Loads}
\label{\detokenize{making_a_model:nodes-loads}}
Nodes Loads or Point Load describes the force that acts at a point.
Point loads can be only defined using an existing node index number and the following parameters:

\(\vec{F}= \left( \begin{array}{c} F_x \\ F_y \\ F_z \end{array}\right)\) is Force vector
and
\(\vec{M}= \left( \begin{array}{c} M_x \\ M_y \\ M_z \end{array}\right)\) is momentum vector on the specified node


\subsubsection{With Geometry File}
\label{\detokenize{making_a_model:id11}}
To define node loads use the keyword ‘{\color{red}\bfseries{}*}load’. Each entry starts with a valid node index number followed by 6
floating\sphinxhyphen{}point numbers for force and momentum vector respectively.

The following syntax diagram hopefully clarifies syntax for point load.

\begin{figure}[htbp]
\centering
\capstart

\noindent\sphinxincludegraphics{{loadEntry}.png}
\caption{Syntax diagram for load entry in geometry files}\label{\detokenize{making_a_model:id25}}\end{figure}

\sphinxstylestrong{Example}:
from nodes example above we can define following loads:

\begin{sphinxVerbatim}[commandchars=\\\{\}]
\PYG{o}{*}\PYG{n}{load}
\PYG{l+m+mi}{2}\PYG{p}{,}\PYG{l+m+mi}{10000}\PYG{p}{,}\PYG{l+m+mi}{0}\PYG{p}{,}\PYG{l+m+mi}{0}\PYG{p}{,}\PYG{l+m+mi}{0}\PYG{p}{,}\PYG{l+m+mi}{0}\PYG{p}{,}\PYG{l+m+mi}{0}
\PYG{l+m+mi}{2}\PYG{p}{,} \PYG{l+m+mi}{1}\PYG{p}{,}\PYG{l+m+mi}{1}
\PYG{l+m+mi}{1}\PYG{p}{,} \PYG{l+m+mi}{1}\PYG{p}{,}\PYG{l+m+mi}{1}\PYG{p}{,}\PYG{l+m+mi}{1}\PYG{p}{,}\PYG{l+m+mi}{1}\PYG{p}{,}\PYG{l+m+mi}{1}\PYG{p}{,}\PYG{l+m+mi}{1}
\end{sphinxVerbatim}

Note that loads on node 1 are superposition in the calculation.
The second entry contains only the first two components of the force vector.
In this case, all other components will be assumed to be zero as if ‘2,1,1,0,0,0,0’ where given.


\subsubsection{With User Interface}
\label{\detokenize{making_a_model:id14}}
Analogous to {\hyperref[\detokenize{making_a_model:nodes-input}]{\sphinxcrossref{\DUrole{std,std-ref}{Nodes And Boundry Conditions}}}}


\subsubsection{Nodes Loads Visualization}
\label{\detokenize{making_a_model:nodes-loads-visualization}}
for \(\vec{F}= \left(\begin{array}{c} 1 \\ 1 \\ 0 \end{array}\right)\) and
\(\vec{M}= \left(\begin{array}{c} 0 \\ 1 \\ 1 \end{array}\right)\)

we get the following:

\noindent\sphinxincludegraphics{{vis_loads}.png}


\subsection{Element Loads}
\label{\detokenize{making_a_model:element-loads}}
Element load or distributed load is understood to be a load that is distributed across a connecting element.
In Beamon such loads are constantly distributed across elements.
Element loads can be only defined using an existing element index number and the following four parameters:
* \(q_x\) force in local x\sphinxhyphen{}axis direction
* \(q_y\) force in local y\sphinxhyphen{}axis direction
* \(q_z\) force in local z\sphinxhyphen{}axis direction
* \(q_w\) momentum in local x\sphinxhyphen{}axis direction


\subsubsection{With Geometry File}
\label{\detokenize{making_a_model:id15}}
To define element loads use the keyword ‘{\color{red}\bfseries{}*}lineload’. Each entry starts with a valid node index number followed by 6
floating\sphinxhyphen{}point numbers for force and momentum vector respectively.

The following syntax diagram describes element load entries.

\begin{figure}[htbp]
\centering
\capstart

\noindent\sphinxincludegraphics{{lineloadEntry}.png}
\caption{Syntax Diagram of element loads entry in geometry files}\label{\detokenize{making_a_model:id26}}\end{figure}

\sphinxstylestrong{Example}:

\begin{sphinxVerbatim}[commandchars=\\\{\}]
\PYG{o}{*}\PYG{n}{lineload}
\PYG{l+m+mi}{2}\PYG{p}{,}\PYG{l+m+mi}{0}\PYG{p}{,}\PYG{l+m+mi}{1}\PYG{p}{,}\PYG{o}{\PYGZhy{}}\PYG{l+m+mf}{7.5e+07}\PYG{p}{,}\PYG{l+m+mi}{0}
\PYG{l+m+mi}{1}\PYG{p}{,}\PYG{l+m+mi}{1}\PYG{p}{,}\PYG{l+m+mi}{0}\PYG{p}{,}\PYG{l+m+mi}{0}\PYG{p}{,}\PYG{l+m+mi}{0}
\PYG{l+m+mi}{3}\PYG{p}{,}\PYG{l+m+mi}{0}\PYG{p}{,}\PYG{l+m+mi}{0}\PYG{p}{,}\PYG{l+m+mi}{0}\PYG{p}{,}\PYG{l+m+mi}{1}
\end{sphinxVerbatim}

first element load entry has constant \(q_z=-7.5e+06\) value on element 2.


\subsubsection{With User Interface}
\label{\detokenize{making_a_model:id18}}
Analogous to {\hyperref[\detokenize{making_a_model:nodes-input}]{\sphinxcrossref{\DUrole{std,std-ref}{Nodes And Boundry Conditions}}}}


\subsubsection{Element Loads Visualization}
\label{\detokenize{making_a_model:element-loads-visualization}}
From example above we get following visualization

\noindent\sphinxincludegraphics{{vis_lineloads}.png}


\subsection{Profiles and BeamSize}
\label{\detokenize{making_a_model:profiles-and-beamsize}}
Simulating 3D Beams requires specific element properties (profiles). Each profile must contain following parameters:
\begin{itemize}
\item {} 
\(E\): modulus of elasticity E

\item {} 
\(G\): shear modulus

\item {} 
\(A\): cross section area

\item {} 
\(Iy\): moment of inertia with respect to the local y\sphinxhyphen{}axis

\item {} 
\(Iz\): moment of inertia with respect to the local z\sphinxhyphen{}axis

\item {} 
\(Kv\): St Venant torsional stiffness

\item {} 
\(K\): optional spring stiffness (excluded from simulation momentarily)

\end{itemize}

\sphinxstylestrong{Note}: It is assumed that \(I_{yz}\) is zero.


\subsubsection{With Geometry File}
\label{\detokenize{making_a_model:id19}}
To define profiles use keyword ‘*profile’.
The following syntax diagram shows how each profile entry should be.

\begin{figure}[htbp]
\centering
\capstart

\noindent\sphinxincludegraphics{{profileEntry}.png}
\caption{Syntax diagram for profile entries in geometry files}\label{\detokenize{making_a_model:id27}}\end{figure}

\sphinxstylestrong{Example}:

\begin{sphinxVerbatim}[commandchars=\\\{\}]
\PYG{o}{*}\PYG{n}{profile}
\PYG{l+m+mf}{7e+10}\PYG{p}{,}\PYG{l+m+mf}{3e+10}\PYG{p}{,}\PYG{l+m+mf}{0.25}\PYG{p}{,}\PYG{l+m+mf}{0.00520833}\PYG{p}{,}\PYG{l+m+mf}{0.00520833}\PYG{p}{,}\PYG{l+m+mf}{0.00880208}\PYG{p}{,}\PYG{l+m+mi}{0}
\end{sphinxVerbatim}


\subsubsection{With User Interface}
\label{\detokenize{making_a_model:id20}}
Analogous to {\hyperref[\detokenize{making_a_model:nodes-input}]{\sphinxcrossref{\DUrole{std,std-ref}{Nodes And Boundry Conditions}}}}


\subsubsection{With BeamSize}
\label{\detokenize{making_a_model:with-beamsize}}
Calculating profile values could be tricky, especially if you are dealing with complex profiles.
\sphinxstylestrong{BeamSize} should simplify calculating profile values depending on which geometry you use.

It is worth noting that in the current development only profiles with \(I_{yz}=0\) yield correct simulation results.

Steps to calculate profiles:
\begin{enumerate}
\sphinxsetlistlabels{\arabic}{enumi}{enumii}{}{.}%
\item {} 
Choose geometry type (one of the tabs)

\item {} 
Enter profile dimensions in “Dimensions” section or just click and drag one of the handles in the drawing on the right side

\item {} 
Results in “Output” section can be saved by clicking “save” button

\item {} 
Save results in your project by entering modulus of elasticity and shear modulus.

\item {} 
Calculated and entered values can be seen in profiles table in the visualizer.

\end{enumerate}

\begin{figure}[htbp]
\centering
\capstart

\noindent\sphinxincludegraphics{{beamsize}.png}
\caption{BeamSize GUI depicting interaction with drawing.}\label{\detokenize{making_a_model:id28}}\end{figure}


\section{Theory}
\label{\detokenize{theory:theory}}\label{\detokenize{theory::doc}}
This project is base on bachelor thesis “Entwicklung eines Softwaresystems zur Visualisierung und Simulation von
Tragwerksstrukturen” in english “Development of a software system for the visualization and simulation of truss
structures” at institute of structural mechanics and lightweight design at RWTH\sphinxhyphen{}Aachen University.


\subsection{Element Orientation}
\label{\detokenize{theory:element-orientation}}
Supposedly we have an element defined between points A and B.
We define elements local z\sphinxhyphen{}axis by an auxiliary vector \(\vec{v}\). As shown in Figure below, \(\vec{v}\) has one
orthogonal projection on the normal plane E. The normal vector of E points in the direction
of local x\sphinxhyphen{}axis. This projection of \(\vec{v}\) onto plane E points in the direction of the local z\sphinxhyphen{}axis.

\begin{figure}[htbp]
\centering
\capstart

\noindent\sphinxincludegraphics{{DefiningBeamOrientation}.png}
\caption{The rotation of an element is based only on the local z\sphinxhyphen{}axis, which is defined by the vector \(\vec{v}\)}\label{\detokenize{theory:id1}}\end{figure}


\subsection{CALFEM}
\label{\detokenize{theory:calfem}}
Beamon’s simulation is based on open source library CALFEM. For further information read
\sphinxcode{\sphinxupquote{CALFEM for Matlab}} and
\sphinxcode{\sphinxupquote{CALFEM for Python}}


\section{API}
\label{\detokenize{api:api}}\label{\detokenize{api::doc}}

\subsection{Core}
\label{\detokenize{api:module-beamon.core}}\label{\detokenize{api:core}}\index{module@\spxentry{module}!beamon.core@\spxentry{beamon.core}}\index{beamon.core@\spxentry{beamon.core}!module@\spxentry{module}}
Core functions for beamon
\index{assemble\_bc() (in module beamon.core)@\spxentry{assemble\_bc()}\spxextra{in module beamon.core}}

\begin{fulllineitems}
\phantomsection\label{\detokenize{api:beamon.core.assemble_bc}}\pysiglinewithargsret{\sphinxcode{\sphinxupquote{beamon.core.}}\sphinxbfcode{\sphinxupquote{assemble\_bc}}}{\emph{\DUrole{n}{nodes\_bc}}, \emph{\DUrole{n}{Dof}}}{}
Assemble boundary condition information in a vector.
nodes\_bc should contain {[}u\_x,u\_y,u\_z,phi\_x,phi\_y,phi\_z{]} for each node, which can only contain
0 (free) or 1 (locked).
The assembled bc vector could be ex. {[}1,2,5,6{]} that locks movement in those numbered directions.
\begin{quote}\begin{description}
\item[{Parameters}] \leavevmode\begin{itemize}
\item {} 
\sphinxstyleliteralstrong{\sphinxupquote{nodes\_bc}} (\sphinxstyleliteralemphasis{\sphinxupquote{integer matrix}}) \textendash{} boundary condition information from database

\item {} 
\sphinxstyleliteralstrong{\sphinxupquote{Dof}} (\sphinxstyleliteralemphasis{\sphinxupquote{integer matrix}}) \textendash{} nodes degrees of freedom

\end{itemize}

\item[{Returns}] \leavevmode
bc vector

\item[{Return type}] \leavevmode
integer 1 x n vector

\end{description}\end{quote}

\end{fulllineitems}

\index{assemble\_element\_loads() (in module beamon.core)@\spxentry{assemble\_element\_loads()}\spxextra{in module beamon.core}}

\begin{fulllineitems}
\phantomsection\label{\detokenize{api:beamon.core.assemble_element_loads}}\pysiglinewithargsret{\sphinxcode{\sphinxupquote{beamon.core.}}\sphinxbfcode{\sphinxupquote{assemble\_element\_loads}}}{\emph{\DUrole{n}{loads}}, \emph{\DUrole{n}{Edof}}}{}
Assemble element local loads vector from element loads.
Loads should contain {[}link\_id, qx, qy, qz, qw{]}
\begin{quote}\begin{description}
\item[{Parameters}] \leavevmode\begin{itemize}
\item {} 
\sphinxstyleliteralstrong{\sphinxupquote{loads}} (\sphinxstyleliteralemphasis{\sphinxupquote{float matrix}}) \textendash{} element loads

\item {} 
\sphinxstyleliteralstrong{\sphinxupquote{Edof}} (\sphinxstyleliteralemphasis{\sphinxupquote{integer matrix}}) \textendash{} Element degree of freedom (Topology matrix)

\end{itemize}

\item[{Returns}] \leavevmode
local forces

\item[{Return type}] \leavevmode
float matrix

\end{description}\end{quote}

\end{fulllineitems}

\index{assemble\_global\_f() (in module beamon.core)@\spxentry{assemble\_global\_f()}\spxextra{in module beamon.core}}

\begin{fulllineitems}
\phantomsection\label{\detokenize{api:beamon.core.assemble_global_f}}\pysiglinewithargsret{\sphinxcode{\sphinxupquote{beamon.core.}}\sphinxbfcode{\sphinxupquote{assemble\_global\_f}}}{\emph{\DUrole{n}{loads}}, \emph{\DUrole{n}{Dof}}, \emph{\DUrole{n}{ndof}}}{}
Assemble global loads vector from nodes loads.
Loads should contain nodes coordinates and the forces applied to each node {[}x,y,z, u,v,w, m\_x,m\_y,m\_z{]}
\begin{quote}\begin{description}
\item[{Parameters}] \leavevmode\begin{itemize}
\item {} 
\sphinxstyleliteralstrong{\sphinxupquote{loads}} (\sphinxstyleliteralemphasis{\sphinxupquote{float matrix}}) \textendash{} Nodes loads array

\item {} 
\sphinxstyleliteralstrong{\sphinxupquote{Dof}} (\sphinxstyleliteralemphasis{\sphinxupquote{integer matrix}}) \textendash{} nodes degrees of freedom

\item {} 
\sphinxstyleliteralstrong{\sphinxupquote{ndof}} (\sphinxstyleliteralemphasis{\sphinxupquote{integer}}) \textendash{} number of dofs n (defines the dimension of f)

\end{itemize}

\item[{Returns}] \leavevmode
global force vector (n x 1 vector)

\item[{Return type}] \leavevmode
float vector

\end{description}\end{quote}

\end{fulllineitems}

\index{assemble\_nodes\_results() (in module beamon.core)@\spxentry{assemble\_nodes\_results()}\spxextra{in module beamon.core}}

\begin{fulllineitems}
\phantomsection\label{\detokenize{api:beamon.core.assemble_nodes_results}}\pysiglinewithargsret{\sphinxcode{\sphinxupquote{beamon.core.}}\sphinxbfcode{\sphinxupquote{assemble\_nodes\_results}}}{\emph{\DUrole{n}{o\_edof}}, \emph{\DUrole{n}{nodes}}, \emph{\DUrole{n}{a}}, \emph{\DUrole{n}{Q}}}{}
Assemble nodes displacements and support forces according without joints dof results
\begin{quote}\begin{description}
\item[{Parameters}] \leavevmode\begin{itemize}
\item {} 
\sphinxstyleliteralstrong{\sphinxupquote{o\_edof}} (\sphinxstyleliteralemphasis{\sphinxupquote{integer matrix}}) \textendash{} origin edof containing zeros for free dof

\item {} 
\sphinxstyleliteralstrong{\sphinxupquote{nodes}} (\sphinxstyleliteralemphasis{\sphinxupquote{float matrix}}) \textendash{} nodes coordinates

\item {} 
\sphinxstyleliteralstrong{\sphinxupquote{a}} (\sphinxstyleliteralemphasis{\sphinxupquote{numpy float vector}}) \textendash{} nodes displacements

\item {} 
\sphinxstyleliteralstrong{\sphinxupquote{Q}} (\sphinxstyleliteralemphasis{\sphinxupquote{numpy float vector}}) \textendash{} nodes support forces

\end{itemize}

\item[{Returns}] \leavevmode
table with columns {[}‘Node Nr.’, ‘ax’, ‘ay’, ‘az’, ‘aphi\_x’, ‘aphi\_y’, ‘aphi\_z’, ‘x’, ‘y’, ‘z’, ‘Qx’,

\end{description}\end{quote}

‘Qy’,’Qz’, ‘Qphi\_x’, ‘Qphi\_y’, ‘Qphi\_z’{]}
:rtype: pandas Dataframe

\end{fulllineitems}

\index{get\_local\_orientation() (in module beamon.core)@\spxentry{get\_local\_orientation()}\spxextra{in module beamon.core}}

\begin{fulllineitems}
\phantomsection\label{\detokenize{api:beamon.core.get_local_orientation}}\pysiglinewithargsret{\sphinxcode{\sphinxupquote{beamon.core.}}\sphinxbfcode{\sphinxupquote{get\_local\_orientation}}}{\emph{\DUrole{n}{orientation}}}{}~\begin{description}
\item[{Calculates all local axes orientation vectors according to v vector and}] \leavevmode
normal vectors n = (node2\sphinxhyphen{}node1)/abs(node2\sphinxhyphen{}node1)

\end{description}

Z axes is the projection of v on the plane with orthogonal vector X axes
\begin{quote}\begin{description}
\item[{Returns}] \leavevmode
x,y,z directions in each row

\end{description}\end{quote}

\end{fulllineitems}

\index{get\_rmat() (in module beamon.core)@\spxentry{get\_rmat()}\spxextra{in module beamon.core}}

\begin{fulllineitems}
\phantomsection\label{\detokenize{api:beamon.core.get_rmat}}\pysiglinewithargsret{\sphinxcode{\sphinxupquote{beamon.core.}}\sphinxbfcode{\sphinxupquote{get\_rmat}}}{\emph{\DUrole{n}{M}}, \emph{\DUrole{n}{N}}}{}
Gets the 3D Rotation matrix of a plane defined with vector normal M.
N will be the vector normal to the plane you want to rotate into.
\begin{quote}\begin{description}
\item[{Parameters}] \leavevmode\begin{itemize}
\item {} 
\sphinxstyleliteralstrong{\sphinxupquote{M}} (\sphinxstyleliteralemphasis{\sphinxupquote{double}}) \textendash{} 3 component list

\item {} 
\sphinxstyleliteralstrong{\sphinxupquote{N}} (\sphinxstyleliteralemphasis{\sphinxupquote{double}}) \textendash{} 3 component list

\end{itemize}

\item[{Returns}] \leavevmode
3x3 matrix

\item[{Return type}] \leavevmode
double

\end{description}\end{quote}

\end{fulllineitems}

\index{transform() (in module beamon.core)@\spxentry{transform()}\spxextra{in module beamon.core}}

\begin{fulllineitems}
\phantomsection\label{\detokenize{api:beamon.core.transform}}\pysiglinewithargsret{\sphinxcode{\sphinxupquote{beamon.core.}}\sphinxbfcode{\sphinxupquote{transform}}}{\emph{\DUrole{n}{T}}, \emph{\DUrole{n}{points}}}{}
Transforms each point in points matrix using the transformation matrix T.
points should have row wise point coordinates.
\begin{quote}\begin{description}
\item[{Parameters}] \leavevmode\begin{itemize}
\item {} 
\sphinxstyleliteralstrong{\sphinxupquote{T}} (\sphinxstyleliteralemphasis{\sphinxupquote{double}}\sphinxstyleliteralemphasis{\sphinxupquote{{[}}}\sphinxstyleliteralemphasis{\sphinxupquote{{]}}}\sphinxstyleliteralemphasis{\sphinxupquote{{[}}}\sphinxstyleliteralemphasis{\sphinxupquote{{]}}}) \textendash{} 3x3 Transformation Matrix

\item {} 
\sphinxstyleliteralstrong{\sphinxupquote{points}} (\sphinxstyleliteralemphasis{\sphinxupquote{double}}\sphinxstyleliteralemphasis{\sphinxupquote{{[}}}\sphinxstyleliteralemphasis{\sphinxupquote{{]}}}\sphinxstyleliteralemphasis{\sphinxupquote{{[}}}\sphinxstyleliteralemphasis{\sphinxupquote{{]}}}) \textendash{} nx3 points matrix

\end{itemize}

\item[{Returns}] \leavevmode
nx3 points matrix

\item[{Return type}] \leavevmode
double

\end{description}\end{quote}

\end{fulllineitems}



\subsection{Simulation}
\label{\detokenize{api:module-beamon.simulation}}\label{\detokenize{api:simulation}}\index{module@\spxentry{module}!beamon.simulation@\spxentry{beamon.simulation}}\index{beamon.simulation@\spxentry{beamon.simulation}!module@\spxentry{module}}\index{Simulation (class in beamon.simulation)@\spxentry{Simulation}\spxextra{class in beamon.simulation}}

\begin{fulllineitems}
\phantomsection\label{\detokenize{api:beamon.simulation.Simulation}}\pysiglinewithargsret{\sphinxbfcode{\sphinxupquote{class }}\sphinxcode{\sphinxupquote{beamon.simulation.}}\sphinxbfcode{\sphinxupquote{Simulation}}}{\emph{\DUrole{n}{database}\DUrole{p}{:} \DUrole{n}{{\hyperref[\detokenize{api:beamon.database.database.Database}]{\sphinxcrossref{beamon.database.database.Database}}}}}}{}
Main simulation module. Uses core and database to do calculations.
\index{element\_stats() (beamon.simulation.Simulation method)@\spxentry{element\_stats()}\spxextra{beamon.simulation.Simulation method}}

\begin{fulllineitems}
\phantomsection\label{\detokenize{api:beamon.simulation.Simulation.element_stats}}\pysiglinewithargsret{\sphinxbfcode{\sphinxupquote{element\_stats}}}{\emph{\DUrole{n}{m\_id}\DUrole{p}{:} \DUrole{n}{\sphinxhref{https://docs.python.org/3/library/functions.html\#int}{int}}}}{{ $\rightarrow$ pandas.core.frame.DataFrame}}
Statistics about Simulation results for elements as pandas Dataframe with following columns:
{[}‘Element Nr.’, ‘Min N’, ‘Max N’, ‘Min Vy’, ‘Max Vy’, ‘Min Vz’, ‘Max Vz’, ‘Min T’, ‘Max T’, ‘Min My’, ‘Max My’,
‘Min Mz’, ‘Max Mz’{]}
\begin{quote}\begin{description}
\item[{Returns}] \leavevmode
Simulation results statistics matrix

\item[{Return type}] \leavevmode
pandas Dataframe

\end{description}\end{quote}

\end{fulllineitems}

\index{global\_element\_displaced\_points() (beamon.simulation.Simulation method)@\spxentry{global\_element\_displaced\_points()}\spxextra{beamon.simulation.Simulation method}}

\begin{fulllineitems}
\phantomsection\label{\detokenize{api:beamon.simulation.Simulation.global_element_displaced_points}}\pysiglinewithargsret{\sphinxbfcode{\sphinxupquote{global\_element\_displaced\_points}}}{\emph{\DUrole{n}{m\_id}\DUrole{p}{:} \DUrole{n}{\sphinxhref{https://docs.python.org/3/library/functions.html\#int}{int}}}}{{ $\rightarrow$ pandas.core.frame.DataFrame}}
Returns displaced points coordinates in global coordinate system for each evaluation point on elements
as pandas Dataframe with following columns:
{[}‘Element Nr.’, ‘x’, ‘y’, ‘z’{]}
\begin{quote}\begin{description}
\item[{Returns}] \leavevmode
global element evaluation points matrix

\item[{Return type}] \leavevmode
pandas Dataframe

\end{description}\end{quote}

\end{fulllineitems}

\index{global\_element\_displacements() (beamon.simulation.Simulation method)@\spxentry{global\_element\_displacements()}\spxextra{beamon.simulation.Simulation method}}

\begin{fulllineitems}
\phantomsection\label{\detokenize{api:beamon.simulation.Simulation.global_element_displacements}}\pysiglinewithargsret{\sphinxbfcode{\sphinxupquote{global\_element\_displacements}}}{\emph{\DUrole{n}{m\_id}\DUrole{p}{:} \DUrole{n}{\sphinxhref{https://docs.python.org/3/library/functions.html\#int}{int}}}}{{ $\rightarrow$ pandas.core.frame.DataFrame}}
Simulation results for global element displacements for translation as pandas
Dataframe with following columns:
{[}‘Element Nr.’, ‘u’, ‘v’, ‘w’{]}
\begin{quote}\begin{description}
\item[{Returns}] \leavevmode
local element displacements matrix

\item[{Return type}] \leavevmode
pandas Dataframe

\end{description}\end{quote}

\end{fulllineitems}

\index{global\_element\_evaluation\_points() (beamon.simulation.Simulation method)@\spxentry{global\_element\_evaluation\_points()}\spxextra{beamon.simulation.Simulation method}}

\begin{fulllineitems}
\phantomsection\label{\detokenize{api:beamon.simulation.Simulation.global_element_evaluation_points}}\pysiglinewithargsret{\sphinxbfcode{\sphinxupquote{global\_element\_evaluation\_points}}}{\emph{\DUrole{n}{m\_id}\DUrole{p}{:} \DUrole{n}{\sphinxhref{https://docs.python.org/3/library/functions.html\#int}{int}}}}{{ $\rightarrow$ pandas.core.frame.DataFrame}}
Returns simulation evaluation point coordinates in global coordinate system
as pandas Dataframe with following columns:
{[}‘Element Nr.’, ‘x’, ‘y’, ‘z’{]}
\begin{quote}\begin{description}
\item[{Returns}] \leavevmode
global element evaluation points matrix

\item[{Return type}] \leavevmode
pandas Dataframe

\end{description}\end{quote}

\end{fulllineitems}

\index{local\_element\_displacements() (beamon.simulation.Simulation method)@\spxentry{local\_element\_displacements()}\spxextra{beamon.simulation.Simulation method}}

\begin{fulllineitems}
\phantomsection\label{\detokenize{api:beamon.simulation.Simulation.local_element_displacements}}\pysiglinewithargsret{\sphinxbfcode{\sphinxupquote{local\_element\_displacements}}}{\emph{\DUrole{n}{m\_id}\DUrole{p}{:} \DUrole{n}{\sphinxhref{https://docs.python.org/3/library/functions.html\#int}{int}}}}{{ $\rightarrow$ pandas.core.frame.DataFrame}}
Simulation results for local element displacements as pandas Dataframe with following columns:
{[}‘Element Nr.’, ‘xi’, ‘u’, ‘v’, ‘w’, ‘phi’{]}
\begin{quote}\begin{description}
\item[{Returns}] \leavevmode
local element displacements matrix

\item[{Return type}] \leavevmode
pandas Dataframe

\end{description}\end{quote}

\end{fulllineitems}

\index{local\_element\_section\_forces() (beamon.simulation.Simulation method)@\spxentry{local\_element\_section\_forces()}\spxextra{beamon.simulation.Simulation method}}

\begin{fulllineitems}
\phantomsection\label{\detokenize{api:beamon.simulation.Simulation.local_element_section_forces}}\pysiglinewithargsret{\sphinxbfcode{\sphinxupquote{local\_element\_section\_forces}}}{\emph{\DUrole{n}{m\_id}\DUrole{p}{:} \DUrole{n}{\sphinxhref{https://docs.python.org/3/library/functions.html\#int}{int}}}}{{ $\rightarrow$ pandas.core.frame.DataFrame}}
Simulation results for local element section forces as pandas Dataframe with following columns:
{[}‘Element Nr.’, ‘xi’, ‘N’, ‘Vy’, ‘Vz’, ‘T’, ‘My’,’Mz’{]}
\begin{quote}\begin{description}
\item[{Returns}] \leavevmode
local element section forces matrix

\item[{Return type}] \leavevmode
pandas Dataframe

\end{description}\end{quote}

\end{fulllineitems}

\index{nodes\_results() (beamon.simulation.Simulation method)@\spxentry{nodes\_results()}\spxextra{beamon.simulation.Simulation method}}

\begin{fulllineitems}
\phantomsection\label{\detokenize{api:beamon.simulation.Simulation.nodes_results}}\pysiglinewithargsret{\sphinxbfcode{\sphinxupquote{nodes\_results}}}{\emph{\DUrole{n}{m\_id}\DUrole{p}{:} \DUrole{n}{\sphinxhref{https://docs.python.org/3/library/functions.html\#int}{int}}}}{{ $\rightarrow$ pandas.core.frame.DataFrame}}
Simulation results for global nodes displacements (ax,ay,az,aphi\_x,aphi\_y,a\_phiz) with nodes coordinates
(x,y,z) and normal forces (Qx,Qy,Qz,Qphi\_x,Qphi\_y,Qphi\_z) as pandas Dataframe with following columns:
{[}‘Node Nr.’, ‘ax’, ‘ay’, ‘az’, ‘aphi\_x’, ‘aphi\_y’, ‘aphi\_z’, ‘x’, ‘y’, ‘z’, ‘Qx’, ‘Qy’, ‘Qz’, ‘Qphi\_x’,
‘Qphi\_y’, ‘Qphi\_z’{]}
\begin{quote}\begin{description}
\item[{Returns}] \leavevmode
local element displacements matrix

\item[{Return type}] \leavevmode
pandas Dataframe

\end{description}\end{quote}

\end{fulllineitems}

\index{simulate() (beamon.simulation.Simulation method)@\spxentry{simulate()}\spxextra{beamon.simulation.Simulation method}}

\begin{fulllineitems}
\phantomsection\label{\detokenize{api:beamon.simulation.Simulation.simulate}}\pysiglinewithargsret{\sphinxbfcode{\sphinxupquote{simulate}}}{\emph{\DUrole{n}{m\_id}\DUrole{p}{:} \DUrole{n}{\sphinxhref{https://docs.python.org/3/library/functions.html\#int}{int}}}, \emph{\DUrole{n}{n}\DUrole{p}{:} \DUrole{n}{\sphinxhref{https://docs.python.org/3/library/functions.html\#int}{int}} \DUrole{o}{=} \DUrole{default_value}{20}}}{}
Run the simulation according to data in the database.
\begin{quote}\begin{description}
\item[{Parameters}] \leavevmode\begin{itemize}
\item {} 
\sphinxstyleliteralstrong{\sphinxupquote{m\_id}} (\sphinxstyleliteralemphasis{\sphinxupquote{integer}}) \textendash{} model number

\item {} 
\sphinxstyleliteralstrong{\sphinxupquote{n}} (\sphinxstyleliteralemphasis{\sphinxupquote{integer}}) \textendash{} number of evaluation points

\end{itemize}

\end{description}\end{quote}

\end{fulllineitems}


\end{fulllineitems}



\subsection{Database}
\label{\detokenize{api:module-beamon.database.database}}\label{\detokenize{api:database}}\index{module@\spxentry{module}!beamon.database.database@\spxentry{beamon.database.database}}\index{beamon.database.database@\spxentry{beamon.database.database}!module@\spxentry{module}}\index{Database (class in beamon.database.database)@\spxentry{Database}\spxextra{class in beamon.database.database}}

\begin{fulllineitems}
\phantomsection\label{\detokenize{api:beamon.database.database.Database}}\pysiglinewithargsret{\sphinxbfcode{\sphinxupquote{class }}\sphinxcode{\sphinxupquote{beamon.database.database.}}\sphinxbfcode{\sphinxupquote{Database}}}{\emph{\DUrole{n}{path\_to\_database}\DUrole{o}{=}\DUrole{default_value}{None}}, \emph{\DUrole{n}{check}\DUrole{o}{=}\DUrole{default_value}{True}}, \emph{\DUrole{n}{ram}\DUrole{o}{=}\DUrole{default_value}{False}}, \emph{\DUrole{n}{path\_to\_save\_database}\DUrole{o}{=}\DUrole{default_value}{\textquotesingle{}\textquotesingle{}}}}{}
A data set to save all the data for constructing Elements in Visualization and for calculations in Beamon
\sphinxstylestrong{Warning}: This class should follow singleton pattern.

conn: SQLite3 connection object

cur: cursor for SQLite queries
\index{add\_lineload() (beamon.database.database.Database method)@\spxentry{add\_lineload()}\spxextra{beamon.database.database.Database method}}

\begin{fulllineitems}
\phantomsection\label{\detokenize{api:beamon.database.database.Database.add_lineload}}\pysiglinewithargsret{\sphinxbfcode{\sphinxupquote{add\_lineload}}}{\emph{\DUrole{n}{m\_id}}, \emph{\DUrole{n}{l\_index}}, \emph{\DUrole{n}{qx}}, \emph{\DUrole{n}{qy}}, \emph{\DUrole{n}{qz}}, \emph{\DUrole{n}{qw}}}{}
Add a lineload to a specified model.
\begin{quote}\begin{description}
\item[{Parameters}] \leavevmode\begin{itemize}
\item {} 
\sphinxstyleliteralstrong{\sphinxupquote{m\_id}} (\sphinxstyleliteralemphasis{\sphinxupquote{integer}}) \textendash{} model number

\item {} 
\sphinxstyleliteralstrong{\sphinxupquote{l\_index}} (\sphinxstyleliteralemphasis{\sphinxupquote{integer}}) \textendash{} link index number

\item {} 
\sphinxstyleliteralstrong{\sphinxupquote{qx}} (\sphinxhref{https://docs.python.org/3/library/functions.html\#float}{\sphinxstyleliteralemphasis{\sphinxupquote{float}}}) \textendash{} constant force in x direction

\item {} 
\sphinxstyleliteralstrong{\sphinxupquote{qy}} (\sphinxhref{https://docs.python.org/3/library/functions.html\#float}{\sphinxstyleliteralemphasis{\sphinxupquote{float}}}) \textendash{} constant force in y direction

\item {} 
\sphinxstyleliteralstrong{\sphinxupquote{qz}} (\sphinxhref{https://docs.python.org/3/library/functions.html\#float}{\sphinxstyleliteralemphasis{\sphinxupquote{float}}}) \textendash{} constant force in z direction

\item {} 
\sphinxstyleliteralstrong{\sphinxupquote{qw}} (\sphinxhref{https://docs.python.org/3/library/functions.html\#float}{\sphinxstyleliteralemphasis{\sphinxupquote{float}}}) \textendash{} constant momentum in x direction

\end{itemize}

\item[{Returns}] \leavevmode
False/lineload index

\end{description}\end{quote}

\end{fulllineitems}

\index{add\_link() (beamon.database.database.Database method)@\spxentry{add\_link()}\spxextra{beamon.database.database.Database method}}

\begin{fulllineitems}
\phantomsection\label{\detokenize{api:beamon.database.database.Database.add_link}}\pysiglinewithargsret{\sphinxbfcode{\sphinxupquote{add\_link}}}{\emph{\DUrole{n}{index1}\DUrole{p}{:} \DUrole{n}{\sphinxhref{https://docs.python.org/3/library/functions.html\#int}{int}}}, \emph{\DUrole{n}{index2}\DUrole{p}{:} \DUrole{n}{\sphinxhref{https://docs.python.org/3/library/functions.html\#int}{int}}}, \emph{\DUrole{n}{m\_id}\DUrole{p}{:} \DUrole{n}{\sphinxhref{https://docs.python.org/3/library/functions.html\#int}{int}}}, \emph{\DUrole{n}{dof}\DUrole{o}{=}\DUrole{default_value}{None}}, \emph{\DUrole{n}{profile\_id}\DUrole{o}{=}\DUrole{default_value}{None}}, \emph{\DUrole{n}{v\_x}\DUrole{o}{=}\DUrole{default_value}{None}}, \emph{\DUrole{n}{v\_y}\DUrole{o}{=}\DUrole{default_value}{None}}, \emph{\DUrole{n}{v\_z}\DUrole{o}{=}\DUrole{default_value}{None}}}{}
Adds a link from node with id index1 to node with id index2 and profile with profile\_id to a model with m\_id.
\begin{quote}\begin{description}
\item[{Parameters}] \leavevmode\begin{itemize}
\item {} 
\sphinxstyleliteralstrong{\sphinxupquote{index1}} \textendash{} starting node index

\item {} 
\sphinxstyleliteralstrong{\sphinxupquote{index2}} \textendash{} ending node index

\item {} 
\sphinxstyleliteralstrong{\sphinxupquote{m\_id}} (\sphinxstyleliteralemphasis{\sphinxupquote{integer}}) \textendash{} model id

\item {} 
\sphinxstyleliteralstrong{\sphinxupquote{dof}} \textendash{} degrees of freedom {[}ux1, uy1, uz1, phix1, … , uz2, phix2, phiy2, phiz2{]}

\item {} 
\sphinxstyleliteralstrong{\sphinxupquote{profile\_id}} \textendash{} optional index of the profile

\item {} 
\sphinxstyleliteralstrong{\sphinxupquote{v\_x}} \textendash{} x component of the direction determiner of the z local axes

\item {} 
\sphinxstyleliteralstrong{\sphinxupquote{v\_y}} \textendash{} y component of the direction determiner of the z local axes

\item {} 
\sphinxstyleliteralstrong{\sphinxupquote{v\_z}} \textendash{} z component of the direction determiner of the z local axes

\end{itemize}

\item[{Returns}] \leavevmode
True: if successfully added a link, False: if one of the given indices don’t exist

\end{description}\end{quote}

\end{fulllineitems}

\index{add\_load() (beamon.database.database.Database method)@\spxentry{add\_load()}\spxextra{beamon.database.database.Database method}}

\begin{fulllineitems}
\phantomsection\label{\detokenize{api:beamon.database.database.Database.add_load}}\pysiglinewithargsret{\sphinxbfcode{\sphinxupquote{add\_load}}}{\emph{\DUrole{n}{index}\DUrole{p}{:} \DUrole{n}{\sphinxhref{https://docs.python.org/3/library/functions.html\#int}{int}}}, \emph{\DUrole{n}{m\_id}\DUrole{p}{:} \DUrole{n}{\sphinxhref{https://docs.python.org/3/library/functions.html\#int}{int}}}, \emph{\DUrole{n}{x}}, \emph{\DUrole{n}{y}}, \emph{\DUrole{n}{z}}, \emph{\DUrole{n}{m\_x}}, \emph{\DUrole{n}{m\_y}}, \emph{\DUrole{n}{m\_z}}, \emph{\DUrole{n}{lineload\_id}\DUrole{o}{=}\DUrole{default_value}{None}}}{}
adds a load on the node specified with index and model number.
\begin{quote}\begin{description}
\item[{Parameters}] \leavevmode\begin{itemize}
\item {} 
\sphinxstyleliteralstrong{\sphinxupquote{index}} (\sphinxstyleliteralemphasis{\sphinxupquote{integer}}) \textendash{} node sub index number

\item {} 
\sphinxstyleliteralstrong{\sphinxupquote{m\_id}} (\sphinxstyleliteralemphasis{\sphinxupquote{integer}}) \textendash{} model number

\item {} 
\sphinxstyleliteralstrong{\sphinxupquote{x}} (\sphinxhref{https://docs.python.org/3/library/functions.html\#float}{\sphinxstyleliteralemphasis{\sphinxupquote{float}}}) \textendash{} first component of the force vector

\item {} 
\sphinxstyleliteralstrong{\sphinxupquote{y}} (\sphinxhref{https://docs.python.org/3/library/functions.html\#float}{\sphinxstyleliteralemphasis{\sphinxupquote{float}}}) \textendash{} second component of the force vector

\item {} 
\sphinxstyleliteralstrong{\sphinxupquote{z}} (\sphinxhref{https://docs.python.org/3/library/functions.html\#float}{\sphinxstyleliteralemphasis{\sphinxupquote{float}}}) \textendash{} third component of the force vector

\item {} 
\sphinxstyleliteralstrong{\sphinxupquote{m\_x}} (\sphinxhref{https://docs.python.org/3/library/functions.html\#float}{\sphinxstyleliteralemphasis{\sphinxupquote{float}}}) \textendash{} first component of the momentum vector

\item {} 
\sphinxstyleliteralstrong{\sphinxupquote{m\_y}} (\sphinxhref{https://docs.python.org/3/library/functions.html\#float}{\sphinxstyleliteralemphasis{\sphinxupquote{float}}}) \textendash{} second component of the momentum vector

\item {} 
\sphinxstyleliteralstrong{\sphinxupquote{m\_z}} (\sphinxhref{https://docs.python.org/3/library/functions.html\#float}{\sphinxstyleliteralemphasis{\sphinxupquote{float}}}) \textendash{} third component of the momentum vector

\item {} 
\sphinxstyleliteralstrong{\sphinxupquote{lineload\_id}} (\sphinxstyleliteralemphasis{\sphinxupquote{integer}}) \textendash{} optional number of reference lineload

\end{itemize}

\item[{Returns}] \leavevmode
True: if successfully added or updated / False: if node doesnt exist

\item[{Return type}] \leavevmode
boolean

\end{description}\end{quote}

\end{fulllineitems}

\index{add\_model() (beamon.database.database.Database method)@\spxentry{add\_model()}\spxextra{beamon.database.database.Database method}}

\begin{fulllineitems}
\phantomsection\label{\detokenize{api:beamon.database.database.Database.add_model}}\pysiglinewithargsret{\sphinxbfcode{\sphinxupquote{add\_model}}}{\emph{\DUrole{n}{m\_name}\DUrole{p}{:} \DUrole{n}{\sphinxhref{https://docs.python.org/3/library/stdtypes.html\#str}{str}}}}{{ $\rightarrow$ \sphinxhref{https://docs.python.org/3/library/functions.html\#int}{int}}}
Adds a new model to db and returns newly added model number
\begin{quote}\begin{description}
\item[{Parameters}] \leavevmode
\sphinxstyleliteralstrong{\sphinxupquote{m\_name}} (\sphinxstyleliteralemphasis{\sphinxupquote{string}}) \textendash{} new model name

\item[{Returns}] \leavevmode
model number

\item[{Return type}] \leavevmode
integer

\end{description}\end{quote}

\end{fulllineitems}

\index{add\_node() (beamon.database.database.Database method)@\spxentry{add\_node()}\spxextra{beamon.database.database.Database method}}

\begin{fulllineitems}
\phantomsection\label{\detokenize{api:beamon.database.database.Database.add_node}}\pysiglinewithargsret{\sphinxbfcode{\sphinxupquote{add\_node}}}{\emph{\DUrole{n}{m\_id}\DUrole{p}{:} \DUrole{n}{\sphinxhref{https://docs.python.org/3/library/functions.html\#int}{int}}}, \emph{\DUrole{n}{x}}, \emph{\DUrole{n}{y}}, \emph{\DUrole{n}{z}}, \emph{\DUrole{n}{u\_x}\DUrole{o}{=}\DUrole{default_value}{1}}, \emph{\DUrole{n}{u\_y}\DUrole{o}{=}\DUrole{default_value}{1}}, \emph{\DUrole{n}{u\_z}\DUrole{o}{=}\DUrole{default_value}{1}}, \emph{\DUrole{n}{phi\_x}\DUrole{o}{=}\DUrole{default_value}{1}}, \emph{\DUrole{n}{phi\_y}\DUrole{o}{=}\DUrole{default_value}{1}}, \emph{\DUrole{n}{phi\_z}\DUrole{o}{=}\DUrole{default_value}{1}}}{}
adds a node with coordinates x,y,z and the freedom of movement for all 6 degrees of freedom (dof) to a model
with given model number.

If dof numbers where not given default values will be used.
Default values for dof values {[}u\_x, .. ,phi\_z{]} is a vector with ones {[}1,…,1{]}.
\begin{quote}\begin{description}
\item[{Parameters}] \leavevmode\begin{itemize}
\item {} 
\sphinxstyleliteralstrong{\sphinxupquote{m\_id}} (\sphinxstyleliteralemphasis{\sphinxupquote{integer}}) \textendash{} Model id

\item {} 
\sphinxstyleliteralstrong{\sphinxupquote{x}} (\sphinxhref{https://docs.python.org/3/library/functions.html\#float}{\sphinxstyleliteralemphasis{\sphinxupquote{float}}}) \textendash{} x coordinate

\item {} 
\sphinxstyleliteralstrong{\sphinxupquote{y}} (\sphinxhref{https://docs.python.org/3/library/functions.html\#float}{\sphinxstyleliteralemphasis{\sphinxupquote{float}}}) \textendash{} y coordinate

\item {} 
\sphinxstyleliteralstrong{\sphinxupquote{z}} (\sphinxhref{https://docs.python.org/3/library/functions.html\#float}{\sphinxstyleliteralemphasis{\sphinxupquote{float}}}) \textendash{} z coordinate

\item {} 
\sphinxstyleliteralstrong{\sphinxupquote{u\_x}} (\sphinxstyleliteralemphasis{\sphinxupquote{0}}\sphinxstyleliteralemphasis{\sphinxupquote{ or }}\sphinxstyleliteralemphasis{\sphinxupquote{1}}) \textendash{} dof translation in x direction

\item {} 
\sphinxstyleliteralstrong{\sphinxupquote{u\_y}} (\sphinxstyleliteralemphasis{\sphinxupquote{0}}\sphinxstyleliteralemphasis{\sphinxupquote{ or }}\sphinxstyleliteralemphasis{\sphinxupquote{1}}) \textendash{} dof translation in y direction

\item {} 
\sphinxstyleliteralstrong{\sphinxupquote{u\_z}} (\sphinxstyleliteralemphasis{\sphinxupquote{0}}\sphinxstyleliteralemphasis{\sphinxupquote{ or }}\sphinxstyleliteralemphasis{\sphinxupquote{1}}) \textendash{} dof translation in z direction

\item {} 
\sphinxstyleliteralstrong{\sphinxupquote{phi\_x}} (\sphinxstyleliteralemphasis{\sphinxupquote{0}}\sphinxstyleliteralemphasis{\sphinxupquote{ or }}\sphinxstyleliteralemphasis{\sphinxupquote{1}}) \textendash{} dof translation in phi\_x direction

\item {} 
\sphinxstyleliteralstrong{\sphinxupquote{phi\_y}} (\sphinxstyleliteralemphasis{\sphinxupquote{0}}\sphinxstyleliteralemphasis{\sphinxupquote{ or }}\sphinxstyleliteralemphasis{\sphinxupquote{1}}) \textendash{} dof translation in phi\_y direction

\item {} 
\sphinxstyleliteralstrong{\sphinxupquote{phi\_z}} (\sphinxstyleliteralemphasis{\sphinxupquote{0}}\sphinxstyleliteralemphasis{\sphinxupquote{ or }}\sphinxstyleliteralemphasis{\sphinxupquote{1}}) \textendash{} dof translation in phi\_z direction

\end{itemize}

\item[{Returns}] \leavevmode
False: if node already exists, m\_id if new node was added

\end{description}\end{quote}

\end{fulllineitems}

\index{add\_profile() (beamon.database.database.Database method)@\spxentry{add\_profile()}\spxextra{beamon.database.database.Database method}}

\begin{fulllineitems}
\phantomsection\label{\detokenize{api:beamon.database.database.Database.add_profile}}\pysiglinewithargsret{\sphinxbfcode{\sphinxupquote{add\_profile}}}{\emph{\DUrole{n}{E}}, \emph{\DUrole{n}{G}}, \emph{\DUrole{n}{A}}, \emph{\DUrole{n}{Iy}}, \emph{\DUrole{n}{Iz}}, \emph{\DUrole{n}{kv}}}{}
Adds a profile

\end{fulllineitems}

\index{backup() (beamon.database.database.Database method)@\spxentry{backup()}\spxextra{beamon.database.database.Database method}}

\begin{fulllineitems}
\phantomsection\label{\detokenize{api:beamon.database.database.Database.backup}}\pysiglinewithargsret{\sphinxbfcode{\sphinxupquote{backup}}}{\emph{\DUrole{n}{destination\_path}\DUrole{p}{:} \DUrole{n}{\sphinxhref{https://docs.python.org/3/library/stdtypes.html\#str}{str}}}}{}
Creates Backup to Database with specified path
\begin{quote}\begin{description}
\item[{Parameters}] \leavevmode
\sphinxstyleliteralstrong{\sphinxupquote{destination\_path}} (\sphinxstyleliteralemphasis{\sphinxupquote{string}}) \textendash{} Backup database path

\end{description}\end{quote}

\end{fulllineitems}

\index{change\_dof() (beamon.database.database.Database method)@\spxentry{change\_dof()}\spxextra{beamon.database.database.Database method}}

\begin{fulllineitems}
\phantomsection\label{\detokenize{api:beamon.database.database.Database.change_dof}}\pysiglinewithargsret{\sphinxbfcode{\sphinxupquote{change\_dof}}}{\emph{\DUrole{n}{index}\DUrole{p}{:} \DUrole{n}{\sphinxhref{https://docs.python.org/3/library/functions.html\#int}{int}}}, \emph{\DUrole{n}{u\_x}\DUrole{o}{=}\DUrole{default_value}{None}}, \emph{\DUrole{n}{u\_y}\DUrole{o}{=}\DUrole{default_value}{None}}, \emph{\DUrole{n}{u\_z}\DUrole{o}{=}\DUrole{default_value}{None}}, \emph{\DUrole{n}{phi\_x}\DUrole{o}{=}\DUrole{default_value}{None}}, \emph{\DUrole{n}{phi\_y}\DUrole{o}{=}\DUrole{default_value}{None}}, \emph{\DUrole{n}{phi\_z}\DUrole{o}{=}\DUrole{default_value}{None}}}{}~\begin{description}
\item[{change the degree of freedom for a given node with specified node index. DOF specification:}] \leavevmode
1 is for free movement and 0 is for locked movement in a specified direction.

\end{description}

for example (index=300, u\_x=1, u\_y=1, u\_z=1, phi\_x=1, phi\_y=1, phi\_z=0) locks the rotation
in global z\sphinxhyphen{}axes direction of node 300.
\begin{quote}\begin{description}
\item[{Parameters}] \leavevmode\begin{itemize}
\item {} 
\sphinxstyleliteralstrong{\sphinxupquote{index}} \textendash{} index of the node

\item {} 
\sphinxstyleliteralstrong{\sphinxupquote{u\_x}} \textendash{} movement in x direction

\item {} 
\sphinxstyleliteralstrong{\sphinxupquote{u\_y}} \textendash{} movement in y direction

\item {} 
\sphinxstyleliteralstrong{\sphinxupquote{u\_z}} \textendash{} movement in z direction

\item {} 
\sphinxstyleliteralstrong{\sphinxupquote{phi\_x}} \textendash{} rotation in x axes

\item {} 
\sphinxstyleliteralstrong{\sphinxupquote{phi\_y}} \textendash{} rotation in y axes

\item {} 
\sphinxstyleliteralstrong{\sphinxupquote{phi\_z}} \textendash{} rotation in z axes

\end{itemize}

\item[{Returns}] \leavevmode
True: if successfully changed/ False: if node not found

\end{description}\end{quote}

\end{fulllineitems}

\index{change\_edof() (beamon.database.database.Database method)@\spxentry{change\_edof()}\spxextra{beamon.database.database.Database method}}

\begin{fulllineitems}
\phantomsection\label{\detokenize{api:beamon.database.database.Database.change_edof}}\pysiglinewithargsret{\sphinxbfcode{\sphinxupquote{change\_edof}}}{\emph{\DUrole{n}{edof}}, \emph{\DUrole{n}{n\_index}\DUrole{o}{=}\DUrole{default_value}{None}}, \emph{\DUrole{n}{e\_index}\DUrole{o}{=}\DUrole{default_value}{None}}}{}
Changes given degrees of freedom for all elements connected to the node with n\_index.
If e\_index is given instead, only the degrees of freedom for that element will be changed.
\begin{quote}\begin{description}
\item[{Parameters}] \leavevmode\begin{itemize}
\item {} 
\sphinxstyleliteralstrong{\sphinxupquote{n\_index}} \textendash{} index of the node | integer

\item {} 
\sphinxstyleliteralstrong{\sphinxupquote{edof}} \textendash{} list of integers {[}ux1, uy1, uz1, phix1, … , uz2, phix2, phiy2, phiz2{]}

\item {} 
\sphinxstyleliteralstrong{\sphinxupquote{e\_index}} \textendash{} index of the element | integer

\end{itemize}

\end{description}\end{quote}

\end{fulllineitems}

\index{change\_lineload() (beamon.database.database.Database method)@\spxentry{change\_lineload()}\spxextra{beamon.database.database.Database method}}

\begin{fulllineitems}
\phantomsection\label{\detokenize{api:beamon.database.database.Database.change_lineload}}\pysiglinewithargsret{\sphinxbfcode{\sphinxupquote{change\_lineload}}}{\emph{\DUrole{n}{index}}, \emph{\DUrole{n}{l\_index}}, \emph{\DUrole{n}{m\_id}\DUrole{p}{:} \DUrole{n}{\sphinxhref{https://docs.python.org/3/library/functions.html\#int}{int}}}, \emph{\DUrole{n}{qx}}, \emph{\DUrole{n}{qy}}, \emph{\DUrole{n}{qz}}, \emph{\DUrole{n}{qw}}}{}
Change lineload data with specified lineload index.
\begin{quote}\begin{description}
\item[{Parameters}] \leavevmode\begin{itemize}
\item {} 
\sphinxstyleliteralstrong{\sphinxupquote{index}} (\sphinxstyleliteralemphasis{\sphinxupquote{integer}}) \textendash{} lineload index number

\item {} 
\sphinxstyleliteralstrong{\sphinxupquote{l\_index}} (\sphinxstyleliteralemphasis{\sphinxupquote{integer}}) \textendash{} link index number

\item {} 
\sphinxstyleliteralstrong{\sphinxupquote{m\_id}} (\sphinxstyleliteralemphasis{\sphinxupquote{integer}}) \textendash{} model number

\item {} 
\sphinxstyleliteralstrong{\sphinxupquote{qx}} (\sphinxhref{https://docs.python.org/3/library/functions.html\#float}{\sphinxstyleliteralemphasis{\sphinxupquote{float}}}) \textendash{} constant force in x direction

\item {} 
\sphinxstyleliteralstrong{\sphinxupquote{qy}} (\sphinxhref{https://docs.python.org/3/library/functions.html\#float}{\sphinxstyleliteralemphasis{\sphinxupquote{float}}}) \textendash{} constant force in y direction

\item {} 
\sphinxstyleliteralstrong{\sphinxupquote{qz}} (\sphinxhref{https://docs.python.org/3/library/functions.html\#float}{\sphinxstyleliteralemphasis{\sphinxupquote{float}}}) \textendash{} constant force in z direction

\item {} 
\sphinxstyleliteralstrong{\sphinxupquote{qw}} (\sphinxhref{https://docs.python.org/3/library/functions.html\#float}{\sphinxstyleliteralemphasis{\sphinxupquote{float}}}) \textendash{} constant momentum in x direction

\end{itemize}

\item[{Returns}] \leavevmode
True

\end{description}\end{quote}

\end{fulllineitems}

\index{change\_link() (beamon.database.database.Database method)@\spxentry{change\_link()}\spxextra{beamon.database.database.Database method}}

\begin{fulllineitems}
\phantomsection\label{\detokenize{api:beamon.database.database.Database.change_link}}\pysiglinewithargsret{\sphinxbfcode{\sphinxupquote{change\_link}}}{\emph{\DUrole{n}{m\_id}\DUrole{p}{:} \DUrole{n}{\sphinxhref{https://docs.python.org/3/library/functions.html\#int}{int}}}, \emph{\DUrole{n}{index}\DUrole{p}{:} \DUrole{n}{\sphinxhref{https://docs.python.org/3/library/functions.html\#int}{int}}}, \emph{\DUrole{n}{n1\_id}\DUrole{p}{:} \DUrole{n}{\sphinxhref{https://docs.python.org/3/library/functions.html\#int}{int}}}, \emph{\DUrole{n}{n2\_id}\DUrole{p}{:} \DUrole{n}{\sphinxhref{https://docs.python.org/3/library/functions.html\#int}{int}}}, \emph{\DUrole{n}{v\_x}}, \emph{\DUrole{n}{v\_y}}, \emph{\DUrole{n}{v\_z}}, \emph{\DUrole{n}{profile\_id}\DUrole{p}{:} \DUrole{n}{\sphinxhref{https://docs.python.org/3/library/functions.html\#int}{int}}}}{}
Changes link values with specific index and model number
\begin{quote}\begin{description}
\item[{Returns}] \leavevmode
True/False

\end{description}\end{quote}

\end{fulllineitems}

\index{change\_load() (beamon.database.database.Database method)@\spxentry{change\_load()}\spxextra{beamon.database.database.Database method}}

\begin{fulllineitems}
\phantomsection\label{\detokenize{api:beamon.database.database.Database.change_load}}\pysiglinewithargsret{\sphinxbfcode{\sphinxupquote{change\_load}}}{\emph{\DUrole{n}{index}\DUrole{p}{:} \DUrole{n}{\sphinxhref{https://docs.python.org/3/library/functions.html\#int}{int}}}, \emph{\DUrole{n}{n\_index}\DUrole{p}{:} \DUrole{n}{\sphinxhref{https://docs.python.org/3/library/functions.html\#int}{int}}}, \emph{\DUrole{n}{m\_id}\DUrole{p}{:} \DUrole{n}{\sphinxhref{https://docs.python.org/3/library/functions.html\#int}{int}}}, \emph{\DUrole{n}{x}}, \emph{\DUrole{n}{y}}, \emph{\DUrole{n}{z}}, \emph{\DUrole{n}{m\_x}}, \emph{\DUrole{n}{m\_y}}, \emph{\DUrole{n}{m\_z}}}{}
Change load information with specific index number.
n\_index is nodes sub index number
\begin{quote}\begin{description}
\item[{Parameters}] \leavevmode\begin{itemize}
\item {} 
\sphinxstyleliteralstrong{\sphinxupquote{index}} (\sphinxstyleliteralemphasis{\sphinxupquote{integer}}) \textendash{} load index to be changed

\item {} 
\sphinxstyleliteralstrong{\sphinxupquote{n\_index}} (\sphinxstyleliteralemphasis{\sphinxupquote{integer}}) \textendash{} node sub index number

\item {} 
\sphinxstyleliteralstrong{\sphinxupquote{m\_id}} (\sphinxstyleliteralemphasis{\sphinxupquote{integer}}) \textendash{} model number

\item {} 
\sphinxstyleliteralstrong{\sphinxupquote{x}} (\sphinxhref{https://docs.python.org/3/library/functions.html\#float}{\sphinxstyleliteralemphasis{\sphinxupquote{float}}}) \textendash{} x\sphinxhyphen{}component

\item {} 
\sphinxstyleliteralstrong{\sphinxupquote{y}} (\sphinxhref{https://docs.python.org/3/library/functions.html\#float}{\sphinxstyleliteralemphasis{\sphinxupquote{float}}}) \textendash{} y\sphinxhyphen{}component

\item {} 
\sphinxstyleliteralstrong{\sphinxupquote{z}} (\sphinxhref{https://docs.python.org/3/library/functions.html\#float}{\sphinxstyleliteralemphasis{\sphinxupquote{float}}}) \textendash{} z\sphinxhyphen{}component

\item {} 
\sphinxstyleliteralstrong{\sphinxupquote{m\_x}} (\sphinxhref{https://docs.python.org/3/library/functions.html\#float}{\sphinxstyleliteralemphasis{\sphinxupquote{float}}}) \textendash{} momentum in x\sphinxhyphen{}direction

\item {} 
\sphinxstyleliteralstrong{\sphinxupquote{m\_y}} (\sphinxhref{https://docs.python.org/3/library/functions.html\#float}{\sphinxstyleliteralemphasis{\sphinxupquote{float}}}) \textendash{} momentum in y\sphinxhyphen{}direction

\item {} 
\sphinxstyleliteralstrong{\sphinxupquote{m\_z}} (\sphinxhref{https://docs.python.org/3/library/functions.html\#float}{\sphinxstyleliteralemphasis{\sphinxupquote{float}}}) \textendash{} momentum in z\sphinxhyphen{}direction

\end{itemize}

\item[{Returns}] \leavevmode
True

\end{description}\end{quote}

\end{fulllineitems}

\index{change\_node() (beamon.database.database.Database method)@\spxentry{change\_node()}\spxextra{beamon.database.database.Database method}}

\begin{fulllineitems}
\phantomsection\label{\detokenize{api:beamon.database.database.Database.change_node}}\pysiglinewithargsret{\sphinxbfcode{\sphinxupquote{change\_node}}}{\emph{\DUrole{n}{index}\DUrole{p}{:} \DUrole{n}{\sphinxhref{https://docs.python.org/3/library/functions.html\#int}{int}}}, \emph{\DUrole{n}{x}}, \emph{\DUrole{n}{y}}, \emph{\DUrole{n}{z}}, \emph{\DUrole{n}{u\_x}\DUrole{o}{=}\DUrole{default_value}{1}}, \emph{\DUrole{n}{u\_y}\DUrole{o}{=}\DUrole{default_value}{1}}, \emph{\DUrole{n}{u\_z}\DUrole{o}{=}\DUrole{default_value}{1}}, \emph{\DUrole{n}{phi\_x}\DUrole{o}{=}\DUrole{default_value}{1}}, \emph{\DUrole{n}{phi\_y}\DUrole{o}{=}\DUrole{default_value}{1}}, \emph{\DUrole{n}{phi\_z}\DUrole{o}{=}\DUrole{default_value}{1}}}{}
Changes node information with specified index number.
\begin{quote}\begin{description}
\item[{Parameters}] \leavevmode\begin{itemize}
\item {} 
\sphinxstyleliteralstrong{\sphinxupquote{index}} (\sphinxstyleliteralemphasis{\sphinxupquote{integer}}) \textendash{} node id

\item {} 
\sphinxstyleliteralstrong{\sphinxupquote{x}} (\sphinxhref{https://docs.python.org/3/library/functions.html\#float}{\sphinxstyleliteralemphasis{\sphinxupquote{float}}}) \textendash{} x coordinate

\item {} 
\sphinxstyleliteralstrong{\sphinxupquote{y}} (\sphinxhref{https://docs.python.org/3/library/functions.html\#float}{\sphinxstyleliteralemphasis{\sphinxupquote{float}}}) \textendash{} y coordinate

\item {} 
\sphinxstyleliteralstrong{\sphinxupquote{z}} (\sphinxhref{https://docs.python.org/3/library/functions.html\#float}{\sphinxstyleliteralemphasis{\sphinxupquote{float}}}) \textendash{} z coordinate

\item {} 
\sphinxstyleliteralstrong{\sphinxupquote{u\_x}} (\sphinxstyleliteralemphasis{\sphinxupquote{0}}\sphinxstyleliteralemphasis{\sphinxupquote{ or }}\sphinxstyleliteralemphasis{\sphinxupquote{1}}) \textendash{} dof translation in x direction

\item {} 
\sphinxstyleliteralstrong{\sphinxupquote{u\_y}} (\sphinxstyleliteralemphasis{\sphinxupquote{0}}\sphinxstyleliteralemphasis{\sphinxupquote{ or }}\sphinxstyleliteralemphasis{\sphinxupquote{1}}) \textendash{} dof translation in y direction

\item {} 
\sphinxstyleliteralstrong{\sphinxupquote{u\_z}} (\sphinxstyleliteralemphasis{\sphinxupquote{0}}\sphinxstyleliteralemphasis{\sphinxupquote{ or }}\sphinxstyleliteralemphasis{\sphinxupquote{1}}) \textendash{} dof translation in z direction

\item {} 
\sphinxstyleliteralstrong{\sphinxupquote{phi\_x}} (\sphinxstyleliteralemphasis{\sphinxupquote{0}}\sphinxstyleliteralemphasis{\sphinxupquote{ or }}\sphinxstyleliteralemphasis{\sphinxupquote{1}}) \textendash{} dof translation in phi\_x direction

\item {} 
\sphinxstyleliteralstrong{\sphinxupquote{phi\_y}} (\sphinxstyleliteralemphasis{\sphinxupquote{0}}\sphinxstyleliteralemphasis{\sphinxupquote{ or }}\sphinxstyleliteralemphasis{\sphinxupquote{1}}) \textendash{} dof translation in phi\_y direction

\item {} 
\sphinxstyleliteralstrong{\sphinxupquote{phi\_z}} (\sphinxstyleliteralemphasis{\sphinxupquote{0}}\sphinxstyleliteralemphasis{\sphinxupquote{ or }}\sphinxstyleliteralemphasis{\sphinxupquote{1}}) \textendash{} dof translation in phi\_z direction

\end{itemize}

\item[{Returns}] \leavevmode
True/False

\end{description}\end{quote}

\end{fulllineitems}

\index{change\_profile() (beamon.database.database.Database method)@\spxentry{change\_profile()}\spxextra{beamon.database.database.Database method}}

\begin{fulllineitems}
\phantomsection\label{\detokenize{api:beamon.database.database.Database.change_profile}}\pysiglinewithargsret{\sphinxbfcode{\sphinxupquote{change\_profile}}}{\emph{\DUrole{n}{index}\DUrole{p}{:} \DUrole{n}{\sphinxhref{https://docs.python.org/3/library/functions.html\#int}{int}}}, \emph{\DUrole{n}{E}}, \emph{\DUrole{n}{G}}, \emph{\DUrole{n}{A}}, \emph{\DUrole{n}{Iy}}, \emph{\DUrole{n}{Iz}}, \emph{\DUrole{n}{kv}}}{}
Changes certain profile values with index
\begin{quote}\begin{description}
\item[{Returns}] \leavevmode
True/False

\end{description}\end{quote}

\end{fulllineitems}

\index{check\_database() (beamon.database.database.Database method)@\spxentry{check\_database()}\spxextra{beamon.database.database.Database method}}

\begin{fulllineitems}
\phantomsection\label{\detokenize{api:beamon.database.database.Database.check_database}}\pysiglinewithargsret{\sphinxbfcode{\sphinxupquote{check\_database}}}{}{}
Checks if the database has already been created or fails integrity check. Also checks if the following tables
has dublicate entries: Nodes, Links
\begin{quote}\begin{description}
\item[{Returns}] \leavevmode
True/False

\end{description}\end{quote}

\end{fulllineitems}

\index{clear\_database() (beamon.database.database.Database method)@\spxentry{clear\_database()}\spxextra{beamon.database.database.Database method}}

\begin{fulllineitems}
\phantomsection\label{\detokenize{api:beamon.database.database.Database.clear_database}}\pysiglinewithargsret{\sphinxbfcode{\sphinxupquote{clear\_database}}}{\emph{\DUrole{n}{m\_id}}}{}
Delete/erase all the data from the database tables of a specified model

\end{fulllineitems}

\index{contains\_link() (beamon.database.database.Database method)@\spxentry{contains\_link()}\spxextra{beamon.database.database.Database method}}

\begin{fulllineitems}
\phantomsection\label{\detokenize{api:beamon.database.database.Database.contains_link}}\pysiglinewithargsret{\sphinxbfcode{\sphinxupquote{contains\_link}}}{\emph{\DUrole{n}{index1}\DUrole{p}{:} \DUrole{n}{\sphinxhref{https://docs.python.org/3/library/functions.html\#int}{int}}}, \emph{\DUrole{n}{index2}\DUrole{p}{:} \DUrole{n}{Optional\DUrole{p}{{[}}\sphinxhref{https://docs.python.org/3/library/functions.html\#int}{int}\DUrole{p}{{]}}} \DUrole{o}{=} \DUrole{default_value}{None}}, \emph{\DUrole{n}{m\_id}\DUrole{p}{:} \DUrole{n}{Optional\DUrole{p}{{[}}\sphinxhref{https://docs.python.org/3/library/functions.html\#int}{int}\DUrole{p}{{]}}} \DUrole{o}{=} \DUrole{default_value}{None}}}{}
If only index1 is given: see if a link with index1 exists.
if index1 and m\_id are given: see if a link with sub index exists in model with specified model number
If index1, index2 and model number are given: see if a link with the nodes sub index1 and sub index2 exists.
\begin{quote}\begin{description}
\item[{Parameters}] \leavevmode\begin{itemize}
\item {} 
\sphinxstyleliteralstrong{\sphinxupquote{index1}} \textendash{} Integer

\item {} 
\sphinxstyleliteralstrong{\sphinxupquote{index2}} \textendash{} Integer

\item {} 
\sphinxstyleliteralstrong{\sphinxupquote{m\_id}} (\sphinxstyleliteralemphasis{\sphinxupquote{integer}}) \textendash{} model number

\end{itemize}

\item[{Returns}] \leavevmode
True/False

\end{description}\end{quote}

\end{fulllineitems}

\index{contains\_node() (beamon.database.database.Database method)@\spxentry{contains\_node()}\spxextra{beamon.database.database.Database method}}

\begin{fulllineitems}
\phantomsection\label{\detokenize{api:beamon.database.database.Database.contains_node}}\pysiglinewithargsret{\sphinxbfcode{\sphinxupquote{contains\_node}}}{\emph{\DUrole{n}{index}\DUrole{p}{:} \DUrole{n}{\sphinxhref{https://docs.python.org/3/library/functions.html\#int}{int}}}, \emph{\DUrole{n}{m\_id}\DUrole{p}{:} \DUrole{n}{Optional\DUrole{p}{{[}}\sphinxhref{https://docs.python.org/3/library/functions.html\#int}{int}\DUrole{p}{{]}}} \DUrole{o}{=} \DUrole{default_value}{None}}}{{ $\rightarrow$ \sphinxhref{https://docs.python.org/3/library/functions.html\#bool}{bool}}}
Check if node with specific index number exists.
If model number is given node sub index number will be searched instead.
\begin{quote}\begin{description}
\item[{Parameters}] \leavevmode\begin{itemize}
\item {} 
\sphinxstyleliteralstrong{\sphinxupquote{index}} (\sphinxstyleliteralemphasis{\sphinxupquote{integer}}) \textendash{} node id inside a model (not pk)

\item {} 
\sphinxstyleliteralstrong{\sphinxupquote{m\_id}} (\sphinxstyleliteralemphasis{\sphinxupquote{integer}}) \textendash{} model number

\end{itemize}

\item[{Returns}] \leavevmode
True if node exist, False if nodes doesnt exist

\end{description}\end{quote}

\end{fulllineitems}

\index{contains\_profile() (beamon.database.database.Database method)@\spxentry{contains\_profile()}\spxextra{beamon.database.database.Database method}}

\begin{fulllineitems}
\phantomsection\label{\detokenize{api:beamon.database.database.Database.contains_profile}}\pysiglinewithargsret{\sphinxbfcode{\sphinxupquote{contains\_profile}}}{\emph{\DUrole{n}{index}\DUrole{p}{:} \DUrole{n}{\sphinxhref{https://docs.python.org/3/library/functions.html\#int}{int}}}}{}
Check if profile with a specified index exists.
\begin{quote}\begin{description}
\item[{Parameters}] \leavevmode
\sphinxstyleliteralstrong{\sphinxupquote{index}} (\sphinxstyleliteralemphasis{\sphinxupquote{integer}}) \textendash{} profile index number

\item[{Returns}] \leavevmode
True/False

\item[{Return type}] \leavevmode
boolean

\end{description}\end{quote}

\end{fulllineitems}

\index{export\_text() (beamon.database.database.Database method)@\spxentry{export\_text()}\spxextra{beamon.database.database.Database method}}

\begin{fulllineitems}
\phantomsection\label{\detokenize{api:beamon.database.database.Database.export_text}}\pysiglinewithargsret{\sphinxbfcode{\sphinxupquote{export\_text}}}{\emph{\DUrole{n}{path}}, \emph{\DUrole{n}{m\_id}\DUrole{p}{:} \DUrole{n}{\sphinxhref{https://docs.python.org/3/library/functions.html\#int}{int}}}}{}
Exports geometry of a specified model to text file
\begin{quote}\begin{description}
\item[{Parameters}] \leavevmode\begin{itemize}
\item {} 
\sphinxstyleliteralstrong{\sphinxupquote{path}} \textendash{} the path including the File name to be saved (ex. filename.csv)

\item {} 
\sphinxstyleliteralstrong{\sphinxupquote{m\_id}} (\sphinxstyleliteralemphasis{\sphinxupquote{integer}}) \textendash{} model number

\end{itemize}

\item[{Returns}] \leavevmode
True if saved, False if not saved

\end{description}\end{quote}

\end{fulllineitems}

\index{get\_all\_dof() (beamon.database.database.Database method)@\spxentry{get\_all\_dof()}\spxextra{beamon.database.database.Database method}}

\begin{fulllineitems}
\phantomsection\label{\detokenize{api:beamon.database.database.Database.get_all_dof}}\pysiglinewithargsret{\sphinxbfcode{\sphinxupquote{get\_all\_dof}}}{\emph{\DUrole{n}{m\_id}\DUrole{p}{:} \DUrole{n}{\sphinxhref{https://docs.python.org/3/library/functions.html\#int}{int}}}}{}
Gets all nodes information that has at least one dof number equals to 1 (locked) from a specified model.
\begin{quote}\begin{description}
\item[{Parameters}] \leavevmode
\sphinxstyleliteralstrong{\sphinxupquote{m\_id}} (\sphinxstyleliteralemphasis{\sphinxupquote{integer}}) \textendash{} model id

\item[{Returns}] \leavevmode
{[}x,y,z,u\_x,u\_y,u\_z,phi\_x,phi\_y,phi\_z{]}

\end{description}\end{quote}

\end{fulllineitems}

\index{get\_all\_lineloads() (beamon.database.database.Database method)@\spxentry{get\_all\_lineloads()}\spxextra{beamon.database.database.Database method}}

\begin{fulllineitems}
\phantomsection\label{\detokenize{api:beamon.database.database.Database.get_all_lineloads}}\pysiglinewithargsret{\sphinxbfcode{\sphinxupquote{get\_all\_lineloads}}}{\emph{\DUrole{n}{m\_id}}}{}
Gets all lineloads from Lineload table of a specified model.
\begin{quote}\begin{description}
\item[{Returns}] \leavevmode
{[}link\_id, qx, qy, qz, qw{]}

\item[{Return type}] \leavevmode
float matrix

\end{description}\end{quote}

\end{fulllineitems}

\index{get\_all\_lineloads\_for\_simulation() (beamon.database.database.Database method)@\spxentry{get\_all\_lineloads\_for\_simulation()}\spxextra{beamon.database.database.Database method}}

\begin{fulllineitems}
\phantomsection\label{\detokenize{api:beamon.database.database.Database.get_all_lineloads_for_simulation}}\pysiglinewithargsret{\sphinxbfcode{\sphinxupquote{get\_all\_lineloads\_for\_simulation}}}{\emph{\DUrole{n}{m\_id}\DUrole{p}{:} \DUrole{n}{\sphinxhref{https://docs.python.org/3/library/functions.html\#int}{int}}}}{{ $\rightarrow$ pandas.core.frame.DataFrame}}
Gets all lineloads from Lineload table of a specified model.
link\_id are links sub index in the specified model.
\begin{quote}\begin{description}
\item[{Parameters}] \leavevmode
\sphinxstyleliteralstrong{\sphinxupquote{m\_id}} (\sphinxstyleliteralemphasis{\sphinxupquote{integer}}) \textendash{} model number

\item[{Returns}] \leavevmode
following data columns: {[}link\_id, qx, qy, qz, qw{]}

\item[{Return type}] \leavevmode
DataFrame

\end{description}\end{quote}

\end{fulllineitems}

\index{get\_all\_links() (beamon.database.database.Database method)@\spxentry{get\_all\_links()}\spxextra{beamon.database.database.Database method}}

\begin{fulllineitems}
\phantomsection\label{\detokenize{api:beamon.database.database.Database.get_all_links}}\pysiglinewithargsret{\sphinxbfcode{\sphinxupquote{get\_all\_links}}}{\emph{\DUrole{n}{m\_id}\DUrole{p}{:} \DUrole{n}{\sphinxhref{https://docs.python.org/3/library/functions.html\#int}{int}}}}{}
Gets all links in the links table for a specified model.
Delivers the table with node’s sub indices.
\begin{quote}\begin{description}
\item[{Parameters}] \leavevmode
\sphinxstyleliteralstrong{\sphinxupquote{m\_id}} (\sphinxstyleliteralemphasis{\sphinxupquote{integer}}) \textendash{} model id

\item[{Returns}] \leavevmode
matrix

\end{description}\end{quote}

\end{fulllineitems}

\index{get\_all\_links\_profile\_indexes() (beamon.database.database.Database method)@\spxentry{get\_all\_links\_profile\_indexes()}\spxextra{beamon.database.database.Database method}}

\begin{fulllineitems}
\phantomsection\label{\detokenize{api:beamon.database.database.Database.get_all_links_profile_indexes}}\pysiglinewithargsret{\sphinxbfcode{\sphinxupquote{get\_all\_links\_profile\_indexes}}}{\emph{\DUrole{n}{m\_id}\DUrole{p}{:} \DUrole{n}{\sphinxhref{https://docs.python.org/3/library/functions.html\#int}{int}}}}{}
Gets all the profiles indexes of the existing links in a specified model.
\begin{quote}\begin{description}
\item[{Returns}] \leavevmode
None: if link index don’t exist. Otherwise index of the profile

\end{description}\end{quote}

\end{fulllineitems}

\index{get\_all\_nodes() (beamon.database.database.Database method)@\spxentry{get\_all\_nodes()}\spxextra{beamon.database.database.Database method}}

\begin{fulllineitems}
\phantomsection\label{\detokenize{api:beamon.database.database.Database.get_all_nodes}}\pysiglinewithargsret{\sphinxbfcode{\sphinxupquote{get\_all\_nodes}}}{\emph{\DUrole{n}{m\_id}\DUrole{p}{:} \DUrole{n}{\sphinxhref{https://docs.python.org/3/library/functions.html\#int}{int}}}}{{ $\rightarrow$ pandas.core.frame.DataFrame}}
Gets all nodes information in the DataSet from a specified model. Used for the GUI\sphinxhyphen{}Table
\begin{quote}\begin{description}
\item[{Parameters}] \leavevmode
\sphinxstyleliteralstrong{\sphinxupquote{m\_id}} (\sphinxstyleliteralemphasis{\sphinxupquote{integer}}) \textendash{} model id

\item[{Returns}] \leavevmode
table {[}x, y, z, u\_x, u\_y, u\_z, phi\_x, phi\_y, phi\_z{]}

\end{description}\end{quote}

\end{fulllineitems}

\index{get\_bc() (beamon.database.database.Database method)@\spxentry{get\_bc()}\spxextra{beamon.database.database.Database method}}

\begin{fulllineitems}
\phantomsection\label{\detokenize{api:beamon.database.database.Database.get_bc}}\pysiglinewithargsret{\sphinxbfcode{\sphinxupquote{get\_bc}}}{\emph{\DUrole{n}{m\_id}\DUrole{p}{:} \DUrole{n}{\sphinxhref{https://docs.python.org/3/library/functions.html\#int}{int}}}}{}
Get only bc information from nodes for a specified model.
\begin{quote}\begin{description}
\item[{Parameters}] \leavevmode
\sphinxstyleliteralstrong{\sphinxupquote{m\_id}} (\sphinxstyleliteralemphasis{\sphinxupquote{integer}}) \textendash{} model id

\item[{Returns}] \leavevmode
{[}u\_x,u\_y,u\_z,phi\_x,phi\_y,phi\_z{]}

\end{description}\end{quote}

\end{fulllineitems}

\index{get\_default\_edof() (beamon.database.database.Database method)@\spxentry{get\_default\_edof()}\spxextra{beamon.database.database.Database method}}

\begin{fulllineitems}
\phantomsection\label{\detokenize{api:beamon.database.database.Database.get_default_edof}}\pysiglinewithargsret{\sphinxbfcode{\sphinxupquote{get\_default\_edof}}}{\emph{\DUrole{n}{index}\DUrole{p}{:} \DUrole{n}{\sphinxhref{https://docs.python.org/3/library/functions.html\#int}{int}}}, \emph{\DUrole{n}{m\_id}\DUrole{p}{:} \DUrole{n}{\sphinxhref{https://docs.python.org/3/library/functions.html\#int}{int}}}}{}
Gets default element degrees of freedom numbers for a specific link in a specified model
\begin{quote}\begin{description}
\item[{Parameters}] \leavevmode\begin{itemize}
\item {} 
\sphinxstyleliteralstrong{\sphinxupquote{index}} (\sphinxstyleliteralemphasis{\sphinxupquote{integer}}) \textendash{} link index number

\item {} 
\sphinxstyleliteralstrong{\sphinxupquote{m\_id}} (\sphinxstyleliteralemphasis{\sphinxupquote{integer}}) \textendash{} model number

\end{itemize}

\item[{Returns}] \leavevmode
1 x 12 array

\item[{Return type}] \leavevmode
integer array

\end{description}\end{quote}

\end{fulllineitems}

\index{get\_default\_link\_orientation() (beamon.database.database.Database method)@\spxentry{get\_default\_link\_orientation()}\spxextra{beamon.database.database.Database method}}

\begin{fulllineitems}
\phantomsection\label{\detokenize{api:beamon.database.database.Database.get_default_link_orientation}}\pysiglinewithargsret{\sphinxbfcode{\sphinxupquote{get\_default\_link\_orientation}}}{\emph{\DUrole{n}{index1}\DUrole{p}{:} \DUrole{n}{\sphinxhref{https://docs.python.org/3/library/functions.html\#int}{int}}}, \emph{\DUrole{n}{index2}\DUrole{p}{:} \DUrole{n}{\sphinxhref{https://docs.python.org/3/library/functions.html\#int}{int}}}}{}
Gets the default orientation vector of the element from starting node with index1 and ending node with index2
\begin{quote}\begin{description}
\item[{Parameters}] \leavevmode\begin{itemize}
\item {} 
\sphinxstyleliteralstrong{\sphinxupquote{index1}} \textendash{} index of the starting node

\item {} 
\sphinxstyleliteralstrong{\sphinxupquote{index2}} \textendash{} index of the ending node

\end{itemize}

\item[{Returns}] \leavevmode
{[}v\_x, v\_y, v\_z{]}

\end{description}\end{quote}

\end{fulllineitems}

\index{get\_default\_static\_edof() (beamon.database.database.Database method)@\spxentry{get\_default\_static\_edof()}\spxextra{beamon.database.database.Database method}}

\begin{fulllineitems}
\phantomsection\label{\detokenize{api:beamon.database.database.Database.get_default_static_edof}}\pysiglinewithargsret{\sphinxbfcode{\sphinxupquote{get\_default\_static\_edof}}}{\emph{\DUrole{n}{id1}\DUrole{p}{:} \DUrole{n}{\sphinxhref{https://docs.python.org/3/library/functions.html\#int}{int}}}, \emph{\DUrole{n}{id2}\DUrole{p}{:} \DUrole{n}{\sphinxhref{https://docs.python.org/3/library/functions.html\#int}{int}}}, \emph{\DUrole{n}{m\_id}\DUrole{p}{:} \DUrole{n}{\sphinxhref{https://docs.python.org/3/library/functions.html\#int}{int}}}}{}
Gets the default element degrees of freedom for a specific element defined with starting and
ending node indices of a specified model. Default means that all degrees of freedom are locked
(stiff connections).
\begin{quote}\begin{description}
\item[{Parameters}] \leavevmode\begin{itemize}
\item {} 
\sphinxstyleliteralstrong{\sphinxupquote{id1}} \textendash{} index of starting node

\item {} 
\sphinxstyleliteralstrong{\sphinxupquote{id2}} \textendash{} index of ending node

\item {} 
\sphinxstyleliteralstrong{\sphinxupquote{m\_id}} (\sphinxstyleliteralemphasis{\sphinxupquote{integer}}) \textendash{} model number

\end{itemize}

\item[{Returns}] \leavevmode
integer list {[}ux1, uy1, uz1, phix1, … , uz2, phix2, phiy2, phiz2{]}

\end{description}\end{quote}

\end{fulllineitems}

\index{get\_dof() (beamon.database.database.Database method)@\spxentry{get\_dof()}\spxextra{beamon.database.database.Database method}}

\begin{fulllineitems}
\phantomsection\label{\detokenize{api:beamon.database.database.Database.get_dof}}\pysiglinewithargsret{\sphinxbfcode{\sphinxupquote{get\_dof}}}{\emph{\DUrole{n}{index}\DUrole{p}{:} \DUrole{n}{\sphinxhref{https://docs.python.org/3/library/functions.html\#int}{int}}}}{}
Get the dof values of a specified node.
\begin{quote}\begin{description}
\item[{Parameters}] \leavevmode
\sphinxstyleliteralstrong{\sphinxupquote{index}} \textendash{} of the node

\item[{Returns}] \leavevmode
{[}u\_x,u\_y,u\_z,phi\_x,phi\_y,phi\_z{]}

\end{description}\end{quote}

\end{fulllineitems}

\index{get\_edof() (beamon.database.database.Database method)@\spxentry{get\_edof()}\spxextra{beamon.database.database.Database method}}

\begin{fulllineitems}
\phantomsection\label{\detokenize{api:beamon.database.database.Database.get_edof}}\pysiglinewithargsret{\sphinxbfcode{\sphinxupquote{get\_edof}}}{\emph{\DUrole{n}{m\_id}}}{}
Gets all links indices and degrees of freedom from links of a specified model.
\begin{quote}\begin{description}
\item[{Returns}] \leavevmode
{[}id, ux1,uy1,uz1,phix1,phiy1,phiz1,ux2,uy2,uz2,phix2,phiy2,phiz2{]}

\item[{Return type}] \leavevmode
integer matrix

\end{description}\end{quote}

\end{fulllineitems}

\index{get\_edof\_for\_simulation() (beamon.database.database.Database method)@\spxentry{get\_edof\_for\_simulation()}\spxextra{beamon.database.database.Database method}}

\begin{fulllineitems}
\phantomsection\label{\detokenize{api:beamon.database.database.Database.get_edof_for_simulation}}\pysiglinewithargsret{\sphinxbfcode{\sphinxupquote{get\_edof\_for\_simulation}}}{\emph{\DUrole{n}{m\_id}}}{}
Gets all links starting and ending nodes indices with degrees of freedom from links of a specified model.
\begin{quote}\begin{description}
\item[{Returns}] \leavevmode
{[}n1\_id, n2\_id, ux1,uy1,uz1,phix1,phiy1,phiz1,ux2,uy2,uz2,phix2,phiy2,phiz2{]}

\item[{Return type}] \leavevmode
integer matrix

\end{description}\end{quote}

\end{fulllineitems}

\index{get\_grid\_settings() (beamon.database.database.Database method)@\spxentry{get\_grid\_settings()}\spxextra{beamon.database.database.Database method}}

\begin{fulllineitems}
\phantomsection\label{\detokenize{api:beamon.database.database.Database.get_grid_settings}}\pysiglinewithargsret{\sphinxbfcode{\sphinxupquote{get\_grid\_settings}}}{}{}
Gets latest grid settings from table VisualizerSettings.
\begin{quote}\begin{description}
\item[{Returns}] \leavevmode
list of params / None if table is empty

\end{description}\end{quote}

\end{fulllineitems}

\index{get\_joints() (beamon.database.database.Database method)@\spxentry{get\_joints()}\spxextra{beamon.database.database.Database method}}

\begin{fulllineitems}
\phantomsection\label{\detokenize{api:beamon.database.database.Database.get_joints}}\pysiglinewithargsret{\sphinxbfcode{\sphinxupquote{get\_joints}}}{\emph{\DUrole{n}{m\_id}\DUrole{p}{:} \DUrole{n}{\sphinxhref{https://docs.python.org/3/library/functions.html\#int}{int}}}}{}
Get joints locations for a specified model.
Those are nodes coordinates + v. v is the direction of links from joints.
\begin{quote}\begin{description}
\item[{Returns}] \leavevmode
{[}x,y,z{]}

\item[{Return type}] \leavevmode
float array

\end{description}\end{quote}

\end{fulllineitems}

\index{get\_lineloads\_for\_plotting() (beamon.database.database.Database method)@\spxentry{get\_lineloads\_for\_plotting()}\spxextra{beamon.database.database.Database method}}

\begin{fulllineitems}
\phantomsection\label{\detokenize{api:beamon.database.database.Database.get_lineloads_for_plotting}}\pysiglinewithargsret{\sphinxbfcode{\sphinxupquote{get\_lineloads\_for\_plotting}}}{\emph{\DUrole{n}{m\_id}}}{}~\begin{description}
\item[{Gets lineloads index with starting and ending node coordinates}] \leavevmode
and lineload forces from a specified model for plotting.

\end{description}
\begin{quote}\begin{description}
\item[{Returns}] \leavevmode
{[}link\_id, x1,y1,z1,x2,y2,z2, qx,qy,qz,qw{]}

\item[{Return type}] \leavevmode
float matrix

\end{description}\end{quote}

\end{fulllineitems}

\index{get\_link\_direction() (beamon.database.database.Database method)@\spxentry{get\_link\_direction()}\spxextra{beamon.database.database.Database method}}

\begin{fulllineitems}
\phantomsection\label{\detokenize{api:beamon.database.database.Database.get_link_direction}}\pysiglinewithargsret{\sphinxbfcode{\sphinxupquote{get\_link\_direction}}}{\emph{\DUrole{n}{index1}\DUrole{p}{:} \DUrole{n}{\sphinxhref{https://docs.python.org/3/library/functions.html\#int}{int}}}, \emph{\DUrole{n}{index2}\DUrole{p}{:} \DUrole{n}{\sphinxhref{https://docs.python.org/3/library/functions.html\#int}{int}}}}{}
Get the normalized direction of the Link (local x\sphinxhyphen{}axis). Link is specified by starting and
ending node indices.
\begin{quote}\begin{description}
\item[{Parameters}] \leavevmode\begin{itemize}
\item {} 
\sphinxstyleliteralstrong{\sphinxupquote{index1}} \textendash{} index of the starting node

\item {} 
\sphinxstyleliteralstrong{\sphinxupquote{index2}} \textendash{} index of the ending node

\end{itemize}

\item[{Returns}] \leavevmode
{[}n\_x, n\_y, n\_z{]}

\end{description}\end{quote}

\end{fulllineitems}

\index{get\_link\_length() (beamon.database.database.Database method)@\spxentry{get\_link\_length()}\spxextra{beamon.database.database.Database method}}

\begin{fulllineitems}
\phantomsection\label{\detokenize{api:beamon.database.database.Database.get_link_length}}\pysiglinewithargsret{\sphinxbfcode{\sphinxupquote{get\_link\_length}}}{\emph{\DUrole{n}{index}\DUrole{p}{:} \DUrole{n}{\sphinxhref{https://docs.python.org/3/library/functions.html\#int}{int}}}}{}
Gets the length of a specified link.
\begin{quote}\begin{description}
\item[{Parameters}] \leavevmode
\sphinxstyleliteralstrong{\sphinxupquote{index}} \textendash{} index of the link

\item[{Returns}] \leavevmode
float

\end{description}\end{quote}

\end{fulllineitems}

\index{get\_link\_nodes() (beamon.database.database.Database method)@\spxentry{get\_link\_nodes()}\spxextra{beamon.database.database.Database method}}

\begin{fulllineitems}
\phantomsection\label{\detokenize{api:beamon.database.database.Database.get_link_nodes}}\pysiglinewithargsret{\sphinxbfcode{\sphinxupquote{get\_link\_nodes}}}{\emph{\DUrole{n}{link\_id}\DUrole{p}{:} \DUrole{n}{\sphinxhref{https://docs.python.org/3/library/functions.html\#int}{int}}}}{}
Gets links starting and ending nodes indices.
\begin{quote}\begin{description}
\item[{Parameters}] \leavevmode
\sphinxstyleliteralstrong{\sphinxupquote{link\_id}} (\sphinxstyleliteralemphasis{\sphinxupquote{integer}}) \textendash{} link id

\item[{Returns}] \leavevmode
{[}n1\_id, n2\_id{]}

\end{description}\end{quote}

\end{fulllineitems}

\index{get\_link\_orientation() (beamon.database.database.Database method)@\spxentry{get\_link\_orientation()}\spxextra{beamon.database.database.Database method}}

\begin{fulllineitems}
\phantomsection\label{\detokenize{api:beamon.database.database.Database.get_link_orientation}}\pysiglinewithargsret{\sphinxbfcode{\sphinxupquote{get\_link\_orientation}}}{\emph{\DUrole{n}{index}\DUrole{p}{:} \DUrole{n}{\sphinxhref{https://docs.python.org/3/library/functions.html\#int}{int}}}}{}
gets all local z\sphinxhyphen{}axes orientation information \sphinxhyphen{} plane normal vector (x axes) and v vector components
(orientation definition vector for z axes).
\begin{quote}\begin{description}
\item[{Parameters}] \leavevmode
\sphinxstyleliteralstrong{\sphinxupquote{index}} (\sphinxstyleliteralemphasis{\sphinxupquote{integer}}) \textendash{} link index number

\item[{Returns}] \leavevmode
x,y,z vectors for local system of the link

\end{description}\end{quote}

\end{fulllineitems}

\index{get\_link\_orientations() (beamon.database.database.Database method)@\spxentry{get\_link\_orientations()}\spxextra{beamon.database.database.Database method}}

\begin{fulllineitems}
\phantomsection\label{\detokenize{api:beamon.database.database.Database.get_link_orientations}}\pysiglinewithargsret{\sphinxbfcode{\sphinxupquote{get\_link\_orientations}}}{\emph{\DUrole{n}{m\_id}\DUrole{p}{:} \DUrole{n}{\sphinxhref{https://docs.python.org/3/library/functions.html\#int}{int}}}}{}
gets all local z\sphinxhyphen{}axes orientation information \sphinxhyphen{} plane normal vector (x axes) and v vector components
(orientation definition vector for z axes) \sphinxhyphen{} from a specified model.
\begin{quote}\begin{description}
\item[{Returns}] \leavevmode
x,y,z vectors for local system of all links

\end{description}\end{quote}

\end{fulllineitems}

\index{get\_link\_with\_id() (beamon.database.database.Database method)@\spxentry{get\_link\_with\_id()}\spxextra{beamon.database.database.Database method}}

\begin{fulllineitems}
\phantomsection\label{\detokenize{api:beamon.database.database.Database.get_link_with_id}}\pysiglinewithargsret{\sphinxbfcode{\sphinxupquote{get\_link\_with\_id}}}{\emph{\DUrole{n}{link\_id}\DUrole{p}{:} \DUrole{n}{\sphinxhref{https://docs.python.org/3/library/functions.html\#int}{int}}}}{}
Gets link data with a certain id
\begin{quote}\begin{description}
\item[{Parameters}] \leavevmode
\sphinxstyleliteralstrong{\sphinxupquote{link\_id}} (\sphinxstyleliteralemphasis{\sphinxupquote{integer}}) \textendash{} link id

\item[{Returns}] \leavevmode
link data

\end{description}\end{quote}

\end{fulllineitems}

\index{get\_links() (beamon.database.database.Database method)@\spxentry{get\_links()}\spxextra{beamon.database.database.Database method}}

\begin{fulllineitems}
\phantomsection\label{\detokenize{api:beamon.database.database.Database.get_links}}\pysiglinewithargsret{\sphinxbfcode{\sphinxupquote{get\_links}}}{\emph{\DUrole{n}{m\_id}\DUrole{p}{:} \DUrole{n}{\sphinxhref{https://docs.python.org/3/library/functions.html\#int}{int}}}}{}
Gets Links information for visualization purposes as line segment information for a specified model.
See PyVista line segments definition.
\begin{quote}\begin{description}
\item[{Parameters}] \leavevmode
\sphinxstyleliteralstrong{\sphinxupquote{m\_id}} (\sphinxstyleliteralemphasis{\sphinxupquote{integer}}) \textendash{} model id

\item[{Returns}] \leavevmode
numpy array for line segments with the form {[}x, y, z{]}

\end{description}\end{quote}

\end{fulllineitems}

\index{get\_links\_ending\_points() (beamon.database.database.Database method)@\spxentry{get\_links\_ending\_points()}\spxextra{beamon.database.database.Database method}}

\begin{fulllineitems}
\phantomsection\label{\detokenize{api:beamon.database.database.Database.get_links_ending_points}}\pysiglinewithargsret{\sphinxbfcode{\sphinxupquote{get\_links\_ending\_points}}}{\emph{\DUrole{n}{m\_id}\DUrole{p}{:} \DUrole{n}{\sphinxhref{https://docs.python.org/3/library/functions.html\#int}{int}}}}{}
gets x,y,z coordinates of all links ending points of a specified model.
This could be used to plot local coordinate systems of the links.
\begin{quote}\begin{description}
\item[{Returns}] \leavevmode
{[}x,y,z{]}

\end{description}\end{quote}

\end{fulllineitems}

\index{get\_links\_middle\_points() (beamon.database.database.Database method)@\spxentry{get\_links\_middle\_points()}\spxextra{beamon.database.database.Database method}}

\begin{fulllineitems}
\phantomsection\label{\detokenize{api:beamon.database.database.Database.get_links_middle_points}}\pysiglinewithargsret{\sphinxbfcode{\sphinxupquote{get\_links\_middle\_points}}}{\emph{\DUrole{n}{m\_id}\DUrole{p}{:} \DUrole{n}{\sphinxhref{https://docs.python.org/3/library/functions.html\#int}{int}}}}{}
gets x,y,z coordinates of all links middle points of a specified model.
This could be used to plot local coordinate systems of the links
\begin{quote}\begin{description}
\item[{Returns}] \leavevmode
{[}x,y,z{]}

\end{description}\end{quote}

\end{fulllineitems}

\index{get\_links\_profiles() (beamon.database.database.Database method)@\spxentry{get\_links\_profiles()}\spxextra{beamon.database.database.Database method}}

\begin{fulllineitems}
\phantomsection\label{\detokenize{api:beamon.database.database.Database.get_links_profiles}}\pysiglinewithargsret{\sphinxbfcode{\sphinxupquote{get\_links\_profiles}}}{\emph{\DUrole{n}{m\_id}\DUrole{p}{:} \DUrole{n}{\sphinxhref{https://docs.python.org/3/library/functions.html\#int}{int}}}}{}
Gets profile data from links of a specified model provided their profile id is set.
\begin{quote}\begin{description}
\item[{Returns}] \leavevmode
{[}E, G, A, Iy, Iz, Kv{]}

\end{description}\end{quote}

\end{fulllineitems}

\index{get\_links\_starting\_points() (beamon.database.database.Database method)@\spxentry{get\_links\_starting\_points()}\spxextra{beamon.database.database.Database method}}

\begin{fulllineitems}
\phantomsection\label{\detokenize{api:beamon.database.database.Database.get_links_starting_points}}\pysiglinewithargsret{\sphinxbfcode{\sphinxupquote{get\_links\_starting\_points}}}{\emph{\DUrole{n}{m\_id}\DUrole{p}{:} \DUrole{n}{\sphinxhref{https://docs.python.org/3/library/functions.html\#int}{int}}}}{}
gets x,y,z coordinates of all links start points of a specified model.
This is being used to plot local coordinate systems of links.
\begin{quote}\begin{description}
\item[{Returns}] \leavevmode
{[}x,y,z{]}

\end{description}\end{quote}

\end{fulllineitems}

\index{get\_loads() (beamon.database.database.Database method)@\spxentry{get\_loads()}\spxextra{beamon.database.database.Database method}}

\begin{fulllineitems}
\phantomsection\label{\detokenize{api:beamon.database.database.Database.get_loads}}\pysiglinewithargsret{\sphinxbfcode{\sphinxupquote{get\_loads}}}{\emph{\DUrole{n}{m\_id}\DUrole{p}{:} \DUrole{n}{\sphinxhref{https://docs.python.org/3/library/functions.html\#int}{int}}}}{}
Gets all the nodes forces from Load table of a specified model.
node\_id are sub indices of nodes in the specified model.
\begin{quote}\begin{description}
\item[{Parameters}] \leavevmode
\sphinxstyleliteralstrong{\sphinxupquote{m\_id}} (\sphinxstyleliteralemphasis{\sphinxupquote{integer}}) \textendash{} model number

\item[{Returns}] \leavevmode
following data columns: {[}id, node\_id, x, y, z, m\_x, m\_y, m\_z{]}

\item[{Return type}] \leavevmode
DataFrame

\end{description}\end{quote}

\end{fulllineitems}

\index{get\_loads\_for\_plotting() (beamon.database.database.Database method)@\spxentry{get\_loads\_for\_plotting()}\spxextra{beamon.database.database.Database method}}

\begin{fulllineitems}
\phantomsection\label{\detokenize{api:beamon.database.database.Database.get_loads_for_plotting}}\pysiglinewithargsret{\sphinxbfcode{\sphinxupquote{get\_loads\_for\_plotting}}}{\emph{\DUrole{n}{m\_id}\DUrole{p}{:} \DUrole{n}{\sphinxhref{https://docs.python.org/3/library/functions.html\#int}{int}}}}{}
Make a 9 column matrix containing (x,y,z,u,v,w, m\_x,m\_y,m\_z) which are the positions and components of the
forces applied to nodes of a specified model.
This method is necessary for plotting the node loads accordingly.
\begin{quote}\begin{description}
\item[{Returns}] \leavevmode
numpy array with {[}x,y,z,u,v,w, m\_x,m\_y,m\_z{]}/ None: if empty loads table

\end{description}\end{quote}

\end{fulllineitems}

\index{get\_loads\_for\_simulation() (beamon.database.database.Database method)@\spxentry{get\_loads\_for\_simulation()}\spxextra{beamon.database.database.Database method}}

\begin{fulllineitems}
\phantomsection\label{\detokenize{api:beamon.database.database.Database.get_loads_for_simulation}}\pysiglinewithargsret{\sphinxbfcode{\sphinxupquote{get\_loads\_for\_simulation}}}{\emph{\DUrole{n}{m\_id}\DUrole{p}{:} \DUrole{n}{\sphinxhref{https://docs.python.org/3/library/functions.html\#int}{int}}}}{}
Gets all the nodes forces from Load table of a specified model.
node\_id are nodes sub indices in the specified model.
\begin{quote}\begin{description}
\item[{Parameters}] \leavevmode
\sphinxstyleliteralstrong{\sphinxupquote{m\_id}} (\sphinxstyleliteralemphasis{\sphinxupquote{integer}}) \textendash{} model number

\item[{Returns}] \leavevmode
following data columns: {[}node\_id, x, y, z, m\_x, m\_y, m\_z{]}

\item[{Return type}] \leavevmode
DataFrame

\end{description}\end{quote}

\end{fulllineitems}

\index{get\_lost\_dependencies() (beamon.database.database.Database method)@\spxentry{get\_lost\_dependencies()}\spxextra{beamon.database.database.Database method}}

\begin{fulllineitems}
\phantomsection\label{\detokenize{api:beamon.database.database.Database.get_lost_dependencies}}\pysiglinewithargsret{\sphinxbfcode{\sphinxupquote{get\_lost\_dependencies}}}{}{}
This method gets nodes indexes from Links and node loads that are not connected anymore.
Warning: Using this method means you could have lost data and generated a delete anomaly in the database
TODO: use this method in integrity check

\end{fulllineitems}

\index{get\_models() (beamon.database.database.Database method)@\spxentry{get\_models()}\spxextra{beamon.database.database.Database method}}

\begin{fulllineitems}
\phantomsection\label{\detokenize{api:beamon.database.database.Database.get_models}}\pysiglinewithargsret{\sphinxbfcode{\sphinxupquote{get\_models}}}{}{}
Get all models as pandas dataframe

\end{fulllineitems}

\index{get\_node() (beamon.database.database.Database method)@\spxentry{get\_node()}\spxextra{beamon.database.database.Database method}}

\begin{fulllineitems}
\phantomsection\label{\detokenize{api:beamon.database.database.Database.get_node}}\pysiglinewithargsret{\sphinxbfcode{\sphinxupquote{get\_node}}}{\emph{\DUrole{n}{index}\DUrole{p}{:} \DUrole{n}{\sphinxhref{https://docs.python.org/3/library/functions.html\#int}{int}}}}{}
gets the nodes coordinates (x,y,z)
\begin{quote}\begin{description}
\item[{Parameters}] \leavevmode
\sphinxstyleliteralstrong{\sphinxupquote{index}} \textendash{} node id

\item[{Returns}] \leavevmode
None if node not found / numpy array with {[}x,y,z{]}

\end{description}\end{quote}

\end{fulllineitems}

\index{get\_nodes() (beamon.database.database.Database method)@\spxentry{get\_nodes()}\spxextra{beamon.database.database.Database method}}

\begin{fulllineitems}
\phantomsection\label{\detokenize{api:beamon.database.database.Database.get_nodes}}\pysiglinewithargsret{\sphinxbfcode{\sphinxupquote{get\_nodes}}}{\emph{\DUrole{n}{m\_id}\DUrole{p}{:} \DUrole{n}{\sphinxhref{https://docs.python.org/3/library/functions.html\#int}{int}}}}{}
Gets all nodes coordinates from a specific model
\begin{quote}\begin{description}
\item[{Parameters}] \leavevmode
\sphinxstyleliteralstrong{\sphinxupquote{m\_id}} (\sphinxstyleliteralemphasis{\sphinxupquote{integer}}) \textendash{} model id

\item[{Returns}] \leavevmode
numpy array for each node with the form {[}x, y, z{]}

\end{description}\end{quote}

\end{fulllineitems}

\index{get\_nodes\_indexes\_with\_bc() (beamon.database.database.Database method)@\spxentry{get\_nodes\_indexes\_with\_bc()}\spxextra{beamon.database.database.Database method}}

\begin{fulllineitems}
\phantomsection\label{\detokenize{api:beamon.database.database.Database.get_nodes_indexes_with_bc}}\pysiglinewithargsret{\sphinxbfcode{\sphinxupquote{get\_nodes\_indexes\_with\_bc}}}{\emph{\DUrole{n}{m\_id}\DUrole{p}{:} \DUrole{n}{\sphinxhref{https://docs.python.org/3/library/functions.html\#int}{int}}}}{}~\begin{description}
\item[{Gets all nodes id numbers which have boundary conditions defined on them for a specified model.}] \leavevmode
Similar to get\_all\_dof but delivers nodes indices instead of all nodes information.

\end{description}
\begin{quote}\begin{description}
\item[{Parameters}] \leavevmode
\sphinxstyleliteralstrong{\sphinxupquote{m\_id}} (\sphinxstyleliteralemphasis{\sphinxupquote{integer}}) \textendash{} model id

\item[{Returns}] \leavevmode
nodes indices

\item[{Return type}] \leavevmode
integer vector

\end{description}\end{quote}

\end{fulllineitems}

\index{get\_profiles() (beamon.database.database.Database method)@\spxentry{get\_profiles()}\spxextra{beamon.database.database.Database method}}

\begin{fulllineitems}
\phantomsection\label{\detokenize{api:beamon.database.database.Database.get_profiles}}\pysiglinewithargsret{\sphinxbfcode{\sphinxupquote{get\_profiles}}}{}{}
Gets all the profiles which are element properties {[}E G A I\_y I\_z K\_z{]}.
\begin{quote}\begin{description}
\item[{Returns}] \leavevmode
list

\end{description}\end{quote}

\end{fulllineitems}

\index{get\_results\_view\_settings() (beamon.database.database.Database method)@\spxentry{get\_results\_view\_settings()}\spxextra{beamon.database.database.Database method}}

\begin{fulllineitems}
\phantomsection\label{\detokenize{api:beamon.database.database.Database.get_results_view_settings}}\pysiglinewithargsret{\sphinxbfcode{\sphinxupquote{get\_results\_view\_settings}}}{}{}
Get all results view settings from table VisualizerSettings.
\begin{quote}\begin{description}
\item[{Returns}] \leavevmode
results view map

\item[{Return type}] \leavevmode
\sphinxhref{https://docs.python.org/3/library/stdtypes.html\#dict}{dict}

\end{description}\end{quote}

\end{fulllineitems}

\index{get\_static\_edof() (beamon.database.database.Database method)@\spxentry{get\_static\_edof()}\spxextra{beamon.database.database.Database method}}

\begin{fulllineitems}
\phantomsection\label{\detokenize{api:beamon.database.database.Database.get_static_edof}}\pysiglinewithargsret{\sphinxbfcode{\sphinxupquote{get\_static\_edof}}}{\emph{\DUrole{n}{m\_id}}}{}
Gets edof matrix for all elements from a specified model.
\begin{quote}\begin{description}
\item[{Parameters}] \leavevmode
\sphinxstyleliteralstrong{\sphinxupquote{m\_id}} (\sphinxstyleliteralemphasis{\sphinxupquote{integer}}) \textendash{} model number

\item[{Returns}] \leavevmode
n x 12 integer matrix

\item[{Return type}] \leavevmode
integer{[}{]}{[}{]}

\end{description}\end{quote}

\end{fulllineitems}

\index{get\_view\_settings() (beamon.database.database.Database method)@\spxentry{get\_view\_settings()}\spxextra{beamon.database.database.Database method}}

\begin{fulllineitems}
\phantomsection\label{\detokenize{api:beamon.database.database.Database.get_view_settings}}\pysiglinewithargsret{\sphinxbfcode{\sphinxupquote{get\_view\_settings}}}{}{}
Get all view settings from table VisualizerSettings.
\begin{quote}\begin{description}
\item[{Returns}] \leavevmode
list of params / None if table is empty

\end{description}\end{quote}

\end{fulllineitems}

\index{has\_links() (beamon.database.database.Database method)@\spxentry{has\_links()}\spxextra{beamon.database.database.Database method}}

\begin{fulllineitems}
\phantomsection\label{\detokenize{api:beamon.database.database.Database.has_links}}\pysiglinewithargsret{\sphinxbfcode{\sphinxupquote{has\_links}}}{\emph{\DUrole{n}{m\_id}\DUrole{p}{:} \DUrole{n}{\sphinxhref{https://docs.python.org/3/library/functions.html\#int}{int}}}}{}
Checks if there is any links in a model.
\begin{quote}\begin{description}
\item[{Parameters}] \leavevmode
\sphinxstyleliteralstrong{\sphinxupquote{m\_id}} (\sphinxstyleliteralemphasis{\sphinxupquote{integer}}) \textendash{} model id

\item[{Returns}] \leavevmode
True/False

\end{description}\end{quote}

\end{fulllineitems}

\index{has\_nodes() (beamon.database.database.Database method)@\spxentry{has\_nodes()}\spxextra{beamon.database.database.Database method}}

\begin{fulllineitems}
\phantomsection\label{\detokenize{api:beamon.database.database.Database.has_nodes}}\pysiglinewithargsret{\sphinxbfcode{\sphinxupquote{has\_nodes}}}{\emph{\DUrole{n}{m\_id}\DUrole{p}{:} \DUrole{n}{\sphinxhref{https://docs.python.org/3/library/functions.html\#int}{int}}}}{}
Checks if there is any nodes in a model
\begin{quote}\begin{description}
\item[{Parameters}] \leavevmode
\sphinxstyleliteralstrong{\sphinxupquote{m\_id}} (\sphinxstyleliteralemphasis{\sphinxupquote{integer}}) \textendash{} model id

\item[{Returns}] \leavevmode
True/False

\end{description}\end{quote}

\end{fulllineitems}

\index{import\_text() (beamon.database.database.Database method)@\spxentry{import\_text()}\spxextra{beamon.database.database.Database method}}

\begin{fulllineitems}
\phantomsection\label{\detokenize{api:beamon.database.database.Database.import_text}}\pysiglinewithargsret{\sphinxbfcode{\sphinxupquote{import\_text}}}{\emph{\DUrole{n}{path}\DUrole{p}{:} \DUrole{n}{\sphinxhref{https://docs.python.org/3/library/stdtypes.html\#str}{str}}}, \emph{\DUrole{n}{m\_id}\DUrole{p}{:} \DUrole{n}{Optional\DUrole{p}{{[}}\sphinxhref{https://docs.python.org/3/library/functions.html\#int}{int}\DUrole{p}{{]}}} \DUrole{o}{=} \DUrole{default_value}{None}}, \emph{\DUrole{n}{m\_name}\DUrole{p}{:} \DUrole{n}{Optional\DUrole{p}{{[}}\sphinxhref{https://docs.python.org/3/library/stdtypes.html\#str}{str}\DUrole{p}{{]}}} \DUrole{o}{=} \DUrole{default_value}{None}}, \emph{\DUrole{n}{overwrite}\DUrole{p}{:} \DUrole{n}{\sphinxhref{https://docs.python.org/3/library/functions.html\#bool}{bool}} \DUrole{o}{=} \DUrole{default_value}{False}}}{}
Import geometry files to the database. If model number is given, geometry will be appended to the existing
model. If model name was given geometry will be added to a new model.
If model number is given and overwrite is True geometry in the specified model will be overwritten.
View specified geometry syntax in documentation.
Keywords: {\color{red}\bfseries{}*}node, {\color{red}\bfseries{}*}element, {\color{red}\bfseries{}*}profile, {\color{red}\bfseries{}*}load, {\color{red}\bfseries{}*}lineload, {\color{red}\bfseries{}*}joint
Tables will be checked for integrity after import.
\begin{quote}\begin{description}
\item[{Parameters}] \leavevmode\begin{itemize}
\item {} 
\sphinxstyleliteralstrong{\sphinxupquote{path}} \textendash{} of the text file to be imported

\item {} 
\sphinxstyleliteralstrong{\sphinxupquote{m\_id}} (\sphinxstyleliteralemphasis{\sphinxupquote{integer}}) \textendash{} model number

\item {} 
\sphinxstyleliteralstrong{\sphinxupquote{m\_name}} (\sphinxstyleliteralemphasis{\sphinxupquote{string}}) \textendash{} model name

\item {} 
\sphinxstyleliteralstrong{\sphinxupquote{overwrite}} (\sphinxstyleliteralemphasis{\sphinxupquote{boolean}}) \textendash{} overwrite geometry if valid model number was given

\end{itemize}

\end{description}\end{quote}

\end{fulllineitems}

\index{is\_settings\_preset() (beamon.database.database.Database method)@\spxentry{is\_settings\_preset()}\spxextra{beamon.database.database.Database method}}

\begin{fulllineitems}
\phantomsection\label{\detokenize{api:beamon.database.database.Database.is_settings_preset}}\pysiglinewithargsret{\sphinxbfcode{\sphinxupquote{is\_settings\_preset}}}{}{}
Checks if settings table has already been changed or not.
\begin{quote}\begin{description}
\item[{Returns}] \leavevmode
True / False

\end{description}\end{quote}

\end{fulllineitems}

\index{make\_dummy\_file() (beamon.database.database.Database method)@\spxentry{make\_dummy\_file()}\spxextra{beamon.database.database.Database method}}

\begin{fulllineitems}
\phantomsection\label{\detokenize{api:beamon.database.database.Database.make_dummy_file}}\pysiglinewithargsret{\sphinxbfcode{\sphinxupquote{make\_dummy\_file}}}{\emph{\DUrole{n}{ram}}, \emph{\DUrole{n}{name}}}{}
Create a new sqlite database according to given settings. This method contains CREATE queries for database
tables.
\begin{quote}\begin{description}
\item[{Parameters}] \leavevmode\begin{itemize}
\item {} 
\sphinxstyleliteralstrong{\sphinxupquote{name}} \textendash{} name of the database file. Default name is a string containing a timestamp as follows:
Database\_dd\sphinxhyphen{}mm\sphinxhyphen{}yyyy\_hh\sphinxhyphen{}mm\sphinxhyphen{}ss

\item {} 
\sphinxstyleliteralstrong{\sphinxupquote{ram}} \textendash{} default:False. if True the database wil be loaded into the ram and not saved to hard drive

\end{itemize}

\end{description}\end{quote}

\end{fulllineitems}

\index{make\_edof() (beamon.database.database.Database method)@\spxentry{make\_edof()}\spxextra{beamon.database.database.Database method}}

\begin{fulllineitems}
\phantomsection\label{\detokenize{api:beamon.database.database.Database.make_edof}}\pysiglinewithargsret{\sphinxbfcode{\sphinxupquote{make\_edof}}}{\emph{\DUrole{n}{m\_id}}}{}
Create element degrees of freedom numbers for a specified model
This method is used when static topology is initiated.

\end{fulllineitems}

\index{remove\_lineload() (beamon.database.database.Database method)@\spxentry{remove\_lineload()}\spxextra{beamon.database.database.Database method}}

\begin{fulllineitems}
\phantomsection\label{\detokenize{api:beamon.database.database.Database.remove_lineload}}\pysiglinewithargsret{\sphinxbfcode{\sphinxupquote{remove\_lineload}}}{\emph{\DUrole{n}{index}}}{}
Remove a specified lineload.
\begin{quote}\begin{description}
\item[{Parameters}] \leavevmode
\sphinxstyleliteralstrong{\sphinxupquote{index}} (\sphinxstyleliteralemphasis{\sphinxupquote{integer}}) \textendash{} lineload index number

\end{description}\end{quote}

\end{fulllineitems}

\index{remove\_link() (beamon.database.database.Database method)@\spxentry{remove\_link()}\spxextra{beamon.database.database.Database method}}

\begin{fulllineitems}
\phantomsection\label{\detokenize{api:beamon.database.database.Database.remove_link}}\pysiglinewithargsret{\sphinxbfcode{\sphinxupquote{remove\_link}}}{\emph{\DUrole{n}{index}\DUrole{p}{:} \DUrole{n}{\sphinxhref{https://docs.python.org/3/library/functions.html\#int}{int}}}}{}
Drops a link from the links table.
\begin{quote}\begin{description}
\item[{Parameters}] \leavevmode
\sphinxstyleliteralstrong{\sphinxupquote{index}} \textendash{} index of the link

\item[{Returns}] \leavevmode
True: if link has been successfully droppend, False: otherwise

\end{description}\end{quote}

\end{fulllineitems}

\index{remove\_load() (beamon.database.database.Database method)@\spxentry{remove\_load()}\spxextra{beamon.database.database.Database method}}

\begin{fulllineitems}
\phantomsection\label{\detokenize{api:beamon.database.database.Database.remove_load}}\pysiglinewithargsret{\sphinxbfcode{\sphinxupquote{remove\_load}}}{\emph{\DUrole{n}{index}\DUrole{p}{:} \DUrole{n}{\sphinxhref{https://docs.python.org/3/library/functions.html\#int}{int}}}}{}
Remove load with certain index number.
\begin{quote}\begin{description}
\item[{Parameters}] \leavevmode
\sphinxstyleliteralstrong{\sphinxupquote{index}} (\sphinxstyleliteralemphasis{\sphinxupquote{integer}}) \textendash{} load index.

\end{description}\end{quote}

\end{fulllineitems}

\index{remove\_loads\_with\_node() (beamon.database.database.Database method)@\spxentry{remove\_loads\_with\_node()}\spxextra{beamon.database.database.Database method}}

\begin{fulllineitems}
\phantomsection\label{\detokenize{api:beamon.database.database.Database.remove_loads_with_node}}\pysiglinewithargsret{\sphinxbfcode{\sphinxupquote{remove\_loads\_with\_node}}}{\emph{\DUrole{n}{index}\DUrole{p}{:} \DUrole{n}{\sphinxhref{https://docs.python.org/3/library/functions.html\#int}{int}}}}{}
remove all forces and moments in relation with the node.
\begin{quote}\begin{description}
\item[{Parameters}] \leavevmode
\sphinxstyleliteralstrong{\sphinxupquote{index}} \textendash{} of the node

\item[{Returns}] \leavevmode
True

\end{description}\end{quote}

\end{fulllineitems}

\index{remove\_node() (beamon.database.database.Database method)@\spxentry{remove\_node()}\spxextra{beamon.database.database.Database method}}

\begin{fulllineitems}
\phantomsection\label{\detokenize{api:beamon.database.database.Database.remove_node}}\pysiglinewithargsret{\sphinxbfcode{\sphinxupquote{remove\_node}}}{\emph{\DUrole{n}{index}\DUrole{p}{:} \DUrole{n}{\sphinxhref{https://docs.python.org/3/library/functions.html\#int}{int}}}}{}
Drops a node from the nodes table.
\sphinxstylestrong{Warning}: Dependent links and loads entries will be deleted.
\begin{quote}\begin{description}
\item[{Parameters}] \leavevmode
\sphinxstyleliteralstrong{\sphinxupquote{index}} \textendash{} index of the node

\item[{Returns}] \leavevmode
True: if node has been successfully dropped, False: otherwise

\end{description}\end{quote}

\end{fulllineitems}

\index{remove\_profile() (beamon.database.database.Database method)@\spxentry{remove\_profile()}\spxextra{beamon.database.database.Database method}}

\begin{fulllineitems}
\phantomsection\label{\detokenize{api:beamon.database.database.Database.remove_profile}}\pysiglinewithargsret{\sphinxbfcode{\sphinxupquote{remove\_profile}}}{\emph{\DUrole{n}{index}\DUrole{p}{:} \DUrole{n}{\sphinxhref{https://docs.python.org/3/library/functions.html\#int}{int}}}}{}
Removes the profile with index
\begin{quote}\begin{description}
\item[{Parameters}] \leavevmode
\sphinxstyleliteralstrong{\sphinxupquote{index}} \textendash{} of the profile

\item[{Returns}] \leavevmode
True: successful / False: not successful

\end{description}\end{quote}

\end{fulllineitems}

\index{set\_grid\_settings() (beamon.database.database.Database method)@\spxentry{set\_grid\_settings()}\spxextra{beamon.database.database.Database method}}

\begin{fulllineitems}
\phantomsection\label{\detokenize{api:beamon.database.database.Database.set_grid_settings}}\pysiglinewithargsret{\sphinxbfcode{\sphinxupquote{set\_grid\_settings}}}{\emph{\DUrole{n}{params}}}{}
Sets all grid settings to the table VisualizerSettings.
\begin{quote}\begin{description}
\item[{Parameters}] \leavevmode
\sphinxstyleliteralstrong{\sphinxupquote{params}} (\sphinxstyleliteralemphasis{\sphinxupquote{doubles list}}) \textendash{} (grid\_x1origin,grid\_x2origin, grid\_x3origin, grid\_n1, grid\_n2, grid\_n3, grid\_xtick, grid\_ytick)

\end{description}\end{quote}

\end{fulllineitems}

\index{set\_results\_view\_settings() (beamon.database.database.Database method)@\spxentry{set\_results\_view\_settings()}\spxextra{beamon.database.database.Database method}}

\begin{fulllineitems}
\phantomsection\label{\detokenize{api:beamon.database.database.Database.set_results_view_settings}}\pysiglinewithargsret{\sphinxbfcode{\sphinxupquote{set\_results\_view\_settings}}}{\emph{\DUrole{n}{mapping}}}{}
Sets all results view settings to the table VisualizerSettings
\begin{quote}\begin{description}
\item[{Parameters}] \leavevmode
\sphinxstyleliteralstrong{\sphinxupquote{mapping}} (\sphinxhref{https://docs.python.org/3/library/stdtypes.html\#dict}{\sphinxstyleliteralemphasis{\sphinxupquote{dict}}}) \textendash{} dictionary to map each parameter

\end{description}\end{quote}

\end{fulllineitems}

\index{set\_view\_settings() (beamon.database.database.Database method)@\spxentry{set\_view\_settings()}\spxextra{beamon.database.database.Database method}}

\begin{fulllineitems}
\phantomsection\label{\detokenize{api:beamon.database.database.Database.set_view_settings}}\pysiglinewithargsret{\sphinxbfcode{\sphinxupquote{set\_view\_settings}}}{\emph{\DUrole{n}{params}}}{}
Sets all view settings to the table VisualizerSettings.
\begin{quote}\begin{description}
\item[{Parameters}] \leavevmode
\sphinxstyleliteralstrong{\sphinxupquote{params}} (\sphinxstyleliteralemphasis{\sphinxupquote{doubles list}}) \textendash{} list with 1 numbers following order (view\_dim)

\end{description}\end{quote}

\end{fulllineitems}


\end{fulllineitems}



\subsection{UI.Main}
\label{\detokenize{api:module-beamon.ui.main}}\label{\detokenize{api:ui-main}}\index{module@\spxentry{module}!beamon.ui.main@\spxentry{beamon.ui.main}}\index{beamon.ui.main@\spxentry{beamon.ui.main}!module@\spxentry{module}}\index{Main (class in beamon.ui.main)@\spxentry{Main}\spxextra{class in beamon.ui.main}}

\begin{fulllineitems}
\phantomsection\label{\detokenize{api:beamon.ui.main.Main}}\pysiglinewithargsret{\sphinxbfcode{\sphinxupquote{class }}\sphinxcode{\sphinxupquote{beamon.ui.main.}}\sphinxbfcode{\sphinxupquote{Main}}}{\emph{\DUrole{n}{input}\DUrole{p}{:} \DUrole{n}{Optional\DUrole{p}{{[}}\sphinxhref{https://docs.python.org/3/library/stdtypes.html\#str}{str}\DUrole{p}{{]}}} \DUrole{o}{=} \DUrole{default_value}{None}}, \emph{\DUrole{n}{m\_name}\DUrole{p}{:} \DUrole{n}{Optional\DUrole{p}{{[}}\sphinxhref{https://docs.python.org/3/library/stdtypes.html\#str}{str}\DUrole{p}{{]}}} \DUrole{o}{=} \DUrole{default_value}{None}}, \emph{\DUrole{n}{testmode}\DUrole{p}{:} \DUrole{n}{\sphinxhref{https://docs.python.org/3/library/functions.html\#bool}{bool}} \DUrole{o}{=} \DUrole{default_value}{False}}, \emph{\DUrole{n}{ram}\DUrole{p}{:} \DUrole{n}{\sphinxhref{https://docs.python.org/3/library/functions.html\#bool}{bool}} \DUrole{o}{=} \DUrole{default_value}{False}}, \emph{\DUrole{n}{database\_path}\DUrole{p}{:} \DUrole{n}{Optional\DUrole{p}{{[}}\sphinxhref{https://docs.python.org/3/library/stdtypes.html\#str}{str}\DUrole{p}{{]}}} \DUrole{o}{=} \DUrole{default_value}{None}}}{}
Main User Interface class for backend implementation.
Note:this class calls the compiled main\_ui.py which contains the GUI of Beamon
\index{about() (beamon.ui.main.Main method)@\spxentry{about()}\spxextra{beamon.ui.main.Main method}}

\begin{fulllineitems}
\phantomsection\label{\detokenize{api:beamon.ui.main.Main.about}}\pysiglinewithargsret{\sphinxbfcode{\sphinxupquote{about}}}{}{}
Opens info page with user manual in the dock widget

\end{fulllineitems}

\index{append\_to\_log() (beamon.ui.main.Main method)@\spxentry{append\_to\_log()}\spxextra{beamon.ui.main.Main method}}

\begin{fulllineitems}
\phantomsection\label{\detokenize{api:beamon.ui.main.Main.append_to_log}}\pysiglinewithargsret{\sphinxbfcode{\sphinxupquote{append\_to\_log}}}{\emph{\DUrole{n}{msg}}}{}
Appends a given message to log and timestampes it with current datetime
\begin{quote}\begin{description}
\item[{Parameters}] \leavevmode
\sphinxstyleliteralstrong{\sphinxupquote{msg}} \textendash{} string

\end{description}\end{quote}

\end{fulllineitems}

\index{cascadeView() (beamon.ui.main.Main method)@\spxentry{cascadeView()}\spxextra{beamon.ui.main.Main method}}

\begin{fulllineitems}
\phantomsection\label{\detokenize{api:beamon.ui.main.Main.cascadeView}}\pysiglinewithargsret{\sphinxbfcode{\sphinxupquote{cascadeView}}}{}{}
Changes Subwindows order to CascadeView

\end{fulllineitems}

\index{closeEvent() (beamon.ui.main.Main method)@\spxentry{closeEvent()}\spxextra{beamon.ui.main.Main method}}

\begin{fulllineitems}
\phantomsection\label{\detokenize{api:beamon.ui.main.Main.closeEvent}}\pysiglinewithargsret{\sphinxbfcode{\sphinxupquote{closeEvent}}}{\emph{\DUrole{n}{event}}}{}
controls closing signal
:param event: the closing event

\end{fulllineitems}

\index{display\_message() (beamon.ui.main.Main method)@\spxentry{display\_message()}\spxextra{beamon.ui.main.Main method}}

\begin{fulllineitems}
\phantomsection\label{\detokenize{api:beamon.ui.main.Main.display_message}}\pysiglinewithargsret{\sphinxbfcode{\sphinxupquote{display\_message}}}{\emph{\DUrole{n}{msg}}, \emph{\DUrole{n}{time}\DUrole{o}{=}\DUrole{default_value}{5000}}}{}
Display a message in status bar for a duration (default 5 seconds)
\begin{quote}\begin{description}
\item[{Parameters}] \leavevmode\begin{itemize}
\item {} 
\sphinxstyleliteralstrong{\sphinxupquote{msg}} (\sphinxstyleliteralemphasis{\sphinxupquote{string}}) \textendash{} Message to display

\item {} 
\sphinxstyleliteralstrong{\sphinxupquote{time}} (\sphinxstyleliteralemphasis{\sphinxupquote{integer}}) \textendash{} milliseconds

\end{itemize}

\end{description}\end{quote}

\end{fulllineitems}

\index{export\_csv() (beamon.ui.main.Main method)@\spxentry{export\_csv()}\spxextra{beamon.ui.main.Main method}}

\begin{fulllineitems}
\phantomsection\label{\detokenize{api:beamon.ui.main.Main.export_csv}}\pysiglinewithargsret{\sphinxbfcode{\sphinxupquote{export\_csv}}}{\emph{\DUrole{n}{dataframe}}}{}
Open user interface to export specific Dataset to a CSV file
\begin{quote}\begin{description}
\item[{Parameters}] \leavevmode
\sphinxstyleliteralstrong{\sphinxupquote{dataframe}} (\sphinxstyleliteralemphasis{\sphinxupquote{DataFrame}}) \textendash{} Dataset as Pandas DataFrame to be exported

\end{description}\end{quote}

\end{fulllineitems}

\index{export\_excel() (beamon.ui.main.Main method)@\spxentry{export\_excel()}\spxextra{beamon.ui.main.Main method}}

\begin{fulllineitems}
\phantomsection\label{\detokenize{api:beamon.ui.main.Main.export_excel}}\pysiglinewithargsret{\sphinxbfcode{\sphinxupquote{export\_excel}}}{\emph{\DUrole{n}{dataframe}}}{}
Open user interface to export specific Dataset to n Excel file
\begin{quote}\begin{description}
\item[{Parameters}] \leavevmode
\sphinxstyleliteralstrong{\sphinxupquote{dataframe}} (\sphinxstyleliteralemphasis{\sphinxupquote{DataFrame}}) \textendash{} Dataset as Pandas DataFrame to be exported

\end{description}\end{quote}

\end{fulllineitems}

\index{keyPressEvent() (beamon.ui.main.Main method)@\spxentry{keyPressEvent()}\spxextra{beamon.ui.main.Main method}}

\begin{fulllineitems}
\phantomsection\label{\detokenize{api:beamon.ui.main.Main.keyPressEvent}}\pysiglinewithargsret{\sphinxbfcode{\sphinxupquote{keyPressEvent}}}{\emph{\DUrole{n}{event}}}{}
Gets triggered when a key is pressed

\end{fulllineitems}

\index{onCustomContextMenuRequested() (beamon.ui.main.Main method)@\spxentry{onCustomContextMenuRequested()}\spxextra{beamon.ui.main.Main method}}

\begin{fulllineitems}
\phantomsection\label{\detokenize{api:beamon.ui.main.Main.onCustomContextMenuRequested}}\pysiglinewithargsret{\sphinxbfcode{\sphinxupquote{onCustomContextMenuRequested}}}{\emph{\DUrole{n}{pos}}}{}
Context Menu callback method
\begin{quote}\begin{description}
\item[{Parameters}] \leavevmode
\sphinxstyleliteralstrong{\sphinxupquote{pos}} (\sphinxstyleliteralemphasis{\sphinxupquote{QPosition}}) \textendash{} position of the context menu event that the widget receives

\end{description}\end{quote}

\end{fulllineitems}

\index{open\_database() (beamon.ui.main.Main method)@\spxentry{open\_database()}\spxextra{beamon.ui.main.Main method}}

\begin{fulllineitems}
\phantomsection\label{\detokenize{api:beamon.ui.main.Main.open_database}}\pysiglinewithargsret{\sphinxbfcode{\sphinxupquote{open\_database}}}{\emph{\DUrole{n}{path}\DUrole{p}{:} \DUrole{n}{\sphinxhref{https://docs.python.org/3/library/stdtypes.html\#str}{str}}}, \emph{\DUrole{n}{overwrite}\DUrole{p}{:} \DUrole{n}{\sphinxhref{https://docs.python.org/3/library/functions.html\#bool}{bool}} \DUrole{o}{=} \DUrole{default_value}{False}}}{}
Opens a database in path. Opens a new empty Database if overwrite is true. if File do not exist and overwrite
is false an exception will be thrown.
\begin{quote}\begin{description}
\item[{Parameters}] \leavevmode\begin{itemize}
\item {} 
\sphinxstyleliteralstrong{\sphinxupquote{path}} (\sphinxstyleliteralemphasis{\sphinxupquote{string}}) \textendash{} path to database file

\item {} 
\sphinxstyleliteralstrong{\sphinxupquote{overwrite}} (\sphinxstyleliteralemphasis{\sphinxupquote{boolean}}) \textendash{} True/False

\end{itemize}

\end{description}\end{quote}

\end{fulllineitems}

\index{open\_database\_diag() (beamon.ui.main.Main method)@\spxentry{open\_database\_diag()}\spxextra{beamon.ui.main.Main method}}

\begin{fulllineitems}
\phantomsection\label{\detokenize{api:beamon.ui.main.Main.open_database_diag}}\pysiglinewithargsret{\sphinxbfcode{\sphinxupquote{open\_database\_diag}}}{}{}
Opens a user interface to load a database file

\end{fulllineitems}

\index{open\_log() (beamon.ui.main.Main method)@\spxentry{open\_log()}\spxextra{beamon.ui.main.Main method}}

\begin{fulllineitems}
\phantomsection\label{\detokenize{api:beamon.ui.main.Main.open_log}}\pysiglinewithargsret{\sphinxbfcode{\sphinxupquote{open\_log}}}{}{}
Opens a Dialog with log history inside

\end{fulllineitems}

\index{save\_database\_as\_diag() (beamon.ui.main.Main method)@\spxentry{save\_database\_as\_diag()}\spxextra{beamon.ui.main.Main method}}

\begin{fulllineitems}
\phantomsection\label{\detokenize{api:beamon.ui.main.Main.save_database_as_diag}}\pysiglinewithargsret{\sphinxbfcode{\sphinxupquote{save\_database\_as\_diag}}}{}{}
Saves the database to a database file

\end{fulllineitems}

\index{startBeamSize() (beamon.ui.main.Main method)@\spxentry{startBeamSize()}\spxextra{beamon.ui.main.Main method}}

\begin{fulllineitems}
\phantomsection\label{\detokenize{api:beamon.ui.main.Main.startBeamSize}}\pysiglinewithargsret{\sphinxbfcode{\sphinxupquote{startBeamSize}}}{}{}
starts the application BeamSize as a subwindow

\end{fulllineitems}

\index{startQswHelp() (beamon.ui.main.Main method)@\spxentry{startQswHelp()}\spxextra{beamon.ui.main.Main method}}

\begin{fulllineitems}
\phantomsection\label{\detokenize{api:beamon.ui.main.Main.startQswHelp}}\pysiglinewithargsret{\sphinxbfcode{\sphinxupquote{startQswHelp}}}{}{}
starts a web browser from QtWebEngineView and displays the readme page

\end{fulllineitems}

\index{startVisualizer() (beamon.ui.main.Main method)@\spxentry{startVisualizer()}\spxextra{beamon.ui.main.Main method}}

\begin{fulllineitems}
\phantomsection\label{\detokenize{api:beamon.ui.main.Main.startVisualizer}}\pysiglinewithargsret{\sphinxbfcode{\sphinxupquote{startVisualizer}}}{\emph{\DUrole{n}{m\_id}\DUrole{p}{:} \DUrole{n}{Optional\DUrole{p}{{[}}\sphinxhref{https://docs.python.org/3/library/functions.html\#int}{int}\DUrole{p}{{]}}} \DUrole{o}{=} \DUrole{default_value}{None}}}{}
starts the application for visualising models.
\begin{quote}\begin{description}
\item[{Parameters}] \leavevmode
\sphinxstyleliteralstrong{\sphinxupquote{m\_id}} (\sphinxstyleliteralemphasis{\sphinxupquote{integer}}) \textendash{} model number

\end{description}\end{quote}

\end{fulllineitems}

\index{start\_visualizer\_manual() (beamon.ui.main.Main method)@\spxentry{start\_visualizer\_manual()}\spxextra{beamon.ui.main.Main method}}

\begin{fulllineitems}
\phantomsection\label{\detokenize{api:beamon.ui.main.Main.start_visualizer_manual}}\pysiglinewithargsret{\sphinxbfcode{\sphinxupquote{start\_visualizer\_manual}}}{}{}
Starts visualizer\sphinxhyphen{}scene keybindings help table

\end{fulllineitems}

\index{tileView() (beamon.ui.main.Main method)@\spxentry{tileView()}\spxextra{beamon.ui.main.Main method}}

\begin{fulllineitems}
\phantomsection\label{\detokenize{api:beamon.ui.main.Main.tileView}}\pysiglinewithargsret{\sphinxbfcode{\sphinxupquote{tileView}}}{}{}
Changes Subwindows order to TileView

\end{fulllineitems}


\end{fulllineitems}



\subsection{UI.Visualizer}
\label{\detokenize{api:module-beamon.ui.callVisualizer}}\label{\detokenize{api:ui-visualizer}}\index{module@\spxentry{module}!beamon.ui.callVisualizer@\spxentry{beamon.ui.callVisualizer}}\index{beamon.ui.callVisualizer@\spxentry{beamon.ui.callVisualizer}!module@\spxentry{module}}\index{VisualizerMainForm (class in beamon.ui.callVisualizer)@\spxentry{VisualizerMainForm}\spxextra{class in beamon.ui.callVisualizer}}

\begin{fulllineitems}
\phantomsection\label{\detokenize{api:beamon.ui.callVisualizer.VisualizerMainForm}}\pysiglinewithargsret{\sphinxbfcode{\sphinxupquote{class }}\sphinxcode{\sphinxupquote{beamon.ui.callVisualizer.}}\sphinxbfcode{\sphinxupquote{VisualizerMainForm}}}{\emph{\DUrole{n}{parent\_ui}}, \emph{\DUrole{n}{parent}}, \emph{\DUrole{n}{m\_id}\DUrole{p}{:} \DUrole{n}{\sphinxhref{https://docs.python.org/3/library/functions.html\#int}{int}}}}{}
a class calls the compiled visualiser.py file and contains the backend functionality of the GUI

Author: Maher Albezem (maher.albezem{[}at{]}sla.rwth\sphinxhyphen{}aachen.de)
This class calls the compiled GUI visualiser.
The GUI is compiled with pyuic5 from {\color{red}\bfseries{}*}.ui to {\color{red}\bfseries{}*}.py extension
This class also contains all the back\sphinxhyphen{}end functionality of visualiser.

Warning: it is important to set the parent class as QWidget only because of QMdiArea functionality
in the referencing class
\index{change\_joints() (beamon.ui.callVisualizer.VisualizerMainForm method)@\spxentry{change\_joints()}\spxextra{beamon.ui.callVisualizer.VisualizerMainForm method}}

\begin{fulllineitems}
\phantomsection\label{\detokenize{api:beamon.ui.callVisualizer.VisualizerMainForm.change_joints}}\pysiglinewithargsret{\sphinxbfcode{\sphinxupquote{change\_joints}}}{}{}
Calls the joints table

\end{fulllineitems}

\index{change\_links() (beamon.ui.callVisualizer.VisualizerMainForm method)@\spxentry{change\_links()}\spxextra{beamon.ui.callVisualizer.VisualizerMainForm method}}

\begin{fulllineitems}
\phantomsection\label{\detokenize{api:beamon.ui.callVisualizer.VisualizerMainForm.change_links}}\pysiglinewithargsret{\sphinxbfcode{\sphinxupquote{change\_links}}}{}{}
Calls the links table

\end{fulllineitems}

\index{change\_nodes() (beamon.ui.callVisualizer.VisualizerMainForm method)@\spxentry{change\_nodes()}\spxextra{beamon.ui.callVisualizer.VisualizerMainForm method}}

\begin{fulllineitems}
\phantomsection\label{\detokenize{api:beamon.ui.callVisualizer.VisualizerMainForm.change_nodes}}\pysiglinewithargsret{\sphinxbfcode{\sphinxupquote{change\_nodes}}}{}{}
Calls the nodes table

\end{fulllineitems}

\index{closeEvent() (beamon.ui.callVisualizer.VisualizerMainForm method)@\spxentry{closeEvent()}\spxextra{beamon.ui.callVisualizer.VisualizerMainForm method}}

\begin{fulllineitems}
\phantomsection\label{\detokenize{api:beamon.ui.callVisualizer.VisualizerMainForm.closeEvent}}\pysiglinewithargsret{\sphinxbfcode{\sphinxupquote{closeEvent}}}{\emph{\DUrole{n}{self}}, \emph{\DUrole{n}{QCloseEvent}}}{}
\end{fulllineitems}

\index{context\_menu\_requested() (beamon.ui.callVisualizer.VisualizerMainForm method)@\spxentry{context\_menu\_requested()}\spxextra{beamon.ui.callVisualizer.VisualizerMainForm method}}

\begin{fulllineitems}
\phantomsection\label{\detokenize{api:beamon.ui.callVisualizer.VisualizerMainForm.context_menu_requested}}\pysiglinewithargsret{\sphinxbfcode{\sphinxupquote{context\_menu\_requested}}}{}{}
Context menu for visualizer

\end{fulllineitems}

\index{display\_log() (beamon.ui.callVisualizer.VisualizerMainForm method)@\spxentry{display\_log()}\spxextra{beamon.ui.callVisualizer.VisualizerMainForm method}}

\begin{fulllineitems}
\phantomsection\label{\detokenize{api:beamon.ui.callVisualizer.VisualizerMainForm.display_log}}\pysiglinewithargsret{\sphinxbfcode{\sphinxupquote{display\_log}}}{\emph{\DUrole{n}{message}}}{}
Displays a log message on the status bar of the Main window and saves the message to the log
@param message: String

\end{fulllineitems}

\index{init\_settings() (beamon.ui.callVisualizer.VisualizerMainForm method)@\spxentry{init\_settings()}\spxextra{beamon.ui.callVisualizer.VisualizerMainForm method}}

\begin{fulllineitems}
\phantomsection\label{\detokenize{api:beamon.ui.callVisualizer.VisualizerMainForm.init_settings}}\pysiglinewithargsret{\sphinxbfcode{\sphinxupquote{init\_settings}}}{}{}
Reads the settings from dataset and adjust visualization settings accordingly
If ther is no predefined settings default settings will be loaded

\end{fulllineitems}

\index{keyPressEvent() (beamon.ui.callVisualizer.VisualizerMainForm method)@\spxentry{keyPressEvent()}\spxextra{beamon.ui.callVisualizer.VisualizerMainForm method}}

\begin{fulllineitems}
\phantomsection\label{\detokenize{api:beamon.ui.callVisualizer.VisualizerMainForm.keyPressEvent}}\pysiglinewithargsret{\sphinxbfcode{\sphinxupquote{keyPressEvent}}}{\emph{\DUrole{n}{self}}, \emph{\DUrole{n}{QKeyEvent}}}{}
\end{fulllineitems}

\index{reset\_inputfields() (beamon.ui.callVisualizer.VisualizerMainForm method)@\spxentry{reset\_inputfields()}\spxextra{beamon.ui.callVisualizer.VisualizerMainForm method}}

\begin{fulllineitems}
\phantomsection\label{\detokenize{api:beamon.ui.callVisualizer.VisualizerMainForm.reset_inputfields}}\pysiglinewithargsret{\sphinxbfcode{\sphinxupquote{reset\_inputfields}}}{}{}
Sets all buttons in the GUI to normal state again

\end{fulllineitems}

\index{start\_grid\_settings() (beamon.ui.callVisualizer.VisualizerMainForm method)@\spxentry{start\_grid\_settings()}\spxextra{beamon.ui.callVisualizer.VisualizerMainForm method}}

\begin{fulllineitems}
\phantomsection\label{\detokenize{api:beamon.ui.callVisualizer.VisualizerMainForm.start_grid_settings}}\pysiglinewithargsret{\sphinxbfcode{\sphinxupquote{start\_grid\_settings}}}{}{}
Call grid settings ui and block the main ui

\end{fulllineitems}

\index{start\_results\_view\_settings() (beamon.ui.callVisualizer.VisualizerMainForm method)@\spxentry{start\_results\_view\_settings()}\spxextra{beamon.ui.callVisualizer.VisualizerMainForm method}}

\begin{fulllineitems}
\phantomsection\label{\detokenize{api:beamon.ui.callVisualizer.VisualizerMainForm.start_results_view_settings}}\pysiglinewithargsret{\sphinxbfcode{\sphinxupquote{start\_results\_view\_settings}}}{}{}
Call results view settings ui and block the main ui

\end{fulllineitems}

\index{start\_view\_settings() (beamon.ui.callVisualizer.VisualizerMainForm method)@\spxentry{start\_view\_settings()}\spxextra{beamon.ui.callVisualizer.VisualizerMainForm method}}

\begin{fulllineitems}
\phantomsection\label{\detokenize{api:beamon.ui.callVisualizer.VisualizerMainForm.start_view_settings}}\pysiglinewithargsret{\sphinxbfcode{\sphinxupquote{start\_view\_settings}}}{}{}
Call view settings ui and block the main ui

\end{fulllineitems}

\index{table\_state\_changed() (beamon.ui.callVisualizer.VisualizerMainForm method)@\spxentry{table\_state\_changed()}\spxextra{beamon.ui.callVisualizer.VisualizerMainForm method}}

\begin{fulllineitems}
\phantomsection\label{\detokenize{api:beamon.ui.callVisualizer.VisualizerMainForm.table_state_changed}}\pysiglinewithargsret{\sphinxbfcode{\sphinxupquote{table\_state\_changed}}}{\emph{\DUrole{n}{state}}}{}
This gets called when dock widget gets toggled
:param state: visibility state of the dock widget

\end{fulllineitems}

\index{treeClicked() (beamon.ui.callVisualizer.VisualizerMainForm method)@\spxentry{treeClicked()}\spxextra{beamon.ui.callVisualizer.VisualizerMainForm method}}

\begin{fulllineitems}
\phantomsection\label{\detokenize{api:beamon.ui.callVisualizer.VisualizerMainForm.treeClicked}}\pysiglinewithargsret{\sphinxbfcode{\sphinxupquote{treeClicked}}}{\emph{\DUrole{n}{index}}}{}
Gets called when an item in the tree is clicked or entered
:param index: QModelItem

\end{fulllineitems}


\end{fulllineitems}



\subsection{UI.Visualization}
\label{\detokenize{api:module-beamon.ui.pyvista_widget}}\label{\detokenize{api:ui-visualization}}\index{module@\spxentry{module}!beamon.ui.pyvista\_widget@\spxentry{beamon.ui.pyvista\_widget}}\index{beamon.ui.pyvista\_widget@\spxentry{beamon.ui.pyvista\_widget}!module@\spxentry{module}}\index{PyVistaWidget (class in beamon.ui.pyvista\_widget)@\spxentry{PyVistaWidget}\spxextra{class in beamon.ui.pyvista\_widget}}

\begin{fulllineitems}
\phantomsection\label{\detokenize{api:beamon.ui.pyvista_widget.PyVistaWidget}}\pysiglinewithargsret{\sphinxbfcode{\sphinxupquote{class }}\sphinxcode{\sphinxupquote{beamon.ui.pyvista\_widget.}}\sphinxbfcode{\sphinxupquote{PyVistaWidget}}}{\emph{\DUrole{n}{parent}}}{}~\index{init\_plot() (beamon.ui.pyvista\_widget.PyVistaWidget method)@\spxentry{init\_plot()}\spxextra{beamon.ui.pyvista\_widget.PyVistaWidget method}}

\begin{fulllineitems}
\phantomsection\label{\detokenize{api:beamon.ui.pyvista_widget.PyVistaWidget.init_plot}}\pysiglinewithargsret{\sphinxbfcode{\sphinxupquote{init\_plot}}}{}{}
Initiates the plots vital objects like the grid and the coordinate system

\end{fulllineitems}

\index{init\_settings() (beamon.ui.pyvista\_widget.PyVistaWidget method)@\spxentry{init\_settings()}\spxextra{beamon.ui.pyvista\_widget.PyVistaWidget method}}

\begin{fulllineitems}
\phantomsection\label{\detokenize{api:beamon.ui.pyvista_widget.PyVistaWidget.init_settings}}\pysiglinewithargsret{\sphinxbfcode{\sphinxupquote{init\_settings}}}{}{}
init default settings and states

\end{fulllineitems}

\index{update\_boundary\_conditions() (beamon.ui.pyvista\_widget.PyVistaWidget method)@\spxentry{update\_boundary\_conditions()}\spxextra{beamon.ui.pyvista\_widget.PyVistaWidget method}}

\begin{fulllineitems}
\phantomsection\label{\detokenize{api:beamon.ui.pyvista_widget.PyVistaWidget.update_boundary_conditions}}\pysiglinewithargsret{\sphinxbfcode{\sphinxupquote{update\_boundary\_conditions}}}{\emph{\DUrole{n}{m\_id}}}{}
scatter two blue cones for momentum and orange blue cones for translation on each node with bc

\end{fulllineitems}

\index{update\_displaced\_nodes() (beamon.ui.pyvista\_widget.PyVistaWidget method)@\spxentry{update\_displaced\_nodes()}\spxextra{beamon.ui.pyvista\_widget.PyVistaWidget method}}

\begin{fulllineitems}
\phantomsection\label{\detokenize{api:beamon.ui.pyvista_widget.PyVistaWidget.update_displaced_nodes}}\pysiglinewithargsret{\sphinxbfcode{\sphinxupquote{update\_displaced\_nodes}}}{}{}
Updates displaced nodes from displaced nodes coordinates in database

\end{fulllineitems}

\index{update\_element\_displacements() (beamon.ui.pyvista\_widget.PyVistaWidget method)@\spxentry{update\_element\_displacements()}\spxextra{beamon.ui.pyvista\_widget.PyVistaWidget method}}

\begin{fulllineitems}
\phantomsection\label{\detokenize{api:beamon.ui.pyvista_widget.PyVistaWidget.update_element_displacements}}\pysiglinewithargsret{\sphinxbfcode{\sphinxupquote{update\_element\_displacements}}}{}{}
Updates elements displacements

\end{fulllineitems}

\index{update\_elements() (beamon.ui.pyvista\_widget.PyVistaWidget method)@\spxentry{update\_elements()}\spxextra{beamon.ui.pyvista\_widget.PyVistaWidget method}}

\begin{fulllineitems}
\phantomsection\label{\detokenize{api:beamon.ui.pyvista_widget.PyVistaWidget.update_elements}}\pysiglinewithargsret{\sphinxbfcode{\sphinxupquote{update\_elements}}}{\emph{\DUrole{n}{m\_id}\DUrole{p}{:} \DUrole{n}{\sphinxhref{https://docs.python.org/3/library/functions.html\#int}{int}}}, \emph{\DUrole{n}{opacity}\DUrole{o}{=}\DUrole{default_value}{1.0}}}{}
Updates elements according to new nodes coordinates from database

\end{fulllineitems}

\index{update\_elements\_orientation() (beamon.ui.pyvista\_widget.PyVistaWidget method)@\spxentry{update\_elements\_orientation()}\spxextra{beamon.ui.pyvista\_widget.PyVistaWidget method}}

\begin{fulllineitems}
\phantomsection\label{\detokenize{api:beamon.ui.pyvista_widget.PyVistaWidget.update_elements_orientation}}\pysiglinewithargsret{\sphinxbfcode{\sphinxupquote{update\_elements\_orientation}}}{\emph{\DUrole{n}{m\_id}\DUrole{p}{:} \DUrole{n}{\sphinxhref{https://docs.python.org/3/library/functions.html\#int}{int}}}}{}
Updates elements local coordinate systems according to element orientation from database

\end{fulllineitems}

\index{update\_global() (beamon.ui.pyvista\_widget.PyVistaWidget method)@\spxentry{update\_global()}\spxextra{beamon.ui.pyvista\_widget.PyVistaWidget method}}

\begin{fulllineitems}
\phantomsection\label{\detokenize{api:beamon.ui.pyvista_widget.PyVistaWidget.update_global}}\pysiglinewithargsret{\sphinxbfcode{\sphinxupquote{update\_global}}}{\emph{\DUrole{n}{m\_id}\DUrole{p}{:} \DUrole{n}{\sphinxhref{https://docs.python.org/3/library/functions.html\#int}{int}}}, \emph{\DUrole{n}{reset\_view}\DUrole{o}{=}\DUrole{default_value}{True}}}{}
Central update for all visualization objects based on their hierarchic dependencies and their data sets.
@see machine state diagram ‘user interaction’ in documentation
\begin{quote}\begin{description}
\item[{Parameters}] \leavevmode\begin{itemize}
\item {} 
\sphinxstyleliteralstrong{\sphinxupquote{m\_id}} (\sphinxstyleliteralemphasis{\sphinxupquote{integer}}) \textendash{} model number

\item {} 
\sphinxstyleliteralstrong{\sphinxupquote{reset\_view}} (\sphinxstyleliteralemphasis{\sphinxupquote{boolean}}) \textendash{} reset camera view

\end{itemize}

\end{description}\end{quote}

\end{fulllineitems}

\index{update\_grid() (beamon.ui.pyvista\_widget.PyVistaWidget method)@\spxentry{update\_grid()}\spxextra{beamon.ui.pyvista\_widget.PyVistaWidget method}}

\begin{fulllineitems}
\phantomsection\label{\detokenize{api:beamon.ui.pyvista_widget.PyVistaWidget.update_grid}}\pysiglinewithargsret{\sphinxbfcode{\sphinxupquote{update\_grid}}}{}{}
Update grid data according to grid settings such as vector normal, size, tick and type

\end{fulllineitems}

\index{update\_labels() (beamon.ui.pyvista\_widget.PyVistaWidget method)@\spxentry{update\_labels()}\spxextra{beamon.ui.pyvista\_widget.PyVistaWidget method}}

\begin{fulllineitems}
\phantomsection\label{\detokenize{api:beamon.ui.pyvista_widget.PyVistaWidget.update_labels}}\pysiglinewithargsret{\sphinxbfcode{\sphinxupquote{update\_labels}}}{\emph{\DUrole{n}{reset\_view}\DUrole{p}{:} \DUrole{n}{\sphinxhref{https://docs.python.org/3/library/functions.html\#bool}{bool}}}, \emph{\DUrole{n}{m\_id}\DUrole{p}{:} \DUrole{n}{\sphinxhref{https://docs.python.org/3/library/functions.html\#int}{int}}}}{}
Updates label positions if they should be shown

\end{fulllineitems}

\index{update\_nodes() (beamon.ui.pyvista\_widget.PyVistaWidget method)@\spxentry{update\_nodes()}\spxextra{beamon.ui.pyvista\_widget.PyVistaWidget method}}

\begin{fulllineitems}
\phantomsection\label{\detokenize{api:beamon.ui.pyvista_widget.PyVistaWidget.update_nodes}}\pysiglinewithargsret{\sphinxbfcode{\sphinxupquote{update\_nodes}}}{\emph{\DUrole{n}{m\_id}\DUrole{p}{:} \DUrole{n}{\sphinxhref{https://docs.python.org/3/library/functions.html\#int}{int}}}}{}
Updates nodes coordinates from database

\end{fulllineitems}

\index{update\_section\_forces() (beamon.ui.pyvista\_widget.PyVistaWidget method)@\spxentry{update\_section\_forces()}\spxextra{beamon.ui.pyvista\_widget.PyVistaWidget method}}

\begin{fulllineitems}
\phantomsection\label{\detokenize{api:beamon.ui.pyvista_widget.PyVistaWidget.update_section_forces}}\pysiglinewithargsret{\sphinxbfcode{\sphinxupquote{update\_section\_forces}}}{\emph{\DUrole{n}{m\_id}\DUrole{p}{:} \DUrole{n}{\sphinxhref{https://docs.python.org/3/library/functions.html\#int}{int}}}}{}
Updates elements section forces

\end{fulllineitems}

\index{update\_support\_forces() (beamon.ui.pyvista\_widget.PyVistaWidget method)@\spxentry{update\_support\_forces()}\spxextra{beamon.ui.pyvista\_widget.PyVistaWidget method}}

\begin{fulllineitems}
\phantomsection\label{\detokenize{api:beamon.ui.pyvista_widget.PyVistaWidget.update_support_forces}}\pysiglinewithargsret{\sphinxbfcode{\sphinxupquote{update\_support\_forces}}}{\emph{\DUrole{n}{m\_id}\DUrole{p}{:} \DUrole{n}{\sphinxhref{https://docs.python.org/3/library/functions.html\#int}{int}}}}{}
Updates nodes support forces

\end{fulllineitems}


\end{fulllineitems}



\subsection{Indices and tables}
\label{\detokenize{api:indices-and-tables}}\begin{itemize}
\item {} 
\DUrole{xref,std,std-ref}{genindex}

\item {} 
\DUrole{xref,std,std-ref}{modindex}

\item {} 
\DUrole{xref,std,std-ref}{search}

\end{itemize}


\section{Contributing}
\label{\detokenize{contrib:contributing}}\label{\detokenize{contrib::doc}}

\subsection{Programming Environment}
\label{\detokenize{env:programming-environment}}\label{\detokenize{env::doc}}
Making your environment optimized to deal with large projects such as Beamon is necessary to ensure low\sphinxhyphen{}cost
maintenance.


\subsubsection{Python Environment}
\label{\detokenize{env:python-environment}}
Windows users should install the Anaconda environment.

\begin{sphinxadmonition}{note}{Note:}
All packages should be installed via pip and not from conda.
\end{sphinxadmonition}

Some anaconda packages are not up to date with the latest releases and will cause compatibility problems.
Nevertheless using Anaconda has proven to be most efficient.

Please Install Anaconda from the \sphinxhref{https://www.anaconda.com/}{official website}


\subsubsection{IDE}
\label{\detokenize{env:ide}}
The most convenient IDE for projects like Beamon is PyCharm, especially PyCharm Professional.

Please Install PyCharm from the \sphinxhref{https://www.jetbrains.com/pycharm/}{official website}


\subsubsection{Project Requirements}
\label{\detokenize{env:project-requirements}}
“Package requirements” are the \sphinxstylestrong{absolute minimum} of python packages that are required to perform \sphinxstylestrong{all}
operation in the project.

All required packages are in \sphinxstyleemphasis{requirements.txt} file and can be installed via pip using the command:

\begin{sphinxVerbatim}[commandchars=\\\{\}]
\PYG{n}{pip} \PYG{n}{install} \PYG{o}{\PYGZhy{}}\PYG{n}{r} \PYG{n}{requirements}\PYG{o}{.}\PYG{n}{txt}
\end{sphinxVerbatim}

or using the command:

\begin{sphinxVerbatim}[commandchars=\\\{\}]
\PYG{n}{make} \PYG{n}{install\PYGZus{}requirements}
\end{sphinxVerbatim}

on your environment console.

For windows users, \sphinxstylestrong{pywin32}, which will be used to build Windows executables, will be required in addition
to \sphinxstyleemphasis{requirements.txt} packages.

\begin{sphinxadmonition}{warning}{Warning:}
Package requirements should be updated regularly to guarantee that latest package releases work.
\end{sphinxadmonition}


\subsection{Project Structure}
\label{\detokenize{structure:project-structure}}\label{\detokenize{structure::doc}}
For structuring python projects read the Hitchhiker’s Guid to Python \sphinxhref{https://docs.python-guide.org/writing/structure/}{here}.


\subsubsection{Directives Structure}
\label{\detokenize{structure:directives-structure}}
\begin{sphinxVerbatim}[commandchars=\\\{\}]
\PYG{g+gp}{\PYGZgt{}\PYGZgt{}\PYGZgt{} }\PYG{n}{tree} \PYG{o}{/}\PYG{n}{F}
\PYG{g+go}{.}
\PYG{g+go}{|   .gitattributes}
\PYG{g+go}{│   .gitignore}
\PYG{g+go}{│   LICENSE}
\PYG{g+go}{│   make.bat}
\PYG{g+go}{│   make.sh}
\PYG{g+go}{│   README.md}
\PYG{g+go}{│   requirements.txt}
\PYG{g+go}{│   runbeamon.spec}
\PYG{g+go}{│   setup.py}
\PYG{g+go}{├───beamon}
\PYG{g+go}{│   ├───resources}
\PYG{g+go}{│   │   ├───custom\PYGZus{}icons}
\PYG{g+go}{│   │   └───ui}
\PYG{g+go}{│   ├───ui}
\PYG{g+go}{│   │   ├───expansions}
\PYG{g+go}{│   │   ├───settings}
\PYG{g+go}{│   │   ├───tables}
\PYG{g+go}{├───build}
\PYG{g+go}{├───dist}
\PYG{g+go}{├───docs}
\PYG{g+go}{│   ├───build}
\PYG{g+go}{│   ├───release}
\PYG{g+go}{│   └───source}
\PYG{g+go}{│       ├───resources}
\PYG{g+go}{│       │   └───figures}
\PYG{g+go}{│       └───\PYGZus{}static}
\PYG{g+go}{├───releases}
\PYG{g+go}{│   └───alpha}
\PYG{g+go}{├───resources}
\PYG{g+go}{└───tests}
\end{sphinxVerbatim}


\subsubsection{Discription}
\label{\detokenize{structure:discription}}

\paragraph{The Actual Module}
\label{\detokenize{structure:the-actual-module}}
This is where the core focus of the repository is.


\begin{savenotes}\sphinxattablestart
\centering
\begin{tabulary}{\linewidth}[t]{|T|T|}
\hline

Location
&
./beamon/
\\
\hline
Purpose
&
Programs code
\\
\hline
\end{tabulary}
\par
\sphinxattableend\end{savenotes}

When user lunches Beamon as a python package using python interpreter, e.g.

\begin{sphinxVerbatim}[commandchars=\\\{\}]
\PYG{n}{python} \PYG{o}{\PYGZhy{}}\PYG{n}{m} \PYG{n}{beamon}
\end{sphinxVerbatim}

file \_\_main\_\_.py will be executed.
On the other hand \_\_main.py will be executed if beamon executable e.g. \sphinxtitleref{runbeamon.exe} is called using bash


\paragraph{Release Files}
\label{\detokenize{structure:release-files}}

\begin{savenotes}\sphinxattablestart
\centering
\begin{tabulary}{\linewidth}[t]{|T|T|}
\hline

Location
&
./releases
\\
\hline
Purpose
&
All executables releases
\\
\hline
\end{tabulary}
\par
\sphinxattableend\end{savenotes}


\begin{savenotes}\sphinxattablestart
\centering
\begin{tabulary}{\linewidth}[t]{|T|T|}
\hline

Location
&
./docs/release
\\
\hline
Purpose
&
latest documentation release
\\
\hline
\end{tabulary}
\par
\sphinxattableend\end{savenotes}


\paragraph{Resources}
\label{\detokenize{structure:resources}}

\begin{savenotes}\sphinxattablestart
\centering
\begin{tabulary}{\linewidth}[t]{|T|T|}
\hline

Location
&
./beamon/resources
\\
\hline
Purpose
&
program dependent resources for running
\\
\hline
\end{tabulary}
\par
\sphinxattableend\end{savenotes}


\begin{savenotes}\sphinxattablestart
\centering
\begin{tabulary}{\linewidth}[t]{|T|T|}
\hline

Location
&
./resources
\\
\hline
Purpose
&
project dependent resources
\\
\hline
\end{tabulary}
\par
\sphinxattableend\end{savenotes}


\paragraph{License}
\label{\detokenize{structure:license}}

\begin{savenotes}\sphinxattablestart
\centering
\begin{tabulary}{\linewidth}[t]{|T|T|}
\hline

Location
&
./LICENSE
\\
\hline
Purpose
&
Copyright
\\
\hline
\end{tabulary}
\par
\sphinxattableend\end{savenotes}


\paragraph{Setup.py}
\label{\detokenize{structure:setup-py}}

\begin{savenotes}\sphinxattablestart
\centering
\begin{tabulary}{\linewidth}[t]{|T|T|}
\hline

Location
&
./setup.py
\\
\hline
Purpose
&
Package and distribution management.
\\
\hline
\end{tabulary}
\par
\sphinxattableend\end{savenotes}


\paragraph{Make}
\label{\detokenize{structure:make}}

\begin{savenotes}\sphinxattablestart
\centering
\begin{tabulary}{\linewidth}[t]{|T|T|}
\hline

Location
&
./make.bat and ./make.sh
\\
\hline
Purpose
&
make for both linux and windows
\\
\hline
\end{tabulary}
\par
\sphinxattableend\end{savenotes}


\paragraph{PyInstaller Specification:}
\label{\detokenize{structure:pyinstaller-specification}}
This file is auto generated from make and excluded from remote repository. It can be modified for testing purposes.


\begin{savenotes}\sphinxattablestart
\centering
\begin{tabulary}{\linewidth}[t]{|T|T|}
\hline

Location
&
./runbeamon.spec
\\
\hline
Purpose
&
secure distribution management.
\\
\hline
\end{tabulary}
\par
\sphinxattableend\end{savenotes}


\paragraph{Distribution files}
\label{\detokenize{structure:distribution-files}}
This directory is also autogenerated and excluded from remote repository.


\begin{savenotes}\sphinxattablestart
\centering
\begin{tabulary}{\linewidth}[t]{|T|T|}
\hline

Location
&
./dist/
\\
\hline
Purpose
&
distributing Beamon as package or executable
\\
\hline
\end{tabulary}
\par
\sphinxattableend\end{savenotes}


\paragraph{Requirements File}
\label{\detokenize{structure:requirements-file}}
Contains all dependencies for the project.


\begin{savenotes}\sphinxattablestart
\centering
\begin{tabulary}{\linewidth}[t]{|T|T|}
\hline

Location
&
./requirements.txt
\\
\hline
Purpose
&
Developers evironment dependencies
\\
\hline
\end{tabulary}
\par
\sphinxattableend\end{savenotes}


\paragraph{Documentation}
\label{\detokenize{structure:documentation}}
It is worth noting that the \sphinxstylestrong{release} directory inside \sphinxstylestrong{docs} is the latest documentation release.


\begin{savenotes}\sphinxattablestart
\centering
\begin{tabulary}{\linewidth}[t]{|T|T|}
\hline

Location
&
./docs/
\\
\hline
Purpose
&
Package reference documentation.
\\
\hline
\end{tabulary}
\par
\sphinxattableend\end{savenotes}


\paragraph{Test Suite}
\label{\detokenize{structure:test-suite}}

\begin{savenotes}\sphinxattablestart
\centering
\begin{tabulary}{\linewidth}[t]{|T|T|}
\hline

Location
&
./tests/
\\
\hline
Purpose
&
Package integration and unit tests.
\\
\hline
\end{tabulary}
\par
\sphinxattableend\end{savenotes}


\subsection{Implementation Details}
\label{\detokenize{implementation:implementation-details}}\label{\detokenize{implementation::doc}}
This chapter include implementation details from database structure to UML class diagrams.
It should help develop an understanding of the project for newcomers.


\subsubsection{The Database}
\label{\detokenize{implementation:the-database}}
The Database is implemented using SQLite3 in Python. Using SQLite database files as project files is quite convenient.
Therefore we will be referring to database files with project files.


\paragraph{Database Structure \sphinxhyphen{} Original Idea and It’s Problems}
\label{\detokenize{implementation:database-structure-original-idea-and-it-s-problems}}
The following is an entity\sphinxhyphen{}relationship diagram (ER\sphinxhyphen{}Diagram) of the first sketch introduced in the bachelor thesis.

\begin{figure}[htbp]
\centering
\capstart

\noindent\sphinxincludegraphics{{er_diagram_alpha}.png}
\caption{ER\sphinxhyphen{}Diagram of the alpha release.}\label{\detokenize{implementation:id1}}\end{figure}

Some entities have composite attributes. For example, Links has attribute \(\vec{v}\), which practically
have to be divided up into v\_x, v\_y, and v\_z.
\sphinxstyleemphasis{dof} represents which degree of freedom is locked/unlocked (0/1) on a node to save boundary conditions.
\sphinxstyleemphasis{edof} is element degree of freedom which consists of 12 integer numbers.

This design is problematic due to the following reasons:
\begin{enumerate}
\sphinxsetlistlabels{\arabic}{enumi}{enumii}{}{.}%
\item {} 
Models should be parent of Nodes and not of Elements. If Models are defined only over Elements, all nodes that are not connected to any elements will be excluded from all models.

\item {} 
Visualizer settings should be child from Models.

\item {} 
Simulation results have no entities.

\item {} 
Names are to be renamed according to the naming convention. Also Links should be named Elements.

\end{enumerate}

After solving the above issues and modeling the database in \sphinxhref{https://www.jetbrains.com/datagrip/}{DataGrip} the database
will be more normalized.

The following diagram illustrates the modified version in an EER\sphinxhyphen{}Diagram.
The central table \sphinxstyleemphasis{model} is root of most other tables (except \sphinxstyleemphasis{profile} and \sphinxstyleemphasis{sqlite\_master}).
For each model there is \sphinxstyleemphasis{setting} and \sphinxstyleemphasis{node}. This way \sphinxstyleemphasis{node}
plays a central role in identifying models structure. That’s because everything else in a structure, ex. \sphinxstyleemphasis{element},
depends on \sphinxstyleemphasis{node} to be defined.

\sphinxstyleemphasis{node\_result} and \sphinxstyleemphasis{e\_result} are children of \sphinxstyleemphasis{node} and \sphinxstyleemphasis{element} respectively. All foreign keys of all tables except
\sphinxstyleemphasis{profile\_id} in \sphinxstyleemphasis{element} have cascade delete option. \sphinxstyleemphasis{profile\_id} in \sphinxstyleemphasis{element} have Null on delete option because
elements still exist if there are no element properties defined. Those elements will be excluded from simulation easily
as well as nodes with no elements attached to them.

\begin{figure}[htbp]
\centering

\noindent\sphinxincludegraphics{{EER-Diagramm-Beta}.png}
\end{figure}


\paragraph{SQLite Queries}
\label{\detokenize{implementation:sqlite-queries}}
The following sql query shows how to pass parameters to an sql query.
cur is connection object cursor.

\begin{sphinxVerbatim}[commandchars=\\\{\}]
\PYG{k+kn}{import} \PYG{n+nn}{sqlite3}
\PYG{n}{conn} \PYG{o}{=} \PYG{n}{sqlite3}\PYG{o}{.}\PYG{n}{connect}\PYG{p}{(}\PYG{l+s+s2}{\PYGZdq{}}\PYG{l+s+s2}{path\PYGZus{}to\PYGZus{}database}\PYG{l+s+s2}{\PYGZdq{}}\PYG{p}{)}
\PYG{n}{cur} \PYG{o}{=} \PYG{n}{conn}\PYG{o}{.}\PYG{n}{cursor}\PYG{p}{(}\PYG{p}{)}
\PYG{n}{cur}\PYG{o}{.}\PYG{n}{execute}\PYG{p}{(}\PYG{l+s+s1}{\PYGZsq{}}\PYG{l+s+s1}{SELECT id FROM Nodes WHERE x=}\PYG{l+s+si}{\PYGZob{}x\PYGZcb{}}\PYG{l+s+s1}{ and y=}\PYG{l+s+si}{\PYGZob{}y\PYGZcb{}}\PYG{l+s+s1}{ and z=}\PYG{l+s+si}{\PYGZob{}z\PYGZcb{}}\PYG{l+s+s1}{;}\PYG{l+s+s1}{\PYGZsq{}}\PYG{o}{.}\PYG{n}{format}\PYG{p}{(}\PYG{n}{x}\PYG{o}{=}\PYG{l+m+mi}{0}\PYG{p}{,} \PYG{n}{y}\PYG{o}{=}\PYG{l+m+mf}{0.1}\PYG{p}{,} \PYG{n}{z}\PYG{o}{=}\PYG{o}{\PYGZhy{}}\PYG{l+m+mi}{1}\PYG{p}{)}\PYG{p}{)}
\end{sphinxVerbatim}

Another way of passing parameters to sql queries is using the tuple params.

\begin{sphinxVerbatim}[commandchars=\\\{\}]
\PYG{n}{conn} \PYG{o}{=} \PYG{n}{sqlite3}\PYG{o}{.}\PYG{n}{connect}\PYG{p}{(}\PYG{l+s+s2}{\PYGZdq{}}\PYG{l+s+s2}{path\PYGZus{}to\PYGZus{}database}\PYG{l+s+s2}{\PYGZdq{}}\PYG{p}{)}
\PYG{n}{cur} \PYG{o}{=} \PYG{n}{conn}\PYG{o}{.}\PYG{n}{cursor}\PYG{p}{(}\PYG{p}{)}
\PYG{n}{params} \PYG{o}{=} \PYG{p}{(}\PYG{l+m+mi}{0}\PYG{p}{,} \PYG{l+m+mf}{0.1}\PYG{p}{,} \PYG{o}{\PYGZhy{}}\PYG{l+m+mi}{1}\PYG{p}{)}
\PYG{n}{cur}\PYG{o}{.}\PYG{n}{execute}\PYG{p}{(}\PYG{l+s+s1}{\PYGZsq{}}\PYG{l+s+s1}{SELECT id FROM Nodes WHERE x=}\PYG{l+s+s1}{\PYGZob{}}\PYG{l+s+s1}{?\PYGZcb{} and y=}\PYG{l+s+s1}{\PYGZob{}}\PYG{l+s+s1}{?\PYGZcb{} and z=}\PYG{l+s+s1}{\PYGZob{}}\PYG{l+s+s1}{?\PYGZcb{};}\PYG{l+s+s1}{\PYGZsq{}}\PYG{p}{,} \PYG{n}{params}\PYG{p}{)}
\end{sphinxVerbatim}

You can also obtain panda Dataframes using sql queries. It is very convenient especially when using QTableViews to show
data tables.

\begin{sphinxVerbatim}[commandchars=\\\{\}]
\PYG{n}{conn} \PYG{o}{=} \PYG{n}{sqlite3}\PYG{o}{.}\PYG{n}{connect}\PYG{p}{(}\PYG{l+s+s2}{\PYGZdq{}}\PYG{l+s+s2}{path\PYGZus{}to\PYGZus{}database}\PYG{l+s+s2}{\PYGZdq{}}\PYG{p}{)}
\PYG{k+kn}{import} \PYG{n+nn}{pandas}\PYG{n+nn}{.}\PYG{n+nn}{io}\PYG{n+nn}{.}\PYG{n+nn}{sql} \PYG{k}{as} \PYG{n+nn}{sql}
\PYG{n}{sql}\PYG{o}{.}\PYG{n}{read\PYGZus{}sql}\PYG{p}{(}\PYG{l+s+s2}{\PYGZdq{}\PYGZdq{}\PYGZdq{}}\PYG{l+s+s2}{SELECT E, G, A, Iy, Iz, kv, k from Profile}\PYG{l+s+s2}{\PYGZdq{}\PYGZdq{}\PYGZdq{}}\PYG{p}{,} \PYG{n}{conn}\PYG{p}{)}
\end{sphinxVerbatim}


\paragraph{Problems with SQL Queries}
\label{\detokenize{implementation:problems-with-sql-queries}}
The Database class is heavily dependent on underlying database design. This is problematic and could cause high costs and efficiency problems in the future.  Each change in the Database causes a chain of changes in SQL Queries.
I tried to use some design patterns to solve this issue.

Using Command in association with Memento resulted in a too complex design that solved some future issues but not the dependency problem.
Memento pattern helps saving the state of the database on each change. This way undo/redo implementation will be possible.
Command pattern could help making all commands the database saved in separate classes. This way each command has it’s
own responsibility and decouple the database from the business logic.

The following diagram illustrates this approach.

\begin{figure}[htbp]
\centering

\noindent\sphinxincludegraphics{{Command_Memento}.png}
\end{figure}

Invoker Class is responsible for firing the execution method of each command. A history of commands can be saved in a stack to record each change to the database and undo those changes when necessary.
This way the business logic of the original database Class is split between invoker and commands. But this will result in too many command classes.

A different Design could solve this problem. I made For each entity in the database a class using the interface Table and used it with the Command pattern (see diagram below). This way only Tables know how to use commands and only commands know how to access the database. But this is also problematic because Tables are very dependent. Changing one Table could cause a cascade of changes in other tables.

\begin{figure}[htbp]
\centering

\noindent\sphinxincludegraphics{{Database_System_v1}.png}
\end{figure}

Approaching this problem from a different angle resulted in using an ORM\sphinxhyphen{}Solution. ORM stands for Object Relational Mapping and is intended to tackle a common issue called \sphinxhref{https://en.wikipedia.org/wiki/Object\%E2\%80\%93relational\_impedance\_mismatch}{Object\sphinxhyphen{}relational impedance mismatch}.
The following design illustrates how this could be implemented.

\begin{figure}[htbp]
\centering

\noindent\sphinxincludegraphics{{Database_ORM}.png}
\end{figure}


\paragraph{Reading Geometry Files Efficiently}
\label{\detokenize{implementation:reading-geometry-files-efficiently}}
Geometry files are, for modifiability purposes, text files with ASCII characters.
Reading large text files into SQLite database could cause run\sphinxhyphen{}time problems.
\sphinxstyleemphasis{import\_text} method in \sphinxstyleemphasis{Database} class avoids those problems by firstly reading the text file in a buffered CSV file,
which will be transformed into pandas dataframe, which is imported eventually into the database efficiently.

The following activity diagram summarizes the steps taken in \sphinxstyleemphasis{import\_text} to import geometry files.

\begin{figure}[htbp]
\centering
\capstart

\noindent\sphinxincludegraphics{{activity_import_text}.png}
\caption{UML activity diagram showing steps to import geometry files in activity import\_text.}\label{\detokenize{implementation:id2}}\end{figure}


\subsubsection{PySide2 or PyQt5}
\label{\detokenize{implementation:pyside2-or-pyqt5}}
PySid2 and PyQt5 have almost the same api.
But it is worth mentioning that using PyQt5 is more beneficial in the long run.
PySide development lagged behind PyQt and PySide2 supports only Linux and MacOs Platforms.
On the Other hand PySide have very convenient Libraries for 3D animations that PyQt5 dont have.

Eventually PyQt5 was chosen due to major popularity and community support.


\subsubsection{Package vs. Executable}
\label{\detokenize{implementation:package-vs-executable}}
Running Beamon as a package is not much challenging as an Executable that has been securely compressed using
e.g. PyInstaller (among other).

In some cases you need to differentiate between the two situations.
Some dependencies have to be formatted first before they get used. For example all .ui files will be formatted
using pyuic5 from xml to python at every run in the main script. This will prevent mistakes like forgetting
to update the user interface that is being used after modification.
In the Executable version of the program this is no longer needed. The program will be in a frozen state and there
are no updates to any files anymore.

For this reason there are two main scripts for each of the program versions
\sphinxstylestrong{\_\_main\_\_.py} for Beamon as a package and \sphinxstylestrong{\_\_main.py} for Beamon as an executable.

\sphinxstylestrong{\_\_main.py} has no formatting commands whereas \sphinxstylestrong{\_\_main\_\_.py} uses pyuic5 module to format local files like .ui and
.rst


\subsubsection{Class Structure (alpha release)}
\label{\detokenize{implementation:class-structure-alpha-release}}
The following class diagram shows the whole class structure in Beamon project.

\begin{figure}[htbp]
\centering
\capstart

\noindent\sphinxincludegraphics{{class_complex}.jpg}
\caption{UML class diagram showing Beamon project structure.}\label{\detokenize{implementation:id3}}\end{figure}

There are two main scripts \sphinxstyleemphasis{\_\_main\_\_} and \sphinxstyleemphasis{\_\_main} that instantiate the application.
\sphinxstyleemphasis{\_\_main is for executables and does not contain formatting instructions. Both main scripts are connected to the *Main}
GUI class.

\sphinxstyleemphasis{Main} resides in the \sphinxstyleemphasis{ui} package, which contains everything related to the user interface.
As you can see \sphinxstyleemphasis{Main} inherits from \sphinxstyleemphasis{QMainWindow}, which is a single tone PyQt5 class.

\sphinxstyleemphasis{VisualizerMainForm} class is responsible for visualization. It contains \sphinxstyleemphasis{PyVistaWidget} which is the visualization
component based on PyVista and should therefore inherit from \sphinxstyleemphasis{QtInteractor}.
Furthermore, the visualizer can call helper dialogs like \sphinxstyleemphasis{GridSettingsMainForm} to control grid settings.

\sphinxstyleemphasis{BeamSizeMainForm} class can be called from \sphinxstyleemphasis{Main}. BeamSize is an expansion of the basic functionality of Beamon and
should therefore reside in the \sphinxstyleemphasis{expansions} package. \sphinxstyleemphasis{BeamSizeMainform} contains a PyQtGraph plot widget. \sphinxstyleemphasis{PlotWidget}
should therefore inherit from \sphinxstyleemphasis{QGraphicsView}.

\sphinxstyleemphasis{QDockWidget} can be triggered from \sphinxstyleemphasis{Main} and contains a widget that could be dynamically changed.

\sphinxstyleemphasis{IntroMainForm} contains a web browser that shows this documentation in PDF\sphinxhyphen{}format.

Inside \sphinxstyleemphasis{tables} package you will find classes that follow the model\sphinxhyphen{}view\sphinxhyphen{}controller design pattern. \sphinxstyleemphasis{TableMainForm}
contains the view, which is QTableView. \sphinxstyleemphasis{InLineEditDelegate} is the controller. All models like \sphinxstyleemphasis{NodeTableModel} inherit
from \sphinxstyleemphasis{QAbstractTableModel}.

\sphinxstyleemphasis{LoadingMainForm} is responsible for triggering the simulation with a dedicated thread \sphinxstyleemphasis{Worker} that runs the simulation
in the background.

\sphinxstyleemphasis{ProjectBrowserMainForm} is an overview of all Models and have a table with a view from \sphinxstyleemphasis{QTableView} that uses the model
\sphinxstyleemphasis{ModelsTableModel}. This UI calls \sphinxstyleemphasis{LoadProjectMainForm} that triggers the loading process in the background using a
thread.


\subsubsection{Applied and Suitable Design Patterns}
\label{\detokenize{implementation:applied-and-suitable-design-patterns}}

\begin{savenotes}\sphinxattablestart
\centering
\begin{tabulary}{\linewidth}[t]{|T|}
\hline
\sphinxstyletheadfamily 
Problem: Multiple table forms can be displayed in the table widget. Each table has its business logic.
\\
\hline
Solution: Implement a Strategy pattern to switch tables dynamically.
\\
\hline
\end{tabulary}
\par
\sphinxattableend\end{savenotes}

\begin{figure}[htbp]
\centering

\noindent\sphinxincludegraphics{{TableStrategy}.jpg}
\end{figure}


\begin{savenotes}\sphinxattablestart
\centering
\begin{tabulary}{\linewidth}[t]{|T|}
\hline
\sphinxstyletheadfamily 
Problem: Too many UI classes for tables causing duplicate code parts. Table classes are also dependent
on the calling class.
\\
\hline
Solution: Implement the MVC pattern separating business logic into models, views, and controllers.
\\
\hline
\end{tabulary}
\par
\sphinxattableend\end{savenotes}

\begin{figure}[htbp]
\centering

\noindent\sphinxincludegraphics{{TableMVC}.jpg}
\end{figure}

The following design pattern could be used and has not yet been implemented.


\begin{savenotes}\sphinxattablestart
\centering
\begin{tabulary}{\linewidth}[t]{|T|}
\hline
\sphinxstyletheadfamily 
Database imports and exports data in different formats using slightly different algorithms.
This causes duplicate code parts and a dependency problem
\\
\hline
Solution: Implement the Template Method design pattern to avoid code duplication by pulling
the duplicate code into a superclass.
\\
\hline
\end{tabulary}
\par
\sphinxattableend\end{savenotes}

\begin{figure}[htbp]
\centering

\noindent\sphinxincludegraphics{{ExportTemplateMethod}.jpg}
\end{figure}


\subsubsection{How The UI Is Built}
\label{\detokenize{implementation:how-the-ui-is-built}}
As can be seen in the figure below, the \sphinxstyleemphasis{.ui} file (which is an XML file) is made using the Qt Designer.
Let’s assume we are formatting the main UI, which is \sphinxstyleemphasis{main.ui}.
This file can be formatted to a python class, \sphinxstyleemphasis{Ui\_MainWindow}, using the pyqt5ac package. Class \sphinxstyleemphasis{Main} is
then connected to \sphinxstyleemphasis{Ui\_MainWindow} and has all the backend functionality of the main UI.

\begin{figure}[htbp]
\centering
\capstart

\noindent\sphinxincludegraphics{{qt}.png}
\caption{Diagram explaining how an exemplary UI class is built.}\label{\detokenize{implementation:id4}}\end{figure}

Main class should be initiated like this.

\begin{sphinxVerbatim}[commandchars=\\\{\}]
\PYG{k}{class} \PYG{n+nc}{Main}\PYG{p}{(}\PYG{n}{QMainWindow}\PYG{p}{)}\PYG{p}{:}
    \PYG{n+nf+fm}{\PYGZus{}\PYGZus{}init\PYGZus{}\PYGZus{}}\PYG{p}{(}\PYG{n+nb+bp}{self}\PYG{p}{)}\PYG{p}{:}
        \PYG{n+nb}{super}\PYG{p}{(}\PYG{p}{)}\PYG{o}{.}\PYG{n+nf+fm}{\PYGZus{}\PYGZus{}init\PYGZus{}\PYGZus{}}\PYG{p}{(}\PYG{p}{)}
        \PYG{n+nb+bp}{self}\PYG{o}{.}\PYG{n}{ui} \PYG{o}{=} \PYG{n}{Ui\PYGZus{}MainWindow}\PYG{p}{(}\PYG{p}{)}
        \PYG{n+nb+bp}{self}\PYG{o}{.}\PYG{n}{ui}\PYG{o}{.}\PYG{n}{setupUi}\PYG{p}{(}\PYG{n+nb+bp}{self}\PYG{p}{)}
\end{sphinxVerbatim}


\subsubsection{The Simulation}
\label{\detokenize{implementation:the-simulation}}
\begin{figure}[htbp]
\centering
\capstart

\noindent\sphinxincludegraphics{{SimulationSteps}.jpg}
\caption{The Simulation is summarized in four steps. Calculating the stiffness matrix for each element, assembling
element stiffness matrices in a global stiffness matrix, solving the systems of equations, and calculating
results.}\label{\detokenize{implementation:id5}}\end{figure}

The following diagram shows how the simulation in Beamon is implemented. Three parts are involved in the Simulation,
\sphinxstyleemphasis{Database}, \sphinxstyleemphasis{Core} and \sphinxstyleemphasis{CALFEM}. Each part has it’s own responsibility. \sphinxstyleemphasis{Database} is responsible for obtaining and
saving data. \sphinxstyleemphasis{Core} is responsible for the calculations implemented in Beamon. \sphinxstyleemphasis{CALFEM} is the Calfem\sphinxhyphen{}Python Package
which is responsible for the most complex calculations.

\begin{figure}[htbp]
\centering
\capstart

\noindent\sphinxincludegraphics{{activity_simulation}.jpg}
\caption{UML activity diagram showing static simulation steps.}\label{\detokenize{implementation:id6}}\end{figure}


\subsection{How TOs}
\label{\detokenize{howto:how-tos}}\label{\detokenize{howto::doc}}
This chapter introduces some best practices and tutorials on how to implement some necessary features in the project.


\subsubsection{How To Build This Documentation}
\label{\detokenize{howto:how-to-build-this-documentation}}
Steps to build sphinx documentation:
\begin{enumerate}
\sphinxsetlistlabels{\arabic}{enumi}{enumii}{}{.}%
\item {} 
Install perl

\item {} 
Install MikTex and latexmk package in MikTex

\item {} 
Install sphinx package in your Python environment

\end{enumerate}

On your environment command line run following commands from project root directory.


\paragraph{Latex Build}
\label{\detokenize{howto:latex-build}}
\begin{sphinxVerbatim}[commandchars=\\\{\}]
\PYG{n}{python} \PYG{o}{\PYGZhy{}}\PYG{n}{m} \PYG{n}{sphinx} \PYG{o}{\PYGZhy{}}\PYG{n}{b} \PYG{n}{latex} \PYG{n}{docs}\PYGZbs{}\PYG{n}{source} \PYG{n}{docs}\PYGZbs{}\PYG{n}{build}\PYGZbs{}\PYG{n}{latex}
\end{sphinxVerbatim}

to build the latex version of the documentation. And then run

\begin{sphinxVerbatim}[commandchars=\\\{\}]
\PYG{n}{cd} \PYG{n}{docs}\PYGZbs{}\PYG{n}{build}\PYGZbs{}\PYG{n}{latex} \PYG{o}{\PYGZam{}}\PYG{o}{\PYGZam{}} \PYG{n}{make}
\end{sphinxVerbatim}

to compile the latex documentation.


\paragraph{HTML Build}
\label{\detokenize{howto:html-build}}
\begin{sphinxVerbatim}[commandchars=\\\{\}]
\PYG{n}{python} \PYG{o}{\PYGZhy{}}\PYG{n}{m} \PYG{n}{sphinx} \PYG{o}{\PYGZhy{}}\PYG{n}{b} \PYG{n}{html} \PYG{n}{docs}\PYGZbs{}\PYG{n}{source} \PYG{n}{docs}\PYGZbs{}\PYG{n}{build}\PYGZbs{}\PYG{n}{html}
\end{sphinxVerbatim}

to build the html version of the documentation.


\paragraph{Epub Build}
\label{\detokenize{howto:epub-build}}
\begin{sphinxVerbatim}[commandchars=\\\{\}]
\PYG{n}{python} \PYG{o}{\PYGZhy{}}\PYG{n}{m} \PYG{n}{sphinx} \PYG{o}{\PYGZhy{}}\PYG{n}{b} \PYG{n}{epub} \PYG{n}{docs}\PYGZbs{}\PYG{n}{source} \PYG{n}{docs}\PYGZbs{}\PYG{n}{build}\PYGZbs{}\PYG{n}{epub}
\end{sphinxVerbatim}


\paragraph{PDF Build}
\label{\detokenize{howto:pdf-build}}
Rhino doesn’t support latex math. That’s why LATEX Build is better.

\begin{sphinxVerbatim}[commandchars=\\\{\}]
\PYG{n}{sphinx}\PYG{o}{\PYGZhy{}}\PYG{n}{build} \PYG{o}{\PYGZhy{}}\PYG{n}{b} \PYG{n}{rinoh} \PYG{n}{docs}\PYGZbs{}\PYG{n}{source} \PYG{n}{docs}\PYGZbs{}\PYG{n}{build}\PYGZbs{}\PYG{n}{pdf}
\end{sphinxVerbatim}


\paragraph{Build All}
\label{\detokenize{howto:build-all}}
To build all documentation versions run the following command on your root directory

\begin{sphinxVerbatim}[commandchars=\\\{\}]
\PYG{n}{make} \PYG{n}{build\PYGZus{}doc}
\end{sphinxVerbatim}


\subsubsection{How to Style QTableView Cells}
\label{\detokenize{howto:how-to-style-qtableview-cells}}
\sphinxurl{https://www.learnpyqt.com/tutorials/qtableview-modelviews-numpy-pandas/}


\renewcommand{\indexname}{Python Module Index}
\begin{sphinxtheindex}
\let\bigletter\sphinxstyleindexlettergroup
\bigletter{b}
\item\relax\sphinxstyleindexentry{beamon.core}\sphinxstyleindexpageref{api:\detokenize{module-beamon.core}}
\item\relax\sphinxstyleindexentry{beamon.database.database}\sphinxstyleindexpageref{api:\detokenize{module-beamon.database.database}}
\item\relax\sphinxstyleindexentry{beamon.simulation}\sphinxstyleindexpageref{api:\detokenize{module-beamon.simulation}}
\item\relax\sphinxstyleindexentry{beamon.ui.callVisualizer}\sphinxstyleindexpageref{api:\detokenize{module-beamon.ui.callVisualizer}}
\item\relax\sphinxstyleindexentry{beamon.ui.main}\sphinxstyleindexpageref{api:\detokenize{module-beamon.ui.main}}
\item\relax\sphinxstyleindexentry{beamon.ui.pyvista\_widget}\sphinxstyleindexpageref{api:\detokenize{module-beamon.ui.pyvista_widget}}
\end{sphinxtheindex}

\renewcommand{\indexname}{Index}
\printindex
\end{document}